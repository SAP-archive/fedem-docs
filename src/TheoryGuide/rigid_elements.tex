% SPDX-FileCopyrightText: 2023 SAP SE
%
% SPDX-License-Identifier: Apache-2.0
%
% This file is part of FEDEM - https://openfedem.org

%%%%%%%%%%%%%%%%%%%%%%%%%%%%%%%%%%%%%%%%%%%%%%%%%%%%%%%%%%%%%%%%%%%%%%%%%%%%%%%%
%
% FEDEM Theory Guide.
%
%%%%%%%%%%%%%%%%%%%%%%%%%%%%%%%%%%%%%%%%%%%%%%%%%%%%%%%%%%%%%%%%%%%%%%%%%%%%%%%%

\section{RBAR}
\label{s:RBAR}

RBAR is a rigid bar element with 2 nodal points and 6 DOFs assigned to each node.
The nodal points located at each end of the bar are numbered 1 and 2 as shown in Figure~\ref{fig:RBAR}.
The nodal DOFs are referred to in the link coordinate system.
The element has no material properties.
The 12 DOFs of the element are related to each other through the following set of equations:
%
\begin{eqnarray}
u_2 &=& u_1\, + e_z\theta_{y1} - e_y\theta_{z1} \label{equ:RBAR1} \\
v_2 &=& v_1\: + e_x\theta_{z1} - e_z\theta_{x1} \\
w_2 &=& w_1   + e_y\theta_{x1} - e_x\theta_{y1} \\
\theta_{x2} &=& \theta_{x1} \\
\theta_{y2} &=& \theta_{y1} \\
\theta_{z2} &=& \theta_{z1} \label{equ:RBAR6}
\end{eqnarray}
%
where $e_x,e_y,e_z$ are the relative distance between the two nodes
in the respective coordinate directions, as depicted in Figure~\ref{fig:RBAR}

\begin{figure}[t]
\begin{center}
\setlength{\unitlength}{1mm}
\begin{picture}(100,65)
\thinlines
\put(15,15){\vector(1,0){70}}
\put(15,15){\vector(0,1){45}}
\put(15,15){\vector(-1,-1){12}}
\put( 0, 0){$x_{\text{link}}$}
\put(87,14){$y_{\text{link}}$}
\put(13,62){$z_{\text{link}}$}
\thicklines
\put(36,30){\circle*{2}}\put(32,30){$1$}
\put(70,50){\circle*{2}}\put(66,50){$2$}
\qbezier(36,30)(53,40)(70,50)
%Eccentric arms
\thinlines
\put(36,30){\line(-1,-1){10}}\put(32,24){$e_x$}
\put(26,20){\line( 1, 0){44}}\put(48,22){$e_y$}
\put(70,20){\line( 0, 1){30}}\put(71,35){$e_z$}
%Nodal dofs
\thinlines
\put(70,50){\vector(-1,-1){4.8}}\put(62,42){$u,\theta_x$}
\put(70,50){\vector( 1, 0){8}}  \put(79,49){$v,\theta_y$}
\put(70,50){\vector( 0, 1){8}}  \put(67,60){$w,\theta_z$}
\end{picture}
\end{center}
\caption{RBAR, Rigid bar element}
\label{fig:RBAR}
\end{figure}

The physical properties of the RBAR element are two sets of component numbers
at each node, identifying the dependent and independent DOFs at the node.
The total number of independent DOFs in the element must be equal to 6 and they
must jointly be capable of representing any general rigid body motion of the element.
The element may have up to 6 dependent DOFs.
If no dependent DOFs are specified, all DOFs that are not specified
as independent will be made dependent.

Since the RBAR element may have dependent DOFs at both nodes, none of the nodes
can be a triad (external node) in Fedem.
If all independent DOFs are gathered at one node (and all the dependent DOFs
are at the other node), the RBAR element is equivalent to a 2-node RGD element,
see Section~\ref{s:RGD}.
During an analysis, rigid bar elements are processed using a DOF elimination method.
The constraint equations~(\ref{equ:RBAR1})--(\ref{equ:RBAR6}) are generated for
each element and are used to eliminate the dependent DOFs from the global
system of equations, before that system is assembled.

\section{RGD}
\label{s:RGD}

RGD is a rigid element with one master node having 6 independent DOFs, 3~translations and 3~rotations.
All remaining nodes are slave nodes having either 3 or 6 dependent DOFs.
Only the master node can be a triad (external node) in Fedem.
The nodal points are numbered 1 through number of nodes as shown in Figure~\ref{fig:RGD}.
A RGD element has no material properties.
The set of dependent DOFs at the slave nodes may optionally be specified as a physical property.
The default behavior if no dependent DOFs are specified is that all DOFs
at the slave nodes are made dependent on the master node DOFs.

\begin{figure}[t]
\begin{center}
\setlength{\unitlength}{1mm}
\begin{picture}(100,65)
\thinlines
\put(15,15){\vector( 1, 0){70}}
\put(15,15){\vector( 0, 1){45}}
\put(15,15){\vector(-1,-1){12}}
\put( 0, 0){$x_{\text{link}}$}
\put(87,14){$y_{\text{link}}$}
\put(13,62){$z_{\text{link}}$}
\thicklines
\put(36,30){\circle{2}} \put(32,27){$1_m$}
\put(80,20){\circle*{2}}\put(82,19){$2_s$}
\put(70,40){\circle*{2}}\put(71,36){$3_s$}
\put(60,60){\circle*{2}}\put(62,60){$4_s$}
\put(30,50){\circle*{2}}\put(29,53){$5_s$}
\qbezier(36,30)(58,25)(80,20)
\qbezier(36,30)(53,35)(70,40)
\qbezier(36,30)(48,45)(60,60)
\qbezier(36,30)(33,40)(30,50)
%Nodal dofs
\thinlines
\put(70,40){\vector(-1,-1){3.6}}\put(64,35){$u$}
\put(70,40){\vector(1,0){6}}    \put(77,39){$v$}
\put(70,40){\vector(0,1){6}}    \put(69,47){$w$}
\end{picture}
\end{center}
\caption{RGD, Multi-node rigid element}
\label{fig:RGD}
\end{figure}

The presence of a rigid element in a model implies that the motion of all the
slave nodes on the element are to be constrained as though they were connected
to the master node by mass less rigid beams (or semi-rigid if not all slave
node DOFs are made dependent).
During an analysis, rigid elements are processed using a DOF elimination method.
A set of constraint equations, equivalent to Equations~(\ref{equ:RBAR1})--(\ref{equ:RBAR6})
is generated for each slave node, relating the dependent DOFs to the
independent DOFs of the master node.
These equations are then used to eliminate all dependent DOFs from the
global system of equations, prior to the system matrix assembly.
Note that a master node in one RGD element may be a slave node in another RGD.
Such chains of RGD elements are resolved explicitly in such a way that all
slave DOFs ultimately are coupled only to independent DOFs that do not depend
on other RGD constraints.
