% SPDX-FileCopyrightText: 2023 SAP SE
%
% SPDX-License-Identifier: Apache-2.0
%
% This file is part of FEDEM - https://openfedem.org

%%%%%%%%%%%%%%%%%%%%%%%%%%%%%%%%%%%%%%%%%%%%%%%%%%%%%%%%%%%%%%%%%%%%%%%%%%%%%%%%
%
% FEDEM Theory Guide.
%
%%%%%%%%%%%%%%%%%%%%%%%%%%%%%%%%%%%%%%%%%%%%%%%%%%%%%%%%%%%%%%%%%%%%%%%%%%%%%%%%

\chapter{Co-rotated Formulation}
\label{chap:SupElCorot}

During mechanism simulation, elements may undergo
significant translational and rotational displacements.
Elements in this context may be superelements, generic parts,
or more ``standard'' elements as described in Chapter~\ref{chap:ElLibrary}.
These large displacements can have both rigid body and deformational components.
To obtain these components from the total displacements, a method called
{\it Shadow Element Configuration} is used, see Figure~\ref{figSC:61}.
In this figure, the configurations are denoted as follows:
%
\begin{namelist}{$C_{0n}$}
\item[$C_0 $]   \mdash~initial (reference) configuration
\item[$C_{0n}$] \mdash~co-rotated shadow element configuration
\item[$C_n $]   \mdash~current (deformed) configuration
\end{namelist}
%
\begin{figure}[b]
\center{
\setlength{\unitlength}{1mm}
\begin{picture}(93,65)
\thinlines
% Global coordinate system
\put( 7,10){\vector(-1,-2){3.5}}\put(1.5, 0){${\mf I}_1$}
\put( 7,10){\vector( 1, 0){ 10}}\put( 19, 9){${\mf I}_2$}
\put( 7,10){\vector( 0, 1){ 10}}\put(  7,21){${\mf I}_3$}
% C0 element and coordinate system
\thinlines
\put( 30,50){\vector(-2,-1){  9}}\put( 18,45){${\mf i}_1^0$}
\put( 30,50){\vector( 3,-4){  6}}\put( 36,43){${\mf i}_2^0$}
\put( 30,50){\vector( 1, 4){2.5}}\put( 32,61){${\mf i}_3^0$}
\thicklines
\put(  7,45){\circle*{1.5}}\put(  0,44){$R2$}
\put( 41,62){\circle*{1.5}}\put( 43,62){$R1$}
\put( 39,37){\circle*{1.5}}\put( 37,32){$R3$}
\qbezier( 7,45)(24,53.5)(41,62)
\qbezier(41,62)(40,49.5)(39,37)
\qbezier(39,37)(23,41)(7,45)
% Cn element and coordinate system
\thinlines
\put( 75,25){\vector(-4,-1){  9}}\put( 62,23){${\mf i}_1^n$}
\put( 75,25){\vector( 1,-2){  3}}\put( 78,20){${\mf i}_2^n$}
\put( 75,25){\vector( 1, 2){  3}}\put( 79,29){${\mf i}_3^n$}
\thicklines
\put( 50,25){\circle*{1.5}}\put( 38,23){$R2$}
\put( 86,34){\circle*{1.5}}\put( 89,38){$R1$}
\put( 82,15){\circle*{1.5}}\put( 81,8){$R3$}
\qbezier[30](50,25)(68,29.5)(86,34)
\qbezier[30](86,34)(84,24.5)(82,15)
\qbezier[30](82,15)(66,20)(50,25)
\put( 45,25){\circle*{1.5}}
\put( 89,36){\circle*{1.5}}
\put( 83,13){\circle*{1.5}}
\qbezier(45,25)(67,30.5)(89,36)
\qbezier(89,36)(86,24.5)(83,13)
\qbezier(83,13)(64,19)(45,25)
\thinlines
\put(50,50){$C_0$ : initial or reference}
\put(35,14){$C_n$ : deformed}
\put(45, 8){$C_{0n}$ : ``shadow''}
%\graphpaper[2](0,0)(100,75)
\end{picture}
}
\caption{Coordinate systems and configurations for an element}
\label{figSC:61}
\end{figure}
%
$C_0$ is the initial, undeformed configuration, and $C_n$ is the deformed
position at step $n$.
The shadow element configuration, $C_{0n}$, is obtained by translating and
rotating the initial configuration $C_0$ as a rigid body in such a way that the
``distance'' between $C_{0n}$ and $C_n$ is minimized.

This rigid body motion moves element nodes from their initial position within
the undeformed configuration, $C_0$, to their position in the shadow element
configuration, $C_{0n}$, whereas the deformational displacement moves the nodes
from their position in the shadow element configuration, $C_{0n}$,
to their positions in the deformed configuration $C_n$.

The local coordinate system of configuration $C_0$ moves and rotates with the
element and becomes the co-rotated coordinate system of $C_{0n}$.
In addition, because the superelement matrices established in
Chapter~\ref{chap:CMS} were defined in the element's local coordinate system,
the matrices remain constant with respect to this co-rotated coordinate system.
Thirdly, the co-rotated coordinate system is common to both $C_{0n}$ and $C_n$.

When establishing the system matrices as described in
Chapter~\ref{c:Dynamics Simulation}, the superelement matrices defined in
Chapter~\ref{chap:CMS} must be transformed to the global coordinates for each
new position of the mechanism during simulation.
This transformation to global coordinates is the subject of the next section.

% SPDX-FileCopyrightText: 2023 SAP SE
%
% SPDX-License-Identifier: Apache-2.0
%
% This file is part of FEDEM - https://openfedem.org

%%%%%%%%%%%%%%%%%%%%%%%%%%%%%%%%%%%%%%%%%%%%%%%%%%%%%%%%%%%%%%%%%%%%%%%%%%%%%%%%
%
% FEDEM Theory Guide.
%
%%%%%%%%%%%%%%%%%%%%%%%%%%%%%%%%%%%%%%%%%%%%%%%%%%%%%%%%%%%%%%%%%%%%%%%%%%%%%%%%

\section{Local element coordinate system}
\label{sec:RefDef}

The element matrices are all expressed in a local element coordinate system.
For superelements this is the coordinate system in which the finite element
mesh was modeled at substructure level, and for generic parts this is simply
the local element coordinate system of the underlying rigid spider.
In the system analysis, this local coordinate system becomes the {\it Element
Co-rotated Coordinate System} (ECCS) because it moves with the element.

Regarding the nonlinear behavior of an element, the co-rotational formulation is
independent of whether the element is a reduced superelement, resulting from the
Component Mode Synthesis reduction described in Section~\ref{sec:CMS reduction},
a generic part element as described in Section~\ref{sec:GenericPart}, or a more
traditional element as the other ones described in Appendix~\ref{chap:ElLibrary}.
In the following description of the co-rotational formulation,
element and superelement are thus interchangeable terms.

To calculate the transformations of the elements at each new position,
the directions for the corresponding ECCS must first be calculated based on the
positions of the element nodes.

%%%%%%%%%%%%%%%%%%%%%%%%%%%%%%%%%%%%%%%%%%%%%%%%%%%%%%%%%%%%%%%%%%%%%%%%%%%%%%%%
\subsection{Method 1: Best fit of max sized triangle}
\label{sec:RefDefMeth1}

In this method, the ECCS is connected to three nodal points for each element.
The goal is to select among all nodal points within an element,
those three that form the largest triangle.
This is achieved by the following algorithm:
%
\begin{enumerate}
\item The first reference point, $R_1$,
is selected as the node furthest away from the centroid of all element nodes.
This first reference point is always at the node itself, i.e., zero offset.
%
\item The second reference point, $R_2$,
is selected as the node furthest away from the point $R_1$.
If the element only has one node the same node is used again,
but with the second reference point defined by an offset of unit length
along the global $x$-axis from the nodal position.
%
\item The third reference point, $R_3$,
is selected as the node furthest from a line through the points $R_1$ and $R_2$.
If the element has only one or two nodes, or the third node is on (or close to)
this straight line\footnote{The criterion for whether the third node is on the
straight line between $R_1$ and $R_2$ can be configured.
Originally, this tolerance was absolute and equal to 0.001 regardless of the
model dimensions. Later, it was changed to half the distance between the first
two points, i.e., $0.5|\vec{R}_{12}|$ but it is now configurable by setting
the relative tolerance parameter $\epsilon$ in the absolute tolerance
$\epsilon|\vec{R}_{12}|$. The default value on $\epsilon$ is now 0.05.},
the reference point is defined using an offset from the
third node so that the three points form a proper triangle\footnote{
The offset on the third node (when needed) was originally selected
simply as a vector of unit length away from the straight line.
This is clearly not a good strategy if the distance between the first two nodes
is much greater (or much less) than 1,
since the resulting triangle then will be far from even-sided.
These shortcomings of the unit offset spurred an improved method where the
offset is scaled with respect to the distance between the first two points.
This results in a more even sided triangle, and a more robust coordinate system.
The original method is available for backward compatibility of older models.}.
The direction of the third-node offset vector is obtained by rotating the vector
$\vec{R}_{12}$ from $R_1$ to $R_2$ 90 degrees about either the global $x$-, $y$-
or $z$-axis, depending on along which global axis direction the $\vec{R}_{12}$
vector has its smallest component.
%
\item A temporary coordinate system is now formed based on the positions
of the three reference points.
The origin is defined at $R_1$,
the $x$-axis is defined in the direction from $R_1$ to $R_2$,
the $z$-axis is next calculated from the cross product of the $x$-axis and the
axis extending from $R_1$ to $R_3$, and finally the $y$-axis is calculated from
the cross product of the $z$- and $x$-axes.
During the analysis, this coordinate system is recalculated for each new
position of the three reference points.
%
\item The position of the triangle system is always fixed in relation to the
ECCS during the analysis.
\end{enumerate}

\clearpage
Using the $4\times4$ position matrix notation defined in \eqnref{eq:posDef}, the
{\underline p}osition matrix, ${\mf P}_{TE}$, relates the {\underline t}riangle
coordinate system to the {\underline e}lement coordinate system:
%
\begin{equation}
{\mf P}_{TE} \;=\; \left[\barr{cc}
{\mf R}_{TE} & {\mf p}_1 \\
{\mf 0} & 1 \earr\right]
\end{equation}
%
where ${\mf R}_{TE}$ is the {\underline r}otation or orientation of the
{\underline t}riangle system measured in the {\underline e}lement system,
and ${\mf p}_1$ is the {\underline p}osition of the triangle system relative
to the element system (${\mf p}_1$ is the position of $R_1$ measured in the
element coordinate system).

It follows that the positions of the reference points measured in global
coordinates establish ${\mf P}_{TG}$, the position of the {\underline t}riangle
coordinate system in relation to the {\underline g}lobal system.
The following relationship then relates the triangle,
element and global coordinate systems:
%
\begin{equation}
{\mf P}_{TG} \;=\; {\mf P}_{EG}{\mf P}_{TE}
\end{equation}
%
where ${\mf P}_{EG}$ is the {\underline e}lement coordinate system relative to
the {\underline g}lobal system, and is stated
%
\begin{equation}
{\mf P}_{EG} \;=\; {\mf P}_{TG}{\mf P}_{TE}^{-1}
\end{equation}

%%%%%%%%%%%%%%%%%%%%%%%%%%%%%%%%%%%%%%%%%%%%%%%%%%%%%%%%%%%%%%%%%%%%%%%%%%%%%%%%
\subsection{Method 2: Mass based weighted average}
\label{sec:RefDefMeth2}

This method ultimately gives an expression requiring a weighted average of the
nodal discrepancies between the actual deformed element and the undeformed
shadow element to be zero.
This expression can be established from an equilibrium point of view.

One assumes that the (rigid) shadow element is attached to the deformed element
with springs, rotational and translational, at each node.
The spring stiffnesses at each node is selected to be, for instance,
proportional to the mass for translation and proportional to the inertia for
the rotation of each node.
When the nodes move, the shadow element will reposition itself according to
equilibrium based on the redistribution of nodal displacements.
The rigid body motion of the shadow element is then represented by the motion of
a fictitious node at the Center of Gravity (CG) of the shadow element.

The equilibrium equations for the shadow element can be established about CG.
This yields six equations for the movement of the shadow element as a function
of the nodal displacements.
The equations end up being non-linear and have to be solved iteratively.

With $k_t$ as the mass-proportional translational stiffness in all three
directions at each node, and $k_r$ as inertia-proportional rotational
stiffness in all three directions, the stiffness of a spring element
with two co-located nodes $i$ and $j$ is given by
%
\begin{equation}
\left[\begin{array}{c}
{\mf f}_i \\ {\mf m}_i \\ {\mf f}_j \\ {\mf m}_j \\
\end{array}
\right]
=\left[\begin{array}{cccc}
 k_t {\mf I}  & {\mf 0}  & -k_t {\mf I}  & {\mf 0} \\
 {\mf 0} &  k_r {\mf I}  & {\mf 0} &  -k_r {\mf I} \\
 -k_t {\mf I}  & {\mf 0}  & k_t {\mf I}  & {\mf 0} \\
 {\mf 0} &  -k_r {\mf I}  & {\mf 0} &  k_r {\mf I} \\
\end{array}\right]
\left[\begin{array}{c}
{\mf v}_i \\ {\tf \theta}_i \\ {\mf v}_j \\ {\tf \theta}_j \end{array}\right]
\label{eqCS2_nodeStiff}
\end{equation}

With node $j$ being a dependent node rigidly attached to the node at \textit{CG},
one can establish the virtual displacement equation
%
\begin{equation}
{\mf v}_j \;=\; {\mf E}_i {\mf v}_\textit{CG}
\end{equation}
or
\begin{equation}
\left[\begin{array}{c}
{\mf v}_j \\ {\tf \theta}_j \end{array}\right] =
\left[\begin{array}{cc}
{\mf 1} & -\widehat{\mf e}_i \\
{\mf 0} & {\mf 1}
\end{array}\right]
\left[\begin{array}{c}
{\mf v}_\textit{CG} \\ {\tf \theta}_\textit{CG} \end{array}\right]
\quad\text{where}\quad
{\mf e}_i = \left[\begin{array}{c}
x_i - x_\textit{CG} \\
y_i - y_\textit{CG} \\
z_i - z_\textit{CG}
\end{array}\right]
\label{eq:rigDispEcc}
\end{equation}

Using the kinematic relationship above in a virtual work expression,
one can establish the stiffness matrix for the spring between node $i$
and the \textit{CG} node as
%
\begin{equation}
\left[\begin{array}{c}
{\mf f}_i \\ {\mf m}_i \\ {\mf f}_\textit{CG} \\ {\mf m}_\text{CG} \\
\end{array}\right]
=
\left[\begin{array}{cccc}
 {\mf k}_t                   &  {\mf 0}    & -{\mf k}_t &  {\mf k}_t \widehat{{\mf e}} \\
 {\mf 0}                     &  {\mf k}_r  &  {\mf 0}   & -{\mf k}_r \\
-{\mf k}_t                   &  {\mf 0}    &  {\mf k}_t & -{\mf k}_t \widehat{{\mf e}} \\
 \widehat{\mf e}^T {\mf k}_t & -{\mf k}_r  & -\widehat{\mf e}^T {\mf k}_t &
 ({\mf k}_r + \widehat{\mf e}^T{\mf k}_T \widehat{\mf e}) \\
\end{array}\right]
\left[\begin{array}{c}
{\mf v}_i \\ {\tf \theta}_i \\ {\mf v}_\textit{CG} \\ {\tf \theta}_\text{CG} \\
\end{array}\right]
\end{equation}
%
where ${\mf k}_t = k_t{\mf I}$ and ${\mf k}_r = k_r{\mf I}$.

The full stiffness matrix for the shadow element is then formed by the assembly
of all nodal contributions.
Rewriting the equation above with a slightly different notation:
%
\begin{equation}
\left[\begin{array}{c}
{\mf f}^i_1 \\ {\mf f}^i_2 \\
\end{array}\right]
=
\left[\begin{array}{cc}
 {\mf k}^i_{11} &  {\mf k}^i_{12} \\
 {\mf k}^i_{21} &  {\mf k}^i_{22} \\
\end{array}\right]
\left[\begin{array}{c}
{\mf v}^i_1 \\ {\mf v}^i_2\\
\end{array}\right]
\end{equation}
%
the assembly for all nodes from $1$ to $n$ (where $n+1$ is the \textit{CG} node)
gives the shadow element stiffness matrix:
%
\begin{equation}
\left[\begin{array}{c}
{\mf f}^1_1 \\ \vdots \\ {\mf f}^i_1 \\ \vdots \\ {\mf f}_\textit{CG} \\
\end{array}\right]
=
\left[\begin{array}{ccccc}
 {\mf k}^1_{11} & \hdots & {\mf 0}        & \hdots   &  {\mf k}^1_{12} \\
 \vdots         & \ddots &                &          &  \vdots         \\
 {\mf 0}        &        & {\mf k}^i_{11} &          &  {\mf k}^i_{12}  \\
 \vdots         &        &                & \ddots   &  \vdots         \\
 {\mf k}^1_{21} & \hdots & {\mf k}^i_{21} & \hdots   &
\left(\sum\limits_{i=1}^n {\mf k}^i_{22}\right) \\
\end{array}\right]
\left[\begin{array}{c}
{\mf v}^1_1 \\ \vdots \\ {\mf v}^i_1 \\ \vdots \\ {\mf v}_\textit{CG} \\
\end{array}\right]
\end{equation}

The equilibrium equation at the \textit{CG} node is seen as the last 6 rows
of the assembled shadow element stiffness matrix, which can be rewritten as
%
\begin{equation}
{\mf f}_\textit{CG} \;=\; \sum_{i=1}^n{\mf k}^i_{21} {\mf v}_i
\quad\text{or}\quad
\left[\begin{array}{c}
{\mf f}_\textit{CG} \\ {\mf m}_\text{CG} \\
\end{array}\right]
\;=\; \sum_{i=1}^n
\left[\begin{array}{cc}
-{\mf k}^i_t               &  {\mf 0}     \\
 \hat{\mf e}^T {\mf k}^i_t & -{\mf k}^i_r \\
\end{array}\right]
\left[\begin{array}{c}
{\mf v}_i \\ {\tf \theta}_i \\
\end{array}\right]
\end{equation}

If the above equation does not yield zero forces and moments
${\mf f}_\textit{CG}$ at the \textit{CG} node when the deformational
\iftoggle{publicedition}{% The following is for the public version
displacement vector is used,
}{% The following is for the in-house edition only
displacement vector ${\mf v}_d$ of \eqnref{eqSC:supelDefDisp} is used,
}
the shadow element is given a corrective translation and rotation that will
again alter the deformation vector for all the nodes in according to
\eqnref{eq:rigDispEcc}, such that the unbalanced forces and moments vanish.

Equilibrium of the \textit{CG} node gives
%
\begin{equation}
{\mf f}_\textit{CG} \;=\;
\sum_{i=1}^n{\mf k}^i_{21}{\mf v}_i =
\sum_{i=1}^n{\mf k}^i_{21}{\mf E}_i\,\Delta{\mf v}_\textit{CG}
\end{equation}
%
Solving this with respect to increment in Shadow element position
$\Delta{\mf v}_\textit{CG}$ gives
%
\begin{equation}
\Delta{\mf v}_\textit{CG} \;=\;
\left(\sum_{i=1}^n{\mf k}^i_{21}{\mf E}_i\right)^{-1}
\sum_{i=1}^n{\mf k}^i_{21}{\mf v}_{d_i}
\label{eq:refSysMeth2}
\end{equation}

The equation above is solved repeatedly until the deformational displacement
vector ${\mf v}_d$ produces no update to the shadow element position given by
$\Delta{\mf v}_\textit{CG}$.


\iftoggle{publicedition}{}{% The following is for the in-house edition only.

\section{Deformational displacements}
\label{s:Deformational displacements}

During mechanism simulation, supernodes move to new positions as a result of the
elastic and rigid body motion of the superelements, see Figure~\ref{figSC:51}.
To calculate the deformations of a superelement, the Element Corotated
Coordinate System (ECCS) transformation matrix ${\mf P}_{SGi}$
(described in the previous section) and its inverse must first be calculated.

\begin{figure}[t]
\center{
\setlength{\unitlength}{1mm}
\begin{picture}(90,70)
\thinlines
% Global coordinate system
\put( 0,0){\vector( 1, 0){ 85}}\put(87,-1){${\mf I}_1$}
\put( 0,0){\vector( 0, 1){ 65}}\put(-1,67){${\mf I}_2$}
\thicklines
% C0 element and nodal vectors
\put(11,11){\vector( 4, 1){8}}
\put(11,11){\vector( -1,4){2}}
\put(61,19){\vector( 4, 1){8}}
\put(61,19){\vector( -1,4){2}}
\qbezier(13,8)(8,7)(7,10)
\qbezier(7,10)(6,13)(11,14)
\qbezier(11,14)(34.5,18)(58,22)
\qbezier(58,22)(63,23)(64,20)
\qbezier(64,20)(65,18)(62,17)
\qbezier(62,17)(37.5,12.5)(13,8)
%
\put( 30.0, 54.0){\vector( 3,-1){8}}
\put( 30.0, 54.0){\vector( 1, 3){3.5}}
\put( 77.3, 36.0){\vector( 3,-1){8}}
\put( 77.3, 36.0){\vector( 1, 3){3.5}}
\qbezier[10]( 30.3, 50.4) ( 25.4, 52.1) ( 26.1, 55.2)
\qbezier[10]( 26.1, 55.2) ( 26.7, 58.3) ( 31.5, 56.6)
\qbezier[60]( 31.5, 56.6) ( 53.9, 48.3) ( 76.2, 40.1)
\qbezier[10]( 76.2, 40.1) ( 81.1, 38.4) ( 80.4, 35.3)
\qbezier[ 8]( 80.4, 35.3) ( 80.3, 33.1) ( 77.2, 33.7)
\qbezier[60]( 77.2, 33.7) ( 53.7, 42.1) ( 30.3, 50.4)
%
\put( 27.0, 55.0){\vector( 1,0){8}}
\put( 27.0, 55.0){\vector( 0,1){8}}
\put( 80.3, 35.0){\vector( 1,-1){6}}
\put( 80.3, 35.0){\vector( 1, 1){6}}
\qbezier( 27.3, 51.4) ( 22.4, 53.1) ( 23.1, 56.2)
\qbezier( 23.1, 56.2) ( 23.7, 59.0) ( 28.5, 58.0)
\qbezier( 28.5, 58.0) ( 55.9, 55.3) ( 80.5, 39.5)
\qbezier( 80.5, 39.5) ( 84.1, 37.2) ( 83.4, 33.0)
\qbezier( 83.4, 34.3) ( 83.3, 31.9) ( 79.2, 32.7)
\qbezier( 79.2, 32.7) ( 51.7, 50.1) ( 27.3, 51.4)
\put(34,6){$C_0$ : initial or reference}
\put(62,52){$C_n$ : deformed}
\put(24,38){$C_{0n}$ : ``shadow''}
%\graphpaper[2](0,0)(100,75)
\end{picture}
}
\caption{Element deformations}
\label{figSC:51}
\end{figure}

Next, the initial positions and orientations ${\mf P}_{NSj}^0$ of the supernodes
(index~$j$) relative to the superelement coordinate system (index~$i$) must be
determined:
%
\begin{equation}
{\mf P}_{NSj}^0 =
\left[\begin{array}{cc}
\tilde{\mf R}_j^0 & \tilde{\mf x}^0 \\
{\mf 0} & 1
\end{array}\right]
\end{equation}
%
where the index~$^0$ represents the initial positions and orientations,
and $\tilde{}$ indicates values measured in the superelement coordinate system.

Using the inverse of the superelement transformation in {\underline g}lobal
coordinates, ${\mf P}_{SGi}^{-1}$, and the supernode position in relation to the
global system, ${\mf P}_{NGj}$, the new position of the supernode $j$ in the
superelement coordinate system, ${\mf P}_{NSj}$, is determined:
%
\begin{equation}
{\mf P}_{NSj} =  {\mf P}_{SGi}^{-1}
{\mf P}_{NGj} = \left[\begin{array}{cc}
\tilde{\mf R}_j & \tilde{\mf x} \\
\mf 0 & 1 \end{array}\right]
\label{eqSC:Png}
\end{equation}
%
where $\tilde{\mf R}_j$ is the rotation/orientation, and $\tilde{\mf x}$ is the
position in the superelement coordinate system.

The rigid body motion is now eliminated by subtracting the initial position from
the current position to form the translational displacement of the node
(measured in the superelement coordinate system) as:
%
\begin{equation}
\tilde{\mf u}_{d_j} = \left[\begin{array}{c}
\tilde{v}_x \\
\tilde{v}_y \\
\tilde{v}_z
\end{array}\right]_j = \left[\begin{array}{c}
\tilde{x} - \tilde{x}^o \\
\tilde{y} - \tilde{y}^o \\
\tilde{z} - \tilde{z}^o
\end{array}\right]_j
\label{eq:defTranslation}
\end{equation}

For the rotations, the present orientation can be achieved from the undeformed
orientation by an incremental rotation $\Delta\tilde{\mf R}_j$ as
%
\begin{equation}
\tilde{\mf R}_j = \Delta\tilde{\mf R}_j\tilde{\mf R}_j^0
\end{equation}
%
from which the incremental rotation is obtained
%
\begin{equation}
\Delta\tilde{\mf R}_j = \tilde{\mf R}_j \tilde{\mf R}_j^{0^T}
%
%\approx
%\left[\begin{array}{ccc}
%{1} & {-\Delta\tilde\gamma} & {\Delta\tilde\beta}  \\
% {\Delta\tilde\gamma} & {1} & {-\Delta\tilde\alpha} \\
% {-\Delta\tilde\beta} & {\Delta \tilde\alpha} & {1}
%\end{array}\right]
%
= \Delta\tilde{\mf R}_j(\tilde{\tf \theta}_{d_j})
\quad\text{where}\quad
\tilde{\tf \theta}_{d_j} = \textbf{axial}(\Delta\tilde{\mf R}_j)
\label{eq:defRotation}
\end{equation}
%
The subscript $d_j$ denotes deformation for node $j$, and $\textbf{axial}(\;)$
represents the inverse relationship of \eqnref{eq:RodriguesDef}.

\subsection{Deformational displacement vector}

The supernode deformation vector, $\tilde{\mf v}_{d_j}$, for node $j$ may now be
collected from the terms in \eqsref{eq:defTranslation}{eq:defRotation}
%
\begin{equation}
\tilde{\mf v}_{d_j} = \left[\begin{array}{c}
\tilde{v}_x \\
\tilde{v}_y \\
\tilde{v}_z \\
\tilde{\theta}_x \\
\tilde{\theta}_y \\
\tilde{\theta}_z
\end{array}\right]_j =
\left[\begin{array}{c}
\tilde{x} - \tilde{x}^0 \\
\tilde{y} - \tilde{y}^0 \\
\tilde{z} - \tilde{z}^0 \\
\tilde{\theta}_{d_x} \\
\tilde{\theta}_{d_y} \\
\tilde{\theta}_{d_z}
\end{array}\right]_j
\label{eqSC:nodeDefDisp}
\end{equation}
%
where $\tilde{x}^0$, $\tilde{y}^0$ and $\tilde{z}^0$ are the undeformed $x$-,
$y$- and $z$-positions of the supernode relative to the superelement coordinate
system.

Repeating the formation of \eqnref{eqSC:nodeDefDisp} for all the supernodes of a
superelement, and noting that the generalized coordinates from the internal
superelement modes of vibration are explicit variables in the simulation,
the superelement deformation vector, $\tilde{\mf v}_d$, may be assembled
(refer to \eqnref{eqCMS:Hdef}) as:
%
\begin{equation}
\left[\begin{array}{c} {\mf v}_e \\ {\mf y} \end{array} \right] =
\left[\begin{array}{*{20}c}
\tilde{\mf v}_{d_1} & \tilde{\mf v}_{d_2} & \cdots & \tilde{\mf v}_{d_l} &
y_1 & y_2 & \cdots & y_s
\end{array}\right]^T
\label{eqSC:supelDefDisp}
\end{equation}
%
In this general case, the superelement has $l$ supernode deformations
calculated from \eqnref{eqSC:nodeDefDisp} and $s$ generalized coordinates.

\section{Superelement internal forces}
\label{sec:SupElIntForce}

The internal superelement force vector is calculated based on the deformational
displacement vector as
%
\begin{equation}
\tilde{\mf f} = \tilde{\mf k}_e \tilde{\mf v}_d
\label{eqSC:forceLocal}
\end{equation}
%
where
%
\begin{namelist}{$\tilde{\mf k}_e$}
\item[$\tilde{\mf f}$]   \mdash~element force vector, superelement system
\item[$\tilde{\mf k}_e$] \mdash~element stiffness matrix, superelement system
\item[$\tilde{\mf v}_d$] \mdash~element displacement vector, superelement system
\end{namelist}
%
and from \eqnref{eqCMS:ReducedModelEq} we have
%
\begin{equation}
\tilde{\mf f} = \left[\begin{array}{c}
\tilde{\mf f}_e \\ {\mf f}_G
\end{array}\right] = \left[\begin{array}{cc}
\tilde{\mf k}_{11} & \tilde{\mf k}_{12} \\
\tilde{\mf k}_{21} & {\mf k}_{22} \\
\end{array} \right] \left[\begin{array}{c}
\tilde{\mf v}_d \\ {\mf y} \end{array} \right]
\label{eqSC:forceFinal}
\end{equation}

\subsection{Transforming internal forces to the global system}

Supernode DOFs are generally expressed in the global coordinate system,
which is the default in Fedem.
However, they can also be expressed in the local coordinate system of the
supernode or another specified supernode.
DOFs referencing to the supernode in this way are said to have
{\it updated local directions}.
Before adding the superelement forces to the system vector,
each node is transformed to either the global or the updated local directions.

Assuming the rotational part of the inverse superelement transformation matrix,
${\mf P}_{SGi}^{-1}$, for superelement $i$ is ${\mf R}^T_{SGi}$, the
transformation for supernode $k$ before adding it to the system matrices is
%
\begin{equation}
{\mf T}_k = {\mf R}_{SGi}^T
\label{eqSC:57}
\end{equation}
%
when global direction is specified for the node.
If the local updated direction for node $k$ is specified as that of supernode
$j$ ($j$ may be the same as node $k$), the transformation ${\mf T}_k$ is
%
\begin{equation}
{\mf T}_k = {\mf R}_{SGi}^T {\mf R}_{NGj}
\label{eqSC:58}
\end{equation}
%
when the rotational part of the position matrix ${\mf P}_{NGj}$ for node $j$
is ${\mf R}_{NGj}$.

The transformation matrix ~${\mf T}_k$ of dimension $3\times3$ is valid for
supernode $k$ for both translation and rotation transformation.
For a superelement with $m$ external nodes and $n$ generalized DOFs,
the overall superelement transformation matrix may be written
%
\begin{equation}
{\mf T}_{Si} = \left\lceil \begin{array}{*{20}c}
{\mf T}_1 & {\mf T}_1 & {\mf T}_2 & {\mf T}_2 & \cdots &
{\mf T}_m & {\mf T}_m & {\mf I}_G \\
\end{array} \right\rfloor
\label{eqSC:59}
\end{equation}
%
where the dimension of ${\mf I}_G$ is $n\!\times\!n$, $n$ being the number of
generalized coordinates for the superelement.
The superelement force vector is now transformed to the directions for the DOFs
at system level through
%
\begin{equation}
{\mf f} = {\mf T}_{Si}^T \tilde{\mf f}
\label{eqSC:forceTrans}
\end{equation}

\section{Superelement tangent stiffness}
\label{s:Superelement tangent stiffness}

The tangent stiffness matrix is said to be consistent with the internal forces
if it relates the variation of internal force, ${\mf f}$, to variations in the
total displacement, ${\mf v}$, i.e.
%
\begin{equation}
\eqalign{
\delta{\mf f} =\;&
\frac{\partial{\mf f}}{\partial{\mf v}} \delta{\mf v} =
{\mf k}\delta{\mf v} \cr =\;&
\delta{\mf T}^T \tilde{\mf k}_e\tilde{\mf v}_d +
{\mf T}^T\delta\tilde{\mf k}_e \tilde{\mf v}_d +
{\mf T}^T\tilde{\mf k}_e \delta\tilde{\mf v}_d =
{\mf k}_G\delta{\mf v} + {\mf k}_M \delta{\mf v}
}\label{eqSC:64}
\end{equation}
%
The parts of \eqnref{eqSC:64} are:
%
\begin{eqnarray}
\delta{\mf T}^T \tilde{\mf k}_e\tilde{\mf v}_d &=& {\mf k}_G \delta{\mf v}
\mbox{~~: described in detail in Section~\ref{s:Rigid rotation gradient matrix}}
\\
{\mf T}^T \delta\tilde{\mf k}_e &=&
{\mf T}^T \frac{\partial\tilde{\mf k}_e}{\partial\tilde{\mf v}_d}
\delta\tilde{\mf v}_d = {\mf 0}
\mbox{~~: since $\tilde{\mf k}_e$ is constant}
\\
{\mf T}^T\tilde{\mf k}_e\delta\tilde{\mf v}_d &=&
{\mf T}^T\tilde{\mf k}_e\frac{\partial\tilde{\mf v}_d}{\partial\tilde{\mf v}}
\delta\tilde{\mf v} =
{\mf T}^T\tilde{\mf k}_e\tilde{\mf P}\delta\tilde{\mf v} \approx
{\mf T}^T\tilde{\mf k}_e\delta\tilde{\mf v} =
{\mf T}^T\tilde{\mf k}_e{\mf T}\delta{\mf v}
\label{eqSC:65}
\end{eqnarray}
%
where
%
\begin{namelist}{${\mf k}_M$}
\item[${\mf f}$]         \mdash~internal force vector
\item[${\mf k}$]         \mdash~tangent stiffness matrix
\item[${\mf k}_G$]       \mdash~rotational geometric stiffness matrix
\item[${\mf k}_M$]       \mdash~material stiffness matrix
\item[$\tilde{\mf k}_e$] \mdash~element stiffness matrix
\item[${\mf v}$]         \mdash~total displacement vector, ordered node by node
\item[$\tilde{\mf v}_d$] \mdash~deformational displacement vector
\item[${\mf T}_n$]       \mdash~transformation matrix from local to global
coordinate system
\end{namelist}

The rotational geometric stiffness term is defined by \eqnref{eqSC:64},
but first the variation of the transformation matrix needs to be established:
%
\begin{equation}
\eqalign{
\delta{\mf T}_n =\;&
\frac{\partial{\mf T}_n}{\partial\tilde{\tf\omega}_i}
\delta\tilde{\mf\omega}_i \cr =\;&
\left[\begin{array}{c}
  {\mf 0}^T \\
 {{\mf i}_3^n}^T \\
-{{\mf i}_2^n}^T
\end{array}\right] \delta\tilde\omega_x +
\left[\begin{array}{c}
-{{\mf i}_3^n}^T \\
  {\mf 0}^T \\
 {{\mf i}_1^n}^T
\end{array}\right] \delta\tilde\omega_y +
\left[\begin{array}{c}
 {{\mf i}_2^n}^T \\
-{{\mf i}_1^n}^T \\
  {\mf 0}^T
\end{array}\right] \delta\tilde\omega_z \cr =\;&
\left[\begin{array}{ccc}
0 & \delta\tilde\omega_z  & -\delta\tilde\omega_y \\
-\delta\tilde\omega_z & 0 & \delta\tilde\omega_x  \\
 \delta\tilde\omega_y & -\delta\tilde\omega_x & 0 \\
\end{array}\right]
\left[\begin{array}{c}
{{\mf i}_1^n}^T \\
{{\mf i}_2^n}^T \\
{{\mf i}_3^n}^T \\
\end{array}\right] \cr =\;&
-\widehat{\delta\tilde{\tf\omega}}{\mf T}_n
}\label{eqSC:68}
\end{equation}
%
and
%
\begin{equation}
\delta{\mf T}_n^T = {\mf T}_n^T \widehat{\delta\tilde{\tf\omega}}
\label{eqSC:69}
\end{equation}
%
where $\delta{\tf\omega}_i$ is the instantaneous rotation axis.

The rotational geometric stiffness matrix can now be established as follows
%
\begin{equation}
\eqalign{
\delta{\mf T}^T   \tilde{\mf f} =\;& \left[\begin{array}{c}
\delta{\mf T}_n^T \tilde{\mf n}_1 \\
\delta{\mf T}_n^T \tilde{\mf m}_1 \\ \vdots \\
\delta{\mf T}_n^T \tilde{\mf n}_N \\
\delta{\mf T}_n^T \tilde{\mf m}_N \\
\end{array}\right] = \left[\begin{array}{c}
{\mf T}_n^T \widehat{\delta\tilde{\tf\omega}}_r \tilde{\mf n}_1 \\
{\mf T}_n^T \widehat{\delta\tilde{\tf\omega}}_r \tilde{\mf m}_1 \\ \vdots \\
{\mf T}_n^T \widehat{\delta\tilde{\tf\omega}}_r \tilde{\mf n}_N \\
{\mf T}_n^T \widehat{\delta\tilde{\tf\omega}}_r \tilde{\mf m}_N
\end{array}\right] \cr =\;& -{\mf T}^T\left[\begin{array}{c}
\widehat{\tilde{\mf n}}_1 \\
\widehat{\tilde{\mf m}}_1 \\ \vdots \\
\widehat{\tilde{\mf n}}_N \\
\widehat{\tilde{\mf m}}_N
\end{array}\right] \delta{\tf\omega}_r \cr =\; &
-{\mf T}^T \tilde{\mf F}_{nm} \delta\tilde{\tf\omega}_r =
-{\mf T}^T \tilde{\mf F}_{nm} \tilde{\mf G} {\mf T}\delta{\mf v} =
{\mf k}_{GR} \delta{\mf v}
}\label{eqSC:610}
\end{equation}
%
with
%
\begin{eqnarray}
{\mf k}_{GR} = -{\mf T}^T \tilde{\mf F}_{nm} \tilde{\mf G} {\mf T} &\text{and}&
\tilde{\mf k}_{GR} = -\tilde{\mf F}_{nm}\tilde{\mf G} \label{eqSC:612} \\[3mm]
%
\tilde{\mf f} = \left[\begin{array}{c}
\tilde{\mf n}_1 \\
\tilde{\mf m}_1 \\ \vdots \\
\tilde{\mf n}_N \\
\tilde{\mf m}_N
\end{array}\right] &\text{and}&
\tilde{\mf F}_{nm} = \left[ \begin{array}{c}
\widehat{\tilde{\mf n}}_1 \\
\widehat{\tilde{\mf m}}_1 \\ \vdots \\
\widehat{\tilde{\mf n}}_N \\
\widehat{\tilde{\mf m}}_N
\end{array}\right] \label{eqSC:614} \\[3mm]
%
\widehat{\delta\tilde{\tf\omega}_r}
\tilde{\mf n}_a = -\widehat{\tilde{\mf n}}_a \delta\tilde{\tf\omega}
&\text{and}&
\delta\tilde{\tf \omega}_r =
\frac{\partial\tilde{\tf\omega}_r}{\partial\tilde{\mf v}}
\delta\tilde{\mf v} = \tilde{\mf G}\delta\tilde{\mf v}
\label{eqSC:RigRotGrad}
\end{eqnarray}
%
where $\delta\tilde{\tf\omega}_r$ is the instantaneous rigid rotation axis.

The expression~\eqref{eqSC:612} is developed next.
While $\tilde{\mf F}_{nm}$ is quite simple, the $\tilde{\mf G}$ matrix must be
expanded, and that is the focus of the following section.

\remark{Geometric stiffness, or stress stiffening, is often associated with
increased transverse stiffness for structures with an in-plane stress state,
such as trampolines, tensioned cables, etc.}
\remark{The geometric stiffness term of \eqnref{eqSC:612} is a rank 3 matrix,
which only adds stiffness contributions to the three rigid body rotations
of the superelement.
It is important to note that the geometric stiffness matrix developed here does
not add stiffness to internal deformations of the superelement.}
\remark{For a pre-tensioned cable with a transverse load at the middle,
the geometric stiffness will not give added stiffness against midpoint
deflection if the cable is modeled using a single superelement.
In this case the deflection is internal to the superelement only, and no rigid
body rotation takes place.
However, if the cable is modeled using two superelements (one for each half of
the cable), the midpoint deflection will give rigid body rotation for both,
and a correct additional stiffness for the midpoint will be achieved.}

\section{Rigid rotation gradient matrix}
\label{s:Rigid rotation gradient matrix}

The $\tilde{\mf G}$ matrix, defined in \eqnref{eqSC:RigRotGrad},
relates the variation of an element's rigid body rotation (the rotation of the
co-rotated element) to the variation of its DOFs, as
%
\begin{equation}
\delta\tilde{\tf\omega}_r =
\frac{\partial\tilde{\tf\omega}_r}{\partial\tilde{\mf v}} \delta\tilde{\mf v} =
\tilde{\mf G}\delta\tilde{\mf v}
\label{eqSC:RigRotGradII}
\end{equation}
%
This gradient relationship is dependent on the positioning algorithm for the
co-rotated coordinate system, as described in Section~\ref{sec:RefDef}.

\subsection{Method 1: Best fit of max sized triangle}

The rigid body rotation of a superelement is controlled by the motion of three
reference points, $R1$, $R2$ and $R3$, as defined in
Section~\ref{sec:RefDefMeth1}.
These reference points define the superelement's co-rotated coordinate system,
see Figure~\ref{figSC:62}.
This coordinate system has its origin at point $R1$; the $x$-axis is $R1-R2$;
and the $z$-axis is perpendicular to the $R1-R2-R3$ plane.

\begin{figure}[t]
\center{
\setlength{\unitlength}{1mm}
\begin{picture}(94,43)
\thinlines
% Global coordinate system
\put( 7,10){\vector(-1,-2){3.5}}\put(1.5, 0){${\mf I}_1$}
\put( 7,10){\vector( 1, 0){ 10}}\put( 19, 9){${\mf I}_2$}
\put( 7,10){\vector( 0, 1){ 10}}\put(  7,21){${\mf I}_3$}
% C0 element and coordinate system
%\thinlines
%\put( 30,50){\vector(-2,-1){  9}}\put( 18,45){${\mf i}_1^0$}
%\put( 30,50){\vector( 3,-4){  6}}\put( 36,43){${\mf i}_2^0$}
%\put( 30,50){\vector( 1, 4){2.5}}\put( 32,61){${\mf i}_3^0$}
%\thicklines
%\put(  7,45){\circle*{1.5}}\put(  1,44){$R2$}
%\put( 41,62){\circle*{1.5}}\put( 43,62){$R1$}
%\put( 39,37){\circle*{1.5}}\put( 37,32){$R3$}
%\qbezier( 7,45)(24,53.5)(41,62)
%\qbezier(41,62)(40,49.5)(39,37)
%\qbezier(39,37)(23,41)(7,45)
% Cn element and coordinate system
\thicklines
\put( 50,25){\circle*{1.5}}\put( 38,23){$R2$}
\put( 86,34){\circle*{1.5}}\put( 89,38){$R1$}
\put( 82,15){\circle*{1.5}}\put( 81,8){$R3$}
\qbezier[30](50,25)(68,29.5)(86,34)
\qbezier[30](86,34)(84,24.5)(82,15)
\qbezier[30](82,15)(66,20)(50,25)
\thicklines
\put(86,42){\circle*{1.5}}\put(86,42){\vector(1,-2){3}}\put( 83,39){${\mf e}_1$}
\put(42,19){\circle*{1.5}}\put(42,19){\vector(1, 2){3}}\put( 45,19){${\mf e}_2$}
\put(45,25){\circle*{1.5}}
\put(89,36){\circle*{1.5}}
\put(83,13){\circle*{1.5}}
\qbezier(45,25)(67,30.5)(89,36)
\qbezier(89,36)(86,24.5)(83,13)
\qbezier(83,13)(64,19)(45,25)
\thinlines
%\put(50,50){$C_0$ : initial or reference}
%\put(35,14){$C_n$ : deformed}
%\put(45, 8){$C_{0n}$ : ``shadow''}
%\graphpaper[2](0,0)(100,80)
\end{picture}
}
\caption{The superelement co-rotated coordinate system;
${\tf e}_i$ is the eccentricity vector from an external control node
to the reference point $Ri$.}
\label{figSC:62}
\end{figure}

The $\tilde{\mf G}$-matrix is established for the triangular element formed by
the reference points, and then transformed to the superelement's external nodes.
The external nodes that control the motion of reference points are also called
{\it control nodes}, as described in Section~\ref{sec:RefDefMeth1}.

\subsubsection{${\mf G}$ in the triangle coordinate system}

The ${\mf G}$-matrix for a triangular element in the coordinate system of the
triangle (denoted by $\hat{\mf G}$), takes a particularly simple form.
Due to this, the relationship
%
\begin{equation}
\delta\hat{\tf \omega}_r =
\frac{\partial\hat{\tf\omega}_r}{\partial\hat{\mf v}}\delta\hat{\mf v} =
\hat{\mf G}\delta\hat{\mf v}
\label{eqSC:RigRotGradTriSys}
\end{equation}
%
is first developed in this coordinate system, and then transformed to the
superelement coordinate system.

Since a variation of the rotational DOFs in ${\mf v}$ has no effect on the
rotation of the triangular element, the rotational DOFs are not taken into
account.
Hence, the $\delta{\mf v}$ vector has nine DOFs, which are ordered node by node.

While maintaining the triangle coordinate system, $\hat{\mf G}$ can be split
into contributions from each node as follows:
%
\begin{eqnarray}
\hat{\mf G} &=& \left[\begin{array}{ccc}
\hat{\mf G}_1 & \hat{\mf G}_2 & \hat{\mf G}_3
\end{array}\right] \label{eqSC:618} \\
%
\hat{\mf G}_1 &=& \frac{1}{2A} \left[\begin{array}{ccc}
0 & 0 & x_{32} \\
0 & 0 & y_{32} \\
0 & -\frac{2A}{x_{21}} & 0
\end{array}\right] \nonumber \\
%
\hat{\mf G}_2 &=& \frac{1}{2A} \left[\begin{array}{ccc}
0 & 0 & x_{13} \\
0 & 0 & y_{13} \\
0 & \frac{2A}{x_{21}} & 0
\end{array}\right] \nonumber \\
%
\hat{\mf G}_3 &=& \frac{1}{2A} \left[\begin{array}{ccc}
0 & 0 & x_{21} \\
0 & 0 & y_{21} \\
0 & 0 & 0
\end{array}\right] \nonumber
\end{eqnarray}
%
where $x$ and $y$ are coordinates for reference element nodes
($R1$, $R2$, and $R3$) in the triangle coordinate system.

In the superelement coordinate system, $\tilde{\mf G}_i$ can now be obtained
from the triangle coordinate system, $\hat{\mf G}_i$:
%
\begin{equation}
\tilde{\mf G}_i = {\mf R}_{TS}\hat{\mf G}_i {\mf R}_{TS}^T
\label{eqSC:619}
\end{equation}
%
To establish the $\tilde{\mf G}$ matrix for the superelement $i$,
a relationship must exist between superelement external DOFs,
$\tilde{\mf v}_{sup}$, and the DOFs for the reference element,
$\tilde{\mf v}_{ref}$, see Figure~\ref{figSC:62}.

As mentioned above, the DOFs in \textit{one} external node must be able to
control the translations of the reference points completely.
The reference point is therefore rigidly connected to the external node.
In other words, both translational \textit{and} rotational DOFs of the external
nodes control the translation of the reference points.
Since the rotational DOFs control the movement of the co-rotated element,
the eccentricity vectors (rigid arms) should be as short as possible to avoid
the introduction of false rotations of the co-rotated element.
The best description is achieved when the reference points and the control nodes
are coincidental.

It is important to choose external nodes whose motion represents the co-rotated
motion of the element, because the rotational geometric stiffness is sensitive
to this factor.

The vector $\tilde{\mf v}_{sup}$ has dimensions $6\times n$, where $n$ is the
number of external nodes; $\tilde{\mf v}_{ref}$ has dimensions $9\times1$.
Both vectors are ordered node by node, and are related through
%
\begin{equation}
\tilde{\mf v}_{ref} = \tilde{\mf M} \tilde{\mf v}_{sup}
\label{eqSC:620}
\end{equation}

The matrix $\tilde{\mf M}$ depends upon which external nodes move the three
reference nodes.
For example, if external node 2 moves $R1$, external node 1 moves $R2$, and
external node 4 moves $R3$ for a 4-node superelement, $\tilde{\mf M}$ becomes
%
\begin{equation}
\tilde{\mf M} = \left[\begin{array}{cccc}
0 & \tilde{\mf m}_1 & 0 & 0 \\
\tilde{\mf m}_2 & 0 & 0 & 0 \\
0 & 0 & 0 & \tilde{\mf m}_3
\end{array}\right]
\label{eqSC:621}
\end{equation}
%
when
%
\begin{equation}
\tilde{\mf m}_i =
\left[\begin{array}{cc} {\mf I} & -\widehat{\mf e}_i \end{array}\right] =
\left[\begin{array}{cccccc}
1 & 0 & 0 &  0      &  e_{iz} & -e_{iy} \\
0 & 1 & 0 & -e_{iz} &  0      & e_{ix}  \\
0 & 0 & 1 &  e_{iy} & -e_{ix} & 0
\end{array}\right]
\label{eqSC:622}
\end{equation}
%
where ${\mf e}_i$ is the position of reference point $i$ relative to node $i$.
In this example, the $\tilde{\mf G}_{sup}$ matrix for the superelement becomes
%
\begin{equation}
\eqalign{
\tilde{\mf G}_{sup} = \tilde{\mf G}_{ref} \tilde{\mf M} =\;&
\left[\begin{array}{ccc}
\tilde{\mf G}_1 & \tilde{\mf G}_2 & \tilde{\mf G}_3
\end{array}\right]\left[\begin{array}{cccc}
0 & \tilde{\mf m}_1 & 0 & 0 \\
\tilde{\mf m}_2 & 0 & 0 & 0 \\
0 & 0 & 0 & \tilde{\mf m}_3
\end{array}\right] \cr =\;&
\left[\begin{array}{*{20}c}
\tilde{\mf G}_2\tilde{\mf m}_2 & \tilde{\mf G}_1 \tilde{\mf m}_1 &
0 & \tilde{\mf G}_3 \tilde{\mf m}_3
\end{array}\right]
}\label{eqSC:623}
\end{equation}
%
where $\tilde{\mf G}_{ref}$ is the $\tilde{\mf G}$ matrix for the reference
element, obtained by combining \eqsref{eqSC:618}{eqSC:619}.

\subsection{Method 2: Mass based weighted average}

The mass based weighted average scheme for positioning the co-rotated coordinate
system resulted in a nonlinear equation system to be solve with respect to
increments in translation and rotation for the co-rotated coordinate system.
From \eqnref{eq:refSysMeth2} we have
%
\begin{equation}
\delta{\mf v}_\textit{CG} =
\left[\sum_{i=1}^n{\mf k}^i_{21}{\mf E}_i\right]^{-1}
\sum_{i=1}^n{\mf k}^i_{21}\,\delta{\mf v}_i
\end{equation}
%
which can be written as
%
\begin{equation}
\left[\begin{array}{c}
\delta{\mf u}_\textit{r} \\ \delta{\tf \omega}_r
\end{array}\right] = \left[\begin{array}{ccccc}
\begin{array}{l} {\mf G}_{t_1} \\ {\mf G}_1 \end{array} &\hdots&
\begin{array}{l} {\mf G}_{t_i} \\ {\mf G}_i \end{array} &\hdots&
\begin{array}{l} {\mf G}_{t_n} \\ {\mf G}_n \end{array}
\end{array}\right]
\left[\begin{array}{c}
\delta{\mf v}_1 \\ \vdots \\ \delta{\mf v}_i \\ \vdots \\ \delta{\mf v}_n
\end{array}\right]
\end{equation}
%
where
%
\begin{equation}
\left[\begin{array}{c} {\mf G}_{t_i} \\ {\mf G}_{i}\end{array}\right] =
\left[\sum_{i=1}^n{\mf k}^i_{21}{\mf E}_i\right]^{-1} {\mf k}^i_{21}
\end{equation}
%
The rotational part of this equation defines the rotational gradient matrix
for this shadow element positioning method.

\section{Substituting into system matrices}
\label{s:Substituting into system matrices}

The force vector and the mass- and stiffness matrices of the superelement are
now denoted $\tilde{\mf f}$, $\tilde{\mf m}$, and $\tilde{\mf k}$, respectively
(refer to \eqnref{eqCMS:ReducedModelEq}).
These superelement matrices are transformed to the directions for the DOFs at
system level through
%
\begin{eqnarray}
{\mf f} &=& {\mf T}_{Si}^T \tilde{\mf f} \\
{\mf m} &=& {\mf T}_{Si}^T \tilde{\mf m} {\mf T}_{Si} \label{eqSC:510} \\
{\mf k} &=& {\mf T}_{Si}^T \tilde{\mf k} {\mf T}_{Si} \label{eqSC:511}
\end{eqnarray}
%
The local superelement matrices $\tilde{\mf f}$, $\tilde{\mf m}$,
and $\tilde{\mf k}$ may be stored in the computer; at each new position of the
mechanism, these matrices are subjected to the geometric transformation given by
\eqsref{eqSC:510}{eqSC:511}, respectively.
The transformed superelement matrices are the added into the system matrices:
%
\begin{eqnarray}
{\mf f} &=& \sum_i {\mf A}_i^T {\mf f}_i \\
{\mf M} &=& \sum_i {\mf A}_i^T {\mf m}_i {\mf A}_i \label{eqSC:512} \\
{\mf K} &=& \sum_i {\mf A}_i^T {\mf k}_i {\mf A}_i \label{eqSC:513}
\end{eqnarray}
%
where ${\mf A}_i$ are incidence matrices that represent the superelement
topology at system level.
The index $i$ spans all superelements.
In a computer program, the summation is performed more efficiently
and no matrix multiplication is actually required.

} % End in-house edition only
