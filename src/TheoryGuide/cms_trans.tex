% SPDX-FileCopyrightText: 2023 SAP SE
%
% SPDX-License-Identifier: Apache-2.0
%
% This file is part of FEDEM - https://openfedem.org

%%%%%%%%%%%%%%%%%%%%%%%%%%%%%%%%%%%%%%%%%%%%%%%%%%%%%%%%%%%%%%%%%%%%%%%%%%%%%%%%
%
% FEDEM Theory Guide.
%
%%%%%%%%%%%%%%%%%%%%%%%%%%%%%%%%%%%%%%%%%%%%%%%%%%%%%%%%%%%%%%%%%%%%%%%%%%%%%%%%

\section{Component mode synthesis reduction}
\label{sec:CMS reduction}

For each substructure modeled, the status codes for the substructure DOFs are set to 1 for
internal DOFs and 2 for external DOFs. With this control information, the system mass and stiffness
matrices are assembled from the corresponding element matrices. The first $n_1$ DOFs of the
system matrices are internal substructure DOFs, and the next $n_2$ DOFs are external.

In symbolic form, the substructure matrices for mass and stiffness can be divided into submatrices,
with index $i$ for internal and $e$ for external as follows:
%
\begin{equation}
{\mf M} \;=
\left[\begin{array}{cc}
{\mf M}_{ii} & {\mf M}_{ie} \\
{\mf M}_{ei} & {\mf M}_{ee} \\
\end{array} \right]
\quad\text{and}\quad
{\mf K} \;=
\left[\begin{array}{cc}
{\mf K}_{ii} & {\mf K}_{ie} \\
{\mf K}_{ei} & {\mf K}_{ee} \\
\end{array} \right]
\end{equation}
%
where ${\mf M}_{xx}$ are mass and ${\mf K}_{xx}$ are stiffness submatrices.

The stiffness relation for the substructure may now be expressed as:
%
\begin{equation}
\left[\begin{array}{cc}
{\mf K}_{ii} & {\mf K}_{ie} \\
{\mf K}_{ei} & {\mf K}_{ee} \\
\end{array}\right] \left[\begin{array}{cc}
{\mf v}_i \\
{\mf v}_e \\
\end{array}\right] = \left[\begin{array}{cc}
{\mf Q}_i \\
{\mf Q}_e \\
\end{array}\right]
\label{eqCMS:41}
\end{equation}
%
where ${\mf v}_x$ is the displacement vector and ${\mf Q}_x$ is the load vector.

\subsection{Static modes}
\label{subsec:Static modes}

The first line of \eqnref{eqCMS:41} may be written:
%
\begin{equation}
\eqalign{
{\mf v}_i \;=\; & {\mf K}_{ii}^{-1} {\mf Q}_i
                - {\mf K}_{ii}^{-1} {\mf K}_{ie} {\mf v}_e \cr =\; &
{\mf v}_i^i + {\mf v}_i^e}
\label{eqCMS:42}
\end{equation}
%
where ${\mf v}_i^i$ represents internal displacements with fixed external DOFs,
and ${\mf v}_i^e$ represents internal displacements as a function of external
displacements.
Examination of each term provides
%
\begin{equation}
{\mf v}_i^i \;=\; {\mf K}_{ii}^{-1} {\mf Q}_i
\label{eqCMS:43}
\end{equation}
%
and
%
\begin{equation}
{\mf v}_i^e \;=\; -{\mf K}_{ii}^{-1}{\mf K}_{ie}{\mf v}_e \;=\; {\mf B}{\mf v}_e
\label{eqCMS:44}
\end{equation}
%
where
%
\begin{equation}
{\mf B} \;=\; -{\mf K}_{ii}^{-1} {\mf K}_{ie}
\label{eqCMS:45}
\end{equation}

The set of nodal displacements $[\;{\mf v}_i^e \;\; {\mf v}_e\;]$ represents the
"exact" solution of static loads and boundary conditions acting on the external
nodes ("exact" in the sense that the model reduction has not in any way changed
the solution in relation to the solution of the full system).

\subsection{Constrained dynamic modes}
\label{secCMS:FixeIntMode}

The substructure matrices can be reduced by CMS transformation to produce a set of modes called
{\it Craig Bampton modes}. This is done by eliminating the internal DOFs as system DOFs, and
replacing them with a limited number of substructure vibration modes called generalized DOFs.
The CMS transformation starts with an eigenvalue analysis of the substructure system matrices
with the external DOFs fixed at zero. The equation for free, undamped vibration of the substructure's
internal DOFs is written:
%
\begin{equation}
{\mf M}_{ii} \ddot{\mf v}_i^i + {\mf K}_{ii} {\mf v}_i^i \;=\; {\mf 0}
\label{eqCMS:46}
\end{equation}

Considering simple harmonic motion, the displacement ${\mf v}_i^i$ may be expressed as:
%
\begin{equation}
{\mf v}_i^i \;=\; {\bs\phi}\,\sin\omega t
\label{eqCMS:47}
\end{equation}
%
where the eigenvector ${\bs\phi}$ is defined by the eigenvalue problem.
%
\begin{equation}
\left({\mf K}_{ii} -\omega^2{\mf M}_{ii}\right) {\bs\phi} \;=\; {\mf 0}
\label{eqCMS:48}
\end{equation}

In a substructure with $n$ active DOFs of which $p$ are external DOFs,
the internal displacements $v_i^i$ can be expressed as:
%
\begin{equation}
{\mf v}_i^i \;=\; \sum_{k=1}^S {\bs\phi}_k \, y_k \;=\; {\bs\Phi}{\mf y}
\label{eqCMS:49}
\end{equation}
%
where $s<n-p$, and
%
\begin{equation}
{\bs\Phi} \;=\; \left[ \begin{array}{cccc}
{\bs\phi}_1 & {\bs\phi}_2 & \cdots & {\bs\phi}_s
\end{array}\right]
\end{equation}
%
is the eigenvector matrix and has dimensions $(n-p)\times s$.

\subsection{Reduced system}
\label{subsec:Reduced system}

The superelement displacements are now expressed by the external DOFs
${\mf v}_e$ and by the new generalized DOFs ${\mf y}$:
%
\begin{equation}
{\mf v} \;= \left[\begin{array}{c}
{\mf v}_e \\ {\mf v}_i
\end{array}\right] = \left[\begin{array}{cc}
{\mf I} & {\mf 0} \\
{\mf B} & {\tf\Phi}
\end{array}\right]
\left[\begin{array}{c}
{\mf v}_e \\ {\mf y}
\end{array}\right] =\; {\mf H} \left[\begin{array}{c}
{\mf v}_e \\ {\mf y}
\end{array}\right]
\label{eqCMS:Hdef}
\end{equation}

Usually only a few of the lowest modes of vibration need to be included to
achieve good results, and this may reduce the size of the problem substantially.
If all eigenmodes are included, $s=n-p$, the CMS transformation is exact.

\iftoggle{publicedition}{}{% The following is for the in-house edition only.

The substructure dynamic equation of motion may be written:
%
\begin{equation}
\label{eqCMS:PartitionedModelEq}
\eqalign{
\left[\begin{array}{cc}
{\mf M}_{ee} & {\mf M}_{ei} \\
{\mf M}_{ie} & {\mf M}_{ii}
\end{array}\right] \left[\begin{array}{cc}
\ddot{\mf v}_e \\
\ddot{\mf v}_i
\end{array}\right] \;+\; &
\left[\begin{array}{cc}
{\mf C}_{ee} & {\mf C}_{ei} \\
{\mf C}_{ie} & {\mf C}_{ii}
\end{array}\right] \left[\begin{array}{c}
\dot{\mf v}_e \\
\dot{\mf v}_i
\end{array}\right] \cr +\; &
\left[\begin{array}{cc}
{\mf K}_{ee} & {\mf K}_{ei} \\
{\mf K}_{ie} & {\mf K}_{ii}
\end{array}\right] \left[\begin{array}{c}
{\mf v}_e \\
{\mf v}_i
\end{array}\right] \;=\;
\left[\begin{array}{c}
{\mf Q}_e \\
{\mf Q}_i
\end{array}\right]}
\end{equation}
%
where ${\mf C}_{xx}$ represents damping.

Combining \eqnref{eqCMS:Hdef} and its first and second time derivatives with
\eqnref{eqCMS:PartitionedModelEq}, and pre-multiplying with ${\mf H}^T$ produces
%
\begin{equation}
\label{eqCMS:ReducedModelEq}
\eqalign{
\left[\begin{array}{cc}
{\mf m}_{11} & {\mf m}_{12} \\
{\mf m}_{21} & {\mf m}_{22}
\end{array}\right]
\left[\begin{array}{c}
\ddot{\mf v}_e \\
\ddot{\mf y}
\end{array}\right] \;+\; &
\left[\begin{array}{cc}
{\mf c}_{11} & {\mf c}_{12} \\
{\mf c}_{21} & {\mf c}_{22}
\end{array}\right] \left[\begin{array}{c}
\dot{\mf v}_e \\
\dot{\mf y}
\end{array}\right] \cr +\; &
\left[\begin{array}{cc}
{\mf k}_{11} & {\mf k}_{12} \\
{\mf k}_{21} & {\mf k}_{22}
\end{array}\right] \left[\begin{array}{c}
{\mf v}_e \\
{\mf y}
\end{array}\right] \;=\;
\left[\begin{array}{c}
{\mf q}_1 \\
{\mf q}_2
\end{array}\right]}
\end{equation}
%
where
%
\begin{eqnarray}
{\mf m}_{11} &=& {\mf M}_{ee} +
{\mf B}^T {\mf M}_{ie} + {\mf M}_{ei} {\mf B} +
{\mf B}^T {\mf M}_{ii} {\mf B}
\label{eqCMS:Msubs}
\\
{\mf m}_{12} &=& {\mf m}_{21}^T \;=\;
{\mf M}_{ei} {\bs\Phi} + {\mf B}^T
{\mf M}_{ii} {\bs\Phi}
\nonumber \\
{\mf m}_{22} &=& {\bs\Phi}^T
{\mf M}_{ii} {\bs\Phi}
\nonumber \\[5mm]
%
{\mf c}_{11} &=& {\mf C}_{ee} +
{\mf B}^T {\mf C}_{ie} + {\mf C}_{ei} {\mf B} +
{\mf B}^T {\mf C}_{ii} {\mf B}
\label{eqCMS:Csubs}\\
{\mf c}_{12} &=&  {\mf c}_{21}^T \;=\;
{\mf C}_{ei} {\bs\Phi} + {\mf B}^T
{\mf C}_{ii} {\bs\Phi}
\nonumber \\
{\mf c}_{22} &=& {\bs\Phi}^T
{\mf C}_{ii} {\bs \Phi }
\nonumber \\[5mm]
%
{\mf k}_{11} &=& {\mf K}_{ee} +
{\mf K}_{ie}^T {\mf B}
\label{eqCMS:Ksubs}\\
{\mf k}_{12} &=& {\mf k}_{21}^T \;=\; {\mf 0}
\nonumber \\
{\mf k}_{22} &=& {\bs \Phi}^T {\mf K}_{ii} {\bs\Phi}
\nonumber \\[5mm]
%
{\mf q}_1 &=& {\mf Q}_e + {\mf B}^T {\mf Q}_i
\label{eqCMS:Qsubs}\\
{\mf q}_2 &=& {\bs\Phi}^T {\mf Q}_i
\nonumber
\end{eqnarray}

The matrix ${\mf m}_{22}$ from \eqnref{eqCMS:Msubs} is diagonal and the
eigenmodes are normalized, so that
%
\begin{equation}
{\mf m}_{22} \;=\; {\mf I} \;=\; {\bs\Phi}^T {\mf M}_{ii} {\bs\Phi}
\label{eqCMS:424}
\end{equation}
%
which is equal to the unit matrix.
Notice that ${\mf m}_{11}$, ${\mf m}_{12}$, and ${\mf m}_{21}$ are not diagonal.

The matrix ${\mf k}_{11}$ from \eqnref{eqCMS:Ksubs} is reduced by the expression
%
\begin{equation}
{\mf k}_{11} \;=\; {\mf K}_{ee} +
{\mf B}^T {\mf K}_{ie} + {\mf K}_{ei} {\mf B} +
{\mf B}^T {\mf K}_{ii} {\mf B}
\label{eqCMS:425}
\end{equation}
%
Expanding the last two terms by \eqnref{eqCMS:45} shows that the terms are
equal, but with opposite signs; therefore, they cancel out of the equation.
The similarity of these terms is due to the symmetric properties of the
stiffness matrix, which produce
%
\begin{equation}
{\mf K}_{ie}^T \;=\; {\mf K}_{ei}
\label{eqCMS:426}
\end{equation}
%
and
%
\begin{equation}
\left({\mf K}_{ii}^{-1}\right)^T \;=\; {\mf K}_{ii}^{-1}
\label{eqCMS:427}
\end{equation}

For the same reasons, the matrices ${\mf k}_{12} = {\mf k}_{21}^T$
are reduced from the equation
%
\begin{equation}
\label{eqCMS:428}
\eqalign{
{\mf k}_{12} \;=\;& {\mf k}_{21}^T \;=\; {\mf K}_{ei} {\bs\Phi} +
{\mf B}^T {\mf K}_{ii} {\bs\Phi} \cr =\; &
{\mf K}_{ei} {\bs\Phi} +
\left(-{\mf K}_{ii}^{-1}{\mf K}_{ie}\right)^T {\mf K}_{ii} {\bs\Phi} \cr =\; &
{\mf K}_{ei} {\bs\Phi} -
{\mf K}_{ie}^T \left({\mf K}_{ii}^{-1}\right)^T
{\mf K}_{ii}{\bs\Phi} = {\mf 0}}
\end{equation}

It may be shown that the stiffness matrix ${\mf k}_{22}$ is diagonal and of a form in which
%
\begin{equation}
{\mf k}_{22} = \left\lceil\begin{array}{cccc}
\omega_1^2 & \omega_2^2 & \cdots & \omega_{n-p}^2
\end{array}\right\rfloor
\label{eqCMS:429}
\end{equation}
%
$\omega_1^2, \omega_2^2, \ldots\omega_{n-p}^2$ are the eigenvalues corresponding
to the eigenmodes of eigenvector matrix $\bs\Phi$.

The substructure matrices reduced by CMS transformation, as shown above,
are the superelement matrices in a standardized form (Craig-Bampton modes).
These matrices will form the basis for the mechanism simulation formulation.

\subsection{Gravitational forces}

Gravitational forces are calculated from unit gravitational acceleration vectors in the substructure's
local $x$-, $y$-, and $z$-directions and denoted ${\mf U}_x$, ${\mf U}_y$, and ${\mf U}_z$,
respectively. This also means that for all DOFs in ${\mf U}_x$ corresponding to $x$-translation,
the acceleration component of ${\mf U}_x$ is set equal to 1, whereas those corresponding to
$y$- and $z$-translation are equal to 0. In addition, for ${\mf U}_y$ and ${\mf U}_z$ the acceleration
components in the $y$- and $z$-directions, respectively, are set equal to 1, while the accelerations
in the other directions are~0.

The gravitational forces ${\mf G}_x$ corresponding to ${\mf U}_x$ are calculate
from (see \eqnref{eqCMS:ReducedModelEq}):
%
\begin{equation}
\left[\begin{array}{cc}
{\mf G}_{xi} \\ {\mf G}_{xe}
\end{array}\right] =
\left[\begin{array}{cc}
{\mf M}_{ii} & {\mf M}_{ie} \\
{\mf M}_{ei} & {\mf M}_{ee}
\end{array}\right]
\left[\begin{array}{cc}
{\mf U}_{xi} \\ {\mf U}_{xe} \\
\end{array} \right] =
\left[\begin{array}{cc}
{\mf M}_{ii}{\mf U}_{xi} + {\mf M}_{ie}{\mf U}_{xe} \\
{\mf M}_{ei}{\mf U}_{xi} + {\mf M}_{ee}{\mf U}_{xe}
\end{array}\right]
\label{eqK35:1}
\end{equation}
%
The forces from unit gravitational acceleration in the $x$-direction are reduced
by CMS-transformation to ${\mf g}_x$ (see \eqnref{eqCMS:PartitionedModelEq}):
%
\begin{equation}
\left[\begin{array}{cc}
{\mf g}_{xe} \\ {\mf g}_{xg}
\end{array}\right]  =
\left[\begin{array}{cc}
{\mf I} & {\mf B}^T \\
{\mf 0} & {\bs\Phi}^T
\end{array}\right]
\left[\begin{array}{cc}
{\mf G}_{xe} \\ {\mf G}_{xi}
\end{array}\right] =
\left[\begin{array}{cc}
{\mf G}_{xe} + {\mf B}^T {\mf G}_{xi} \\
{\bs\Phi}^T {\mf G}_{xi}
\end{array}\right]
\label{eqK35:2}
\end{equation}
%
Expanding by using \eqnref{eqK35:1} results in:
%
\begin{eqnarray}
{\mf g}_{xe} &=&  {\mf M}_{ie}^T {\mf U}_{xi} +
{\mf M}_{ee} {\mf U}_{xe} + {\mf B}^T {\mf M}_{ii}{\mf U}_{xi} +
{\mf B}^T  {\mf M}_{ie} {\mf U}_{xe}
\label{eqK35:3} \\
%
{\mf g}_{xg} &=& {\bs\Phi}^T {\mf M}_{ii}{\mf U}_{xi} +
{\bs\Phi}^T {\mf M}_{ie}{\mf U}_{xe}
\label{eqK35:5}
\end{eqnarray}
%
taking into account the symmetry property ${\mf M}_{ei} = {\mf M}_{ie}^T$.

The corresponding reduced gravitational forces from unit gravitation in the
$y$-direction are easily found by changing indices:
%
\begin{eqnarray}
{\mf g}_{ye} &=& {\mf M}_{ie}^T {\mf U}_{yi} +
{\mf M}_{ee} {\mf U}_{ye} + {\mf B}^T {\mf M}_{ii}{\mf U}_{yi} +
{\mf B}^T {\mf M}_{ie} {\mf U}_{ye}
\label{eqK35:6} \\
%
{\mf g}_{yg} &=& {\bs\Phi}^T {\mf M}_{ii}{\mf U}_{yi} +
{\bs \Phi}^T {\mf M}_{ie}{\mf U}_{ye}
\label{eqK35:7}
\end{eqnarray}
%
and similarly in the $z$-direction:
%
\begin{eqnarray}
{\mf g}_{ze} &=& {\mf M}_{ie}^T {\mf U}_{zi} +
{\mf M}_{ee} {\mf U}_{ze} + {\mf B}^T {\mf M}_{ii}{\mf U}_{zi} +
{\mf B}^T  {\mf M}_{ie} {\mf U}_{ze}
\label{eqK35:8} \\
%
{\mf g}_{zg} &=& {\bs\Phi}^T {\mf M}_{ii}{\mf U}_{zi} +
{\bs \Phi}^T {\mf M}_{ie}{\mf U}_{ze}
\label{eqK35:9}
\end{eqnarray}

The superelement gravitational forces ${\mf g}$ are now calculated from
(see \meqsref{eqK35:3}{eqK35:9}):
%
\begin{equation}
{\mf g} \;=\; a_{g_x}
\left[\begin{array}{cc}
{\mf g}_{xe} \\ {\mf g}_{xg}
\end{array}\right] + a_{g_y}
\left[\begin{array}{cc}
{\mf g}_{ye} \\ {\mf g}_{yg}
\end{array}
\right] + a_{g_z}
\left[\begin{array}{cc}
{\mf g}_{ze} \\ {\mf g}_{zg}
\end{array}\right]
\label{eqK35:11}
\end{equation}
%
where ${\mf a}_g = [\; a_{g_x} \;\; a_{g_y} \;\; a_{g_z}\;]^T$ is the
gravitational acceleration, measured in the superelement coordinate system.

These gravitational forces must now be added to the system force vector,
but only after they have been transformed to the local supernode coordinate
system, if there is one.

} % End in-house edition only
