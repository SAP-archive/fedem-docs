% SPDX-FileCopyrightText: 2023 SAP SE
%
% SPDX-License-Identifier: Apache-2.0
%
% This file is part of FEDEM - https://openfedem.org

%%%%%%%%%%%%%%%%%%%%%%%%%%%%%%%%%%%%%%%%%%%%%%%%%%%%%%%%%%%%%%%%%%%%%%%%%%%%%%%%
%
% FEDEM Theory Guide.
%
%%%%%%%%%%%%%%%%%%%%%%%%%%%%%%%%%%%%%%%%%%%%%%%%%%%%%%%%%%%%%%%%%%%%%%%%%%%%%%%%

\chapter{Flexible Connections}
\label{c:Flexible Connections}

%%%%%%%%%%%%%%%%%%%%%%%%%%%%%%%%%%%%%%%%%%%%%%%%%%%%%%%%%%%%%%%%%%%%%%%%%%%%%%%%
\section{Spring elements}
\label{s:Spring elements}

A spring may be linear with a spring constant, or it may be nonlinear with the
spring stiffness explicitly or implicitly defined by a reference to a function.
The stiffness function is then a function of the spring deflection.
The spring stiffness and associated force can also be scaled by another function
that can be a function of any model variable.
Axial springs are specified between two supernodes in the mechanism model,
while joint springs are specified directly on the joint DOFs.

The actual stiffness expressions depend on whether the stiffness function is
defined explicitly or implicitly, i.e.,
%
\begin{equation}
K_{el}(\delta,\cdot) \;=\; \left\{\begin{array}{ll}
S(\cdot)\left[\, k_0 + f(\delta) \right] &
\text{: $f(\delta)$ is explicit (stiffness function)} \\[1mm]
S(\cdot)\left[ k_0 + \frac{df}{d\delta} \right] &
\text{: $f(\delta)$ is implicit (force function)}
\end{array}\right.
\label{eq:spring stiffness}
\end{equation}
%
where
%
\begin{namelist}{$K_{el}$}
\item[$K_{el}$] : Elastic spring stiffness
\item[$S$]      : Scale function
\item[$k_0$]    : Constant stiffness
\item[$f$]      : Explicit or implicit stiffness function
\item[$\delta$] : Spring deflection, $\delta = l - l_0$,
where $l_0$ is the stress-free length
\end{namelist}
%
The associated spring force is defined by the expression
%
\begin{equation}
F_{el}(\delta,\cdot) \;=\; \left\{\begin{array}{ll}
S(\cdot)\left[k_0\delta + \int\limits_0^\delta f(\hat\delta)d\hat\delta\right] &
\text{: $f(\delta)$ is explicit} \\[5mm]
S(\cdot)\,\left[\,k_0\delta + f(\delta) \,\right] &
\text{: $f(\delta)$ is implicit}
\end{array}\right.
\label{eq:spring force}
\end{equation}
%
where $F_{el}$ is the elastic spring force,
and the other quantities are as above.

\remark{Although the scale function $S(\cdot)$ in
\eqsref{eq:spring stiffness}{eq:spring force} in principle
may be a function of any model variable, you should avoid using one that
directly or indirectly depends on the spring deflection, $\delta$, such as the
spring velocity, $\dot\delta$, for instance.
If such a variable is used, there might be an inconsistency between the spring
stiffness and associated force that may lead to instabilities in the solution,
or poor convergence.
The best is just to let it be some explicit function of time, $t$.}

For joint springs, the stiffness is added directly to the diagonal of
the system matrix on the actual DOF involved.
For axial springs, the stiffness and force act in the direction between
the two end nodes.
Since an axial spring may be rotating, geometric stiffness terms must also be
included, such that the total stiffness matrix of the axial spring becomes
%
\begin{equation}
\label{eq:springK}
{\bf k} \;=\; K\,\bm{i}_1\otimes\bm{i}_1 \;+\;
\frac{F}{l} \left(\bm{i}_2\otimes\bm{i}_2 + \bm{i}_3\otimes\bm{i}_3\right)
\end{equation}
%
where $K$ and $F$ are the current stiffness and force in the spring,
defined through~\eqsref{eq:spring stiffness}{eq:spring force} respectively,
and $l$ is the current spring length.
The local coordinate system of the spring is defined through the basis
$\{\bm{i}_1,\bm{i}_2,\bm{i}_3\}$ where $\bm{i}_1$ goes in the length direction.
The operator $\otimes$ denotes a dyadic product (outer product)
between two vectors.
The $3\times3$ stiffness matrix~\eqref{eq:springK} is then added to the global
system stiffness matrix on the DOFs associated with the two connected nodes.

A spring may be used as a passive element with fixed stress-free length/angle.
However, it may also be used as an active drive element in the mechanism where
the stress-free length varies as a function of time in accordance with an
explicitly defined function.

By specifying springs on joint DOFs, special constraint effects may be imposed,
for instance the free joint may be used in this way to model nonlinear bearings
or rubber bushings.
Effects like end stops, backlash in gears, switches, temperature-dependent
stiffness or other nonlinear effects may easily be modeled in this way.

\subsection{Failure and yield properties}

The nonlinear behavior in a spring may also be modeled by specifying certain
failure and/or yield criteria.
Failure is modeled by using~\eqsref{eq:spring stiffness}{eq:spring force},
only as long as the following conditions are satisfied
%
\begin{equation}
\eqalign{
\delta_{f-} <\;\; & \delta \;\;\;<\; \delta_{f+} \cr
F_{f-} <\;\; & F_{el} <\; F_{f+}
}\label{eq:failure criteria}
\end{equation}
%
where $\delta_{f-}$ and $\delta_{f+}$ define the maximum spring deflection
before the spring fails, on compression and tension, respectively, whereas
$F_{f-}$ and $F_{f+}$ are similar failure limits on the spring force.

When any of the conditions~\eqref{eq:failure criteria} are no longer fulfilled,
the spring state is changed to \textit{failed} and it will no longer contribute
to the system stiffness matrix and force vector of the mechanism.
This has the same effect as choosing a scale function in the stiffness and
force expression~\eqref{eq:spring stiffness} and~\eqref{eq:spring force}
equal to a step function
%
\begin{equation}
S(t) \;=\; \left\{\begin{array}{lcl}
1 &:& t < t_f \\
0 &:& t \ge t_f
\end{array}\right.
\end{equation}
%
where $t_f$ is the time at which the failure occurred.

When a joint spring reaches a failure state, this would normally cause an abrupt
redistribution of the forces in the joint, and in many cases one would like to
signal the failure in the other springs in the same joint too,
when one of its joint springs detects failure.

Hysteresis behavior and permanent deflection after unloading can be introduced
in a spring by specifying a yield criterion, which limits the spring force to
a specified maximum value $F_{y+}$ (on tension),
or minimum value $F_{y-}$ (on compression).
The actual stiffness and force in the spring will then be given by
%
\begin{eqnarray}
K(\delta,\cdot) &=& \left\{\begin{array}{ll}
K_{el}(\delta,\cdot) &: F_{y-} \le F_{el} \le F_{y+} \\[2mm]
0                    &: F_{el} < F_{y-} \text{~~or~~} F_{el} > F_{y+}
\end{array}\right. \\[2mm]
F(\delta,\cdot) &=& \min\left\{\max\left\{ F_{el}(\delta,\cdot),
F_{y-}\right\}, F_{y+}\right\} \label{eq:actual spring force}
\end{eqnarray}
%
where $K_{el}$ and $F_{el}$ are the elastic stiffness and force,
as defined by~\eqsref{eq:spring stiffness}{eq:spring force}, respectively.

When the spring is yielding, i.e., $F_{el} \notin [F_{y-},F_{y+}]$, the
``elastic'' deflection of the spring (the deflection used when evaluating the
stiffness function) is taken as $\delta = l - (l_0 + \delta_y)$,
where $\delta_y$ is the \textit{yield deflection} defined through
%
\begin{equation}
\delta_y(t) \;=\; \int\limits_{t_y}^t \frac{F_{el}-F}{K_{el}} d\hat{t}
\label{eq:yield deflection}
\end{equation}
%
where $t_y$ is the time at which the yielding first started, and $F$ is the
actual spring force defined by~\eqnref{eq:actual spring force}.
When the spring re-enters an elastic state due to unloading, the integrand
of~\eqnref{eq:yield deflection} becomes zero such that $\delta_y$ then
remains constant until the yield limit is reached again.

The yield limits, $F_{y-1}$ and $F_{y+1}$ may be either constant values or
functions of another model variable.
The latter may be used to to model clutch-like behavior in joint springs,
where you can smoothly (or abruptly) engage/disengage the motion constraint
enforced by the spring.

\subsection{Interconnected spring elements}
\label{subs:Interconnected spring}

Springs with a nonlinear force--displacement curve can be used to model certain
typical joint characteristics.
A ``trailer-hitch'' connection has typically a certain play where it has no
stiffness or force resisting displacement.
This can be modeled using a spring with a stiffness--displacement curve where
we have a zero stiffness ``play'' area around zero displacement.
Alternatively, this is achieved with a corresponding force--displacement curve.
However, we seek to have this play and nonlinear force--displacement behavior
to be independent of the displacement direction, when the displacement is taken
in an arbitrary radial direction.
With constant stiffness and independent springs in, for instance, the $X$
and $Y$ directions we will achieve this, but when the stiffness is nonlinear
the force--displacement response gets a distinct ``square'' characteristic when
subjecting the hitch or joint to a rotating force.
This can be circumvented by interconnecting the displacements and
force--displacement curves in the two directions.

The interconnected displacement is defined as
%
\begin{equation}
\bar\delta^i = \sqrt{\sum_j B_{ij}\delta^2_j} \quad\text{where~} B_{ij} =
\left\{\begin{array}{l}
1 : i = j \\
1 : \text{DOF $i$ and $j$ are interconnected} \\
0 : \text{otherwise}
\end{array}\right.
\end{equation}
%
The interpolated displacement $\bar\delta^i$ is used for calculating an
interpolated force based on all the individual DOF springs defined according
to~\eqnref{eq:spring force}
%
\begin{equation}
\bar F^i \;=\; \sum_j \sign(\delta_j) N_{ij} F_j(\bar\delta^i\,\sign(\delta_j))
\end{equation}
%
where
%
\begin{equation}
N_{ij} \;=\; \left(\frac{\delta_j}{\bar\delta^i}\right)^2 \qquad\text{and}\qquad
\sign(\delta_j) = \left\{\begin{array}{rcl}
 1 &:& \delta_j \ge 0 \\
-1 &:& \delta_j < 0
\end{array}\right.
\label{eq:interpolation functions}
\end{equation}

We are using an interpolation of the individual spring functions in order to
handle oval characteristics defined by different stiffness functions for,
for instance, the $X$ and $Y$ DOFs.
The interpolated force value $\bar F^i$ is then decomposed along each of the
interconnected DOFs as
%
\begin{equation}
\bar F^i_j \;=\; c_{ij}\bar F^i \qquad\text{with direction cosines}\qquad
c_{ij} = \frac{\delta_j}{\bar\delta^i}
\label{eq:interconnected force}
\end{equation}
%
\remark{You should avoid using the interconnectivity functionality between
springs with constant stiffness.
No additional cylindrical or spherical behavior is achieved by this,
while reduced stability can be experienced due to $0/0$ terms
in~\eqnref{eq:interconnected force} when displacements are close to zero
and springs are interconnected.}

The (material) stiffness terms for the interconnected springs are obtained in a
similar fashion as the force components, i.e.
%
\begin{equation}
\bar K^{Mi} \;=\; \sum_j N_{ij} K_j(\bar\delta^i\,\sign(\delta_j))
\end{equation}
%
where $N_{ij}$ is defined in~\eqnref{eq:interpolation functions}.
The components of the material stiffness matrix $\bar{\bf K}^{Mi}$
can then be written as
%
\begin{equation}
\label{eq:Kijk}
\bar K^{Mi}_{jk} \;=\; c_{ij} c_{ik} \bar K^{Mi}
\end{equation}

Since the direction of the interpolated spring force, $\bar F^i$,
depends on the current solution state through the direction cosines $c_{ij}$,
geometric stiffness terms are needed in addition to the material
stiffness~\eqref{eq:Kijk}, for the translational components.
Let $\bm{i}_1$ denote the current direction for the interconnected spring, i.e.,
$\bm{i}_1=\{c_{i1},c_{i2},c_{i3}\}^T$ and $\bm{i}_2$ and $\bm{i}_3$ being two
perpendicular unit vectors such that $\{\bm{i}_1,\bm{i}_2,\bm{i}_3\}$ forms an
orthonormal basis.
Assuming the stress-free length of the spring is zero, the total tangent
stiffness matrix for an interconnected translatory spring is then given as
%
\begin{equation}
\label{eq:springKbar3D}
\bar{\bf K}^i \;=\; \bar{\bf K}^{Mi} \:+\: \bar{\bf K}^{Gi}
\quad\mbox{where}\quad \bar{\bf K}^{Gi} \;=\;
\frac{\bar F^i}{\bar\delta^i}
\left(\bm{i}_2\otimes\bm{i}_2 + \bm{i}_3\otimes\bm{i}_3\right)
\end{equation}

The~\eqnref{eq:springKbar3D} holds for translatory springs coupled in all three
coordinate directions (flexible ball joint behavior). If only two of the three
coordinate directions are coupled (flexible revolute joint behavior),
we have
%
\begin{equation}
\label{eq:springKbar2D}
\bar{\bf K}^{Gi} \;=\; \frac{\bar F^i}{\bar\delta^i}
\bm{i}_2\otimes\bm{i}_2
\end{equation}
%
where $\bm{i}_2=\bm{i}_3\times\bm{i}_1$, and $\bm{i}_3$ is the unit vector in
the coordinate direction that is not coupled (rotation axis of the flexible
revolute joint).

\subsection{Global spring elements}
\label{subs:Global spring}

Instead of using an axial spring which has the important property that it
contributes to stress stiffening in the overall system through its geometric
stiffness, an alternative is to specify independent spring properties on one
or several of the global DOFs between two nodes in the system,
or between one node and ground.
This is much the same as using a free joint with all DOFs spring-constrained.
However, in the global spring element the joint variable transformation
outlined in Section~\ref{subs:Free Joint} is not performed and the resulting
stiffness coefficients are added directly into the system stiffness matrix for
the DOFs associated with the two nodes in question.

The updated spring lengths in a global spring element, that are needed for the
computation of the stiffness and force in each DOF, are defined as
%
\begin{eqnarray}
{\mf l} \;=\hskip13pt \left[l_x \; l_y \; l_z\right]^T &=&
{\mf x}_2 - {\mf x}_1 \\
{\tf\alpha} \;=\; \left[\alpha_x \; \alpha_y \; \alpha_z\right]^T &=&
{\rm EulerZYX}({\mf R}_1^T{\mf R}_2)
\end{eqnarray}
%
where ${\mf x}_i$ is the updated global position vector of node $i$, and
${\mf R}_i$ is the orthonormal transformation matrix defining its orientation.
The operator EulerZYX($\cdot$) extracts the Euler $Z$-$Y$-$X$ angles from
the provided orthonormal transformation matrix,
as outlined in Section~\ref{subs:Euler angles extraction}.
The corresponding spring deflections are then taken as
${\tf\delta}={\mf l}-{\mf l}_0$ and ${\tf\theta}={\tf\alpha}-{\tf\alpha}_0$,
with ${\mf l}_0$ and ${\tf\alpha}_0$ being the initial stress free lengths.

\remark{When rotational stiffness is assigned to a global spring, it is
imperative that the orientations of the two nodes are initially identical or
close to each other, and that they are unlikely to exhibit large relative
rotations.
The background for this is that the Euler angle extraction operator is valid
for small rotations only. Actually, the operator is singular for certain
rotation states (i.e., when one have 0/0-situations in some of
the~\meqsref{eq:EulerZ}{eq:EulerX}.
For the same reason, using soft global springs with rotational stiffness may
also render the solution unstable and should be avoided.}

Especially in situations when the distance between the two nodes is large,
global spring elements may yield a more stable solution than using a regular
free joint, since a small rotation change in one end then may correspond to
a large translation change at the other end, due to the long arm.
The global spring element is a system level equivalent to the BUSH element
on the FE link level (see Section~\ref{s:BUSH}) but with the important
distinction that a global spring may account for nonlinear spring properties.

%%%%%%%%%%%%%%%%%%%%%%%%%%%%%%%%%%%%%%%%%%%%%%%%%%%%%%%%%%%%%%%%%%%%%%%%%%%%%%%%
\section{Damper elements}
\label{s:Damper elements}

A damper may be linear with a damping constant, or it may be nonlinear with the
damping coefficient defined by a reference to a function.
The damping coefficient is a function of the damper velocity.
The damping coefficient and associated force can also be scaled by another
function that can be a function of any model variable.
A wide range of energy dissipation effects may be modeled by damper elements.
Axial dampers are specified between two supernodes in the mechanism model,
while joint dampers are specified directly on the joint DOFs.

The actual damping coefficient expression depend on whether the damping
function is defined explicitly or implicitly
%
\begin{equation}
C(v,\cdot) \;=\; \left\{\begin{array}{ll}
S(\cdot)\left[\, C_0 + f(v) \right] &
\text{: $f(v)$ is explicit (damping function)} \\[1mm]
S(\cdot)\left[ C_0 + \frac{df}{dv} \right] &
\text{: $f(v)$ is implicit (force function)}
\end{array}\right.
\label{eq:damping}
\end{equation}
%
where
%
\begin{namelist}{$C_0$}
\item[$C$]   : Damping coefficient
\item[$S$]   : Scale function
\item[$C_0$] : Constant damping
\item[$f$]   : Explicit or implicit damping coefficient function
\item[$v$]   : Damper velocity
\end{namelist}
%
The associated damper force is defined by the expression
%
\begin{equation}
F(v,\cdot) \;=\; \left\{\begin{array}{ll}
S(\cdot) \left[\, C_0 v + \int\limits_0^v f(\hat v) d\hat v \,\right] &
\text{: $f(v)$ is explicit} \\[5mm]
S(\cdot) \,\left[\, C_0 v + f(v) \,\right] &
\text{: $f(v)$ is implicit}
\end{array}\right.
\label{eq:damper force}
\end{equation}
%
where $F$ is the damper force or moment, and the other quantities are as above.

\remark{Although the scale function $S(\cdot)$
in~\eqsref{eq:damping}{eq:damper force} in principle may be a function of any
model variable, you should avoid using one that directly or indirectly depends
on the damper velocity, $c$, such as the damper deflection, for instance.
If such a variable is used, there will be an inconsistency between the damping
coefficient and associated force that may lead to solution instabilities
or poor convergence.
The best is just to let it be some explicit function of time, $t$.}

For joint dampers, the damping coefficient is added directly to the diagonal of
the system damping matrix on the actual DOF involved.
Axial dampers are handled in a similar way as axial springs, that is,
they result both in an damping matrix and geometric stiffness matrix to be
added into the corresponding system matrices for the nodal DOFs involved:
%
\begin{eqnarray}
{\bf c} &=& C\,\bm{i}_1\otimes\bm{i}_1 \\
{\bf k}_g &=& \frac{F}{l} \left(\bm{i}_2\otimes\bm{i}_2 + \bm{i}_3\otimes\bm{i}_3\right)
\end{eqnarray}
%
where $C$ and $F$ are the current damping coefficient and damper force defined
through~\eqsref{eq:damping}{eq:damper force} respectively,
and $l$ is the damper length.

By specifying dampers on joint DOFs, special constraint effects may be imposed,
for instance the free joint may be used in this way to model damping in
nonlinear bearings or rubber bushings.
Effects like switches, temperature dependent damping or other nonlinear
effects may be modeled in this way.

%%%%%%%%%%%%%%%%%%%%%%%%%%%%%%%%%%%%%%%%%%%%%%%%%%%%%%%%%%%%%%%%%%%%%%%%%%%%%%%%
\section{Spring-constrained joints}
\label{s:Spring-constrained joints}

All joint variables for the master-slave based joints, such as revolute joints,
ball joints, prismatic joints, etc., may be constrained by springs.
The free joint has no degree of freedom constraint and is usually the basis for
modeling spring-constrained joints.
The reason for making joints spring-constrained can be to model the flexibility
of an actual bearing or model nonlinear effects like clearings, rubber bushings,
etc.
To be able to easily switch between master-slave constrained joints and
spring-constrained joints, the position and orientation of triads for
spring-constrained joints should be the same as for the corresponding
master-slave constrained joint.

The {\it flexible revolute joint} is usually modeled from a free joint with
relative constraining springs on the $x$, $y$ and $z$ translational variables
and on the $x$ and $y$ rotational joint variables.
The rotation about the $z$-axis will be the joint variable.
The stiffness of the springs on the different joint variables should comply with
the actual translational and rotational stiffness of the bearing.
Nonlinear stiffness like in rubber bushings or in bearings with clearing should
be modeled by nonlinear springs.
The translatory springs in the local $x$ and $y$ directions may need to be
interconnected, as described in Section~\ref{subs:Interconnected spring},
in order to achieve equal behavior for any load direction in the joint.

The {\it flexible ball joint} will also usually be based on a free joint and
with constraining springs for the $x$, $y$ and $z$ translational variables.
The stiffness for the springs will be set according to joint stiffness and
possible nonlinearities as for the revolute joint above.
All three translatory springs may need to be interconnected, as described in
Section~\ref{subs:Interconnected spring}, in order to achieve equal
behavior for any load direction in the joint.

The {\it flexible prismatic joint} will usually be modeled from one or more
free joints.
The $z$-axis of the joint triads should be along the direction of the joint
variable, the joint axis.
If the flexible prismatic joint is based on only one free joint, all joint
variables should have constraining springs except the variable along the joint
axis, that is the $z$ direction of the joint.
If two or more free joints are used, the orientation of the corresponding triads
should be the same and with the $z$-direction along the joint axis.
For the free joints constituting the flexible prismatic joint, the joint
variables in $x$ and $y$ directions and rotation about $z$ should have
constraining springs.
The rotation variables about the $x$ and $y$ directions for the actual free
joints will usually have no constraining springs, see the master-slave based
prismatic joint.
Translational clearing in the joint could be modeled by nonlinear constraining
springs for $x$ and/or $y$ translation and rotational clearing about the
$z$-axis is modeled by nonlinear constraining springs for the $z$ rotation.

The {\it flexible cylindric joint} is modeled in a very similar way as the
flexible prismatic joint described above.
However, there is no constraining spring for the $z$ rotation.
This joint will have two joint variables, the translation along and the rotation
about the joint axis that is the $z$-axis of the triads.

Besides the options for modeling constraints mentioned above, you have almost
unlimited options for modeling different constrained motions.
However, you should be aware of what effects could result from the fact that
rotations in space do not commute, see Section~\ref{subs:Euler angles}.
