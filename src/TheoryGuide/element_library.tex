% SPDX-FileCopyrightText: 2023 SAP SE
%
% SPDX-License-Identifier: Apache-2.0
%
% This file is part of FEDEM - https://openfedem.org

%%%%%%%%%%%%%%%%%%%%%%%%%%%%%%%%%%%%%%%%%%%%%%%%%%%%%%%%%%%%%%%%%%%%%%%%%%%%%%%%
%
% FEDEM Theory Guide.
%
%%%%%%%%%%%%%%%%%%%%%%%%%%%%%%%%%%%%%%%%%%%%%%%%%%%%%%%%%%%%%%%%%%%%%%%%%%%%%%%%

\chapter{Finite Element Library}
\label{chap:ElLibrary}

The finite element library in Fedem contains the element types listed in
Table~\ref{tab:element library}.
The BUSH, SPRING and RSPRING elements are mass less and contribute to the
stiffness matrix only,
whereas the CMASS element contributes to the mass matrix only.
The RBAR, RGD and WAVGM elements have neither mass nor stiffness on their own.
They are constraint elements, i.e., they introduce linear couplings between
the degrees of freedom of the other elements in various manners.
All the other elements listed in Table~\ref{tab:element library} are standard
linearized finite elements with both stiffness and mass.

A brief description of these basic elements is given in the following sections.
In addition, the formulation of the Generic part element is given in
Section~\ref{sec:GenericPart}.
The latter element is used to represent a link in the Fedem Dynamics Solver,
when a proper finite element representation is not available or has not yet
been established (see the \FedemUG, {\em Section~4.1 ``Links''}).

\begin{table}[b]
\medskip
\begin{center}
\begin{tabular}{ll}
\noalign{\hrule\smallskip}
Element name & Element description \\
\noalign{\smallskip\hrule\smallskip}
FFT3   &  3-node triangular shell element \\
FFQ4   &  4-node quadrilateral shell element \\
TET4   &  4-node isoparametric tetrahedron solid element \\
TET10  & 10-node isoparametric tetrahedron solid element \\
WEDG6  &  6-node isoparametric prismatic solid element \\
WEDG15 & 15-node isoparametric prismatic solid element \\
HEX8   &   Linear isoparametric hexahedron solid element \\
HEX20  &Quadratic isoparametric hexahedron solid element \\
BEAM2  & 2-node linear beam element, also used for spot welds \\
BUSH   & 2-node bushing element (generalized spring) \\
SPRING & 2-node translatory spring \\
RSPRING& 2-node rotational spring \\
CMASS  & Single-node concentrated mass element \\
RBAR   & 2-node rigid bar \\
RGD    & Multi-node rigid body \\
WAVGM  & Multi-node weighted averaged motion element \\
\noalign{\smallskip\hrule}
\end{tabular}
\end{center}
\caption{Fedem element library.}
\label{tab:element library}
\end{table}

\section{FFT3}
\label{s:FFT3}

The 3-node triangular shell element FFT3 is composed of a
triangular element for plate bending by Allman~\cite{AllmanTriPlate},
and a triangular membrane element with rotational Degree of Freedom (DOF) by
Bergan and Felippa~\cite{BerganFelippaTri}.
The membrane part of FFT3 has 9 DOFs;
6 corner translations, and 3 corner normal rotations.
The element is coordinate invariant and passes the patch test for any geometry.

The element performs significantly better than the constant strain triangle.
Because of the presence of the normal rotation DOFs, FFT3 is well
suited to modeling general shell structures.

The FFT3 element is referred to in a global Cartesian coordinate system
as shown in Figure~\ref{fig:FFT3}.
The element nodes are numbered 1-2-3, and a local coordinate system $(x,y,z)$
is defined in such a way that the $x-y$ plane coincides with the middle surface
of the element.
Origin is taken at node 1, and positive $x$-axis coincides with edge 1-2,
while positive y-axis is taken in direction from the $x$-axis toward node 3.
Positive $z$-axis is defined in such a way that $x,y$ and $z$ form a
right-handed coordinate system.
The nodal DOFs are $u,v,w,r_x,r_y$ and $r_z$.

\begin{figure}[b]
\center{
\setlength{\unitlength}{1mm}
\begin{picture}(100,71)
\thinlines
\put( 15,15){\vector(1,0){70}}
\put( 15,15){\vector(0,1){55}}
\put( 15,15){\vector(-1,-1){12}}
\put(  0, 0){$x_{\text{link}}$}
\put( 87,14){$y_{\text{link}}$}
\put( 13,72){$z_{\text{link}}$}
\thicklines
\put( 34,21){\circle{2}}\put( 31,18){$1$}
\put( 76,42){\circle{2}}\put( 80,40){$2$}
\put( 42,52){\circle{2}}\put( 43,48){$3$}
\qbezier(34,21)(55,31.5)(76,42)
\qbezier(76,42)(59,47)(42,52)
\qbezier(42,52)(38,36.5)(34,21)
\thinlines
%Nodal dofs
\put(42,52)      {\vector(2,1){5}}
\put(48,55)      {\vector(2,1){5}}
\put(49.07,55.54){\vector(2,1){5}}
\put(46,52){$u$}\put(51,54){$r_x$}
\put(42,52)      {\vector(-1,4){1.2}}
\put(40.5,58)    {\vector(-1,4){1.2}}
\put(40.21,59.16){\vector(-1,4){1.2}}
\put(42,55){$v$}\put(40.5,61){$r_y$}
\put(42,52)      {\vector(-2,1){4}}
\put(36,55)      {\vector(-2,1){4}}
\put(34.93,55.54){\vector(-2,1){4}}
\put(35.5,51.5){$w$}\put(29,54){$r_z$}
%Coordinate system for element
\put(76,42){\vector(2,1){10}}\put(87,46){$x_{\text{elem}}$}
\put(35,21){\vector(-1,4){7}}\put(20,46){$y_{\text{elem}}$}
\put(35,21){\vector(-2,1){13}}\put(16,30){$z_{\text{elem}}$}
\end{picture}
}
\caption{FFT3, Flat triangular shell element}
\label{fig:FFT3}
\end{figure}

\section{FFQ4}
\label{s:FFQ4}

The 4-node quadrilateral shell element FFQ4 is composed of a
Quadrilateral plate Bending Element with Shear deformation (QBESH)
and a Quadrilateral Membrane element with Rotational degrees of Freedom (QMRF).
The QMRF element has 12 DOFs; 8 corner translations $(u,v)$,
and 4 corner normal rotations $(r_z)$.
The element is coordinate invariant and passes the patch test for any geometry.
The element performance is significantly better than that of the
linear isoparametric quadrilateral.
Because of the presence of the normal rotation DOFs,
QMRF is well suited to modeling general shell structures.
The QBESH element is a quadrilateral plate-bending element that passes the
individual element test.
The element has 12 DOFs; 3 DOFs $(w,r_x,r_y)$ at each of the 4 nodes.
Transverse shear deformation is included in the formulation.

The FFQ4 element is capable of handling warped element geometries by utilizing
projection techniques that restore force equilibrium and correct rigid body
motion (see~\cite{RankinNourOmid} and~\cite{BH:phd}).

\begin{figure}[b]
\center{
\setlength{\unitlength}{1mm}
\begin{picture}(100,71)
\thinlines
\put( 15,15){\vector(1,0){70}}
\put( 15,15){\vector(0,1){55}}
\put( 15,15){\vector(-1,-1){12}}
\put(  0, 0){$x_{\text{link}}$}
\put( 87,14){$y_{\text{link}}$}
\put( 13,72){$z_{\text{link}}$}
\thicklines
\put( 34,21){\circle{2}}\put( 31,18){$1$}
\put( 76,42){\circle{2}}\put( 82,41){$2$}
\put( 72,62){\circle{2}}\put( 68,57){$3$}
\put( 42,52){\circle{2}}\put( 39,53){$4$}
\qbezier(34,21)(55,31.5)(76,42)
\qbezier(76,42)(74,52)(72,62)
\qbezier(72,62)(57,57)(42,52)
\qbezier(42,52)(38,36.5)(34,21)
\thinlines
%Nodal dofs
\put(72,62)      {\vector(2,1){5}}
\put(78,65)      {\vector(2,1){5}}
\put(79.07,65.54){\vector(2,1){5}}
\put(75,61){$u$}\put(81,64){$r_x$}
\put(72,62)      {\vector(-1,4){1.2}}
\put(70.5,68)    {\vector(-1,4){1.2}}
\put(70.21,69.16){\vector(-1,4){1.2}}
\put(72,65){$v$}\put(70.5,71){$r_y$}
\put(72,62)      {\vector(-2,1){4}}
\put(66,65)      {\vector(-2,1){4}}
\put(64.93,65.54){\vector(-2,1){4}}
\put(65.5,61.5){$w$}\put(59,64){$r_z$}
%Coordinate system for element
\put(76,42){\vector(2,1){10}}\put(87,46){$x_{\text{elem}}$}
\put(35,21){\vector(-1,4){8}}\put(25,56){$y_{\text{elem}}$}
\put(35,21){\vector(-2,1){13}}\put(16,30){$z_{\text{elem}}$}
\end{picture}
}
\caption{FFQ4, Flat quadrilateral shell element}
\label{fig:FFQ4}
\end{figure}

The FFQ4 element is referred to in a Cartesian link coordinate system
as shown in Figure~\ref{fig:FFQ4}.
The element nodes are numbered counter-clockwise $1$-$2$-$3$-$4$.
The nodal DOFs are $u, v, w, r_x, r_y,$ and $r_z$.

\section{TET4}
\label{s:TET4}

TET4 is a solid constant strain tetrahedron element with 4 nodes and 12 DOFs.
The nodal points at the corners of the tetrahedron are numbered 1 through 4
as shown in Figure~\ref{fig:TET4}.

\begin{figure}[p]
\center{
\setlength{\unitlength}{1mm}
\begin{picture}(100,75)
\thinlines
\put( 15,15){\vector(1,0){70}}
\put( 15,15){\vector(0,1){55}}
\put( 15,15){\vector(-1,-1){12}}
\put(  0, 0){$x_{\text{link}}$}
\put( 87,14){$y_{\text{link}}$}
\put( 13,72){$z_{\text{link}}$}
\thicklines
\put( 25,25){\circle{2}}\put( 22,22){$1$}
\put( 65,21){\circle{2}}\put( 66,17){$2$}
\put( 60,40){\circle{2}}\put( 63,40){$3$}
\put( 42,62){\circle{2}}\put( 41,65){$4$}
\qbezier    (25,25)(45,23)    (65,21)
\qbezier    (25,25)(33.5,43.5)(42,62)
\qbezier[30](25,25)(42.5,32.5)(60,40)
\qbezier    (65,21)(53.5,41.5)(42,62)
\qbezier    (65,21)(62.5,30.5)(60,40)
\qbezier    (60,40)(51,51)    (42,62)
%Nodal dofs
\thinlines
\put(65,21){\vector(1,0){6}}  \put(73,20){$v$}
\put(65,21){\vector(0,1){6}}  \put(64,29){$w$}
\put(65,21){\vector(-1,-1){3.6}}\put(58,17){$u$}
\end{picture}
}
\caption{TET4, Constant strain tetrahedron element}
\label{fig:TET4}
\end{figure}

\section{TET10}
\label{s:TET10}

TET10 is an isoparametric tetrahedron element with 30 DOFs; 3 DOFs $(u,v,w)$
at each of the 10 nodes (4 corner nodes and 6 mid-edge nodes).
The edges may be straight or curved.
Figure~\ref{fig:TET10} shows a typical element and its local node numbering.
The local node numbering 1 through 10 must be carried out in a right-hand
direction with nodes 1, 7 and 10 on the same edge.

\begin{figure}[p]
\center{
\setlength{\unitlength}{1mm}
\begin{picture}(100,75)
\thinlines
\put( 15,15){\vector(1,0){70}}
\put( 15,15){\vector(0,1){55}}
\put( 15,15){\vector(-1,-1){12}}
\put(  0, 0){$x_{\text{link}}$}
\put( 87,14){$y_{\text{link}}$}
\put( 13,72){$z_{\text{link}}$}
\thicklines
\put( 25,25){\circle{2}}\put( 22,22){$1$}
\put( 65,21){\circle{2}}\put( 65,17){$3$}
\put( 60,40){\circle{2}}\put( 63,40){$5$}
\put( 42,62){\circle{2}}\put( 41,65){$10$}
\put( 44,19){\circle{2}}\put( 41,16){$2$}
\put( 62,30){\circle{2}}\put( 64,31){$4$}
\put( 43,36){\circle{2}}\put( 41,38){$6$}
\put( 31.5,42){\circle{2}}\put( 28,41){$7$}
\put( 51,42.5){\circle{2}}\put( 53,42){$8$}
\put( 52,49.5){\circle{2}}\put( 54,49){$9$}
\qbezier(25,25)(45,15)(65,21)
\qbezier(25,25)(31,44)(42,62)
\qbezier[30](25,25)(40,38)(60,40)
\qbezier(65,21)(50,40)(42,62)
\qbezier(65,21)(61,30)(60,40)
\qbezier(60,40)(50,51)(42,62)
\thinlines
%Nodal dofs
\thinlines
\put(65,21){\vector(1,0){6}}  \put(73,20){$v$}
\put(65,21){\vector(0,1){6}}  \put(64,28){$w$}
\put(65,21){\vector(-1,-1){3.6}}\put(58,17){$u$}
%\graphpaper[2](0,0)(100,80)
\end{picture}
}
\caption{TET10, Isoparametric tetrahedron element}
\label{fig:TET10}
\end{figure}

\section{WEDG6}
\label{s:WEDG6}

WEDG6 is an isoparametric triangular prism element with 18 DOFs;
3 DOFs $(u,v,w)$ at each of the 6 corner nodes.
%
\begin{figure}[p]
\center{
\setlength{\unitlength}{1mm}
\begin{picture}(100,75)
\thinlines
\put( 15,15){\vector(1,0){70}}
\put( 15,15){\vector(0,1){55}}
\put( 15,15){\vector(-1,-1){12}}
\put(  0, 0){$x_{\text{link}}$}
\put( 87,14){$y_{\text{link}}$}
\put( 13,72){$z_{\text{link}}$}
\thicklines
\put( 24,24){\circle{2}}\put( 21,21){$1$}
\put( 64,22){\circle{2}}\put( 64,18){$2$}
\put( 56,40){\circle{2}}\put( 53,41){$3$}
\put( 34,60){\circle{2}}\put( 30,58){$4$}
\put( 74,52){\circle{2}}\put( 77,50){$5$}
\put( 66,70){\circle{2}}\put( 63,71){$6$}
\qbezier    (24,24)(44,23)(64,22)
\qbezier[20](64,22)(60,31)(56,40)
\qbezier[30](56,40)(40,32)(24,24)
\qbezier    (34,60)(54,56)(74,52)
\qbezier    (74,52)(70,61)(66,70)
\qbezier    (66,70)(45,64)(34,60)
\qbezier    (24,24)(29,42)(34,60)
\qbezier    (64,22)(69,37)(74,52)
\qbezier[30](56,40)(61,55)(66,70)
%Nodal dofs
\thinlines
\put(64,22){\vector(1,0){6}}  \put(72,21){$v$}
\put(64,22){\vector(0,1){6}}  \put(63,30){$w$}
\put(64,22){\vector(-1,-1){3.6}}\put(57,18){$u$}
%\graphpaper[2](0,0)(100,80)
\end{picture}
}
\caption{WEDG6, Isoparametric triangular prismatic element}
\label{fig:WEDG6}
\end{figure}
%
The edges are straight, and Figure~\ref{fig:WEDG6} shows a typical
element with local node numbering.
The local node numbering 1 through 6 must be carried out in a right-hand
direction, with nodes 1 and 4 on the same edge.
The WEDG6 element has constant strains for triangular cross-sections, and
a limited linear variation of the strains in the direction transverse to the
triangular side.

\section{WEDG15}
\label{s:WEDG15}

WEDG15 is an isoparametric triangular prism element with 45 DOFs;
3 DOFs $(u,v,w)$ at each of the 15 nodes (6 corner and 9 mid-edge nodes).
The edges may be straight or curved.
Figure~\ref{fig:WEDG15} shows a typical element with local node numbering.
The local node numbering 1 through 15 must be carried out in a right-hand
direction with nodes 1, 7 and 10 on the same edge.
WEDG15 has stresses with minimum linear variation; in addition,
it has an incomplete quadratic variation of stresses in the direction transverse
to the triangular side.
Some stress components also have a quadratic variation in the directions
parallel to the triangular side, for example, the transverse stress component.

\begin{figure}[p]
\center{
\setlength{\unitlength}{1mm}
\begin{picture}(100,75)
\thinlines
\put( 15,15){\vector(1,0){70}}
\put( 15,15){\vector(0,1){55}}
\put( 15,15){\vector(-1,-1){12}}
\put(  0, 0){$x_{\text{link}}$}
\put( 87,14){$y_{\text{link}}$}
\put( 13,72){$z_{\text{link}}$}
\thicklines
\put( 24,24){\circle{2}}\put( 21,21){$1$}
\put( 64,22){\circle{2}}\put( 67,20){$3$}
\put( 56,40){\circle{2}}\put( 53,41){$5$}
\put( 34,60){\circle{2}}\put( 28,59){$10$}
\put( 74,52){\circle{2}}\put( 75,48){$12$}
\put( 66,70){\circle{2}}\put( 63,72){$14$}
\put( 44,20.5){\circle{2}}\put( 43,16){$2$}
\put( 59,31){\circle{2}}  \put( 61,31){$4$}
\put( 39,35){\circle{2}}  \put( 37,37){$6$}
\put( 28,42){\circle{2}}  \put( 24,41){$7$}
\put( 67,38){\circle{2}}  \put( 69,37){$8$}
\put( 59,51){\circle{2}}  \put( 56,47){$9$}
\put( 54,54.5){\circle{2}}\put( 51,50){$11$}
\put( 68,62){\circle{2}}  \put( 70,62){$13$}
\put( 50,67){\circle{2}}  \put( 47,69){$15$}
\qbezier    (24,24)(44,18)(64,22)
\qbezier[20](64,22)(57,31)(56,40)
\qbezier[30](56,40)(31,35)(24,24)
\qbezier    (34,60)(54,53)(74,52)
\qbezier    (74,52)(67,61)(66,70)
\qbezier    (66,70)(44,67)(34,60)
\qbezier    (24,24)(27,42)(34,60)
\qbezier    (64,22)(64,37)(74,52)
\qbezier[30](56,40)(59,55)(66,70)
%Nodal dofs
\thinlines
\put(74,52){\vector(1,0){6}}  \put(82,51){$v$}
\put(74,52){\vector(0,1){6}}  \put(73,59){$w$}
\put(74,52){\vector(-1,-1){3.6}}\put(67,47){$u$}
%\graphpaper[2](0,0)(100,80)
\end{picture}
}
\caption{WEDG15, Isoparametric prismatic element}
\label{fig:WEDG15}
\end{figure}

\section{HEX8}
\label{s:HEX8}

HEX8 is a solid hexahedron element with 8 nodal points and 24 DOFs.
The nodal points located at the corners of the hexahedron are
numbered 1 through 8 as shown in Figure~\ref{fig:HEX8}.
HEX8 is an isoparametric element with linear displacement shape functions.

\begin{figure}[p]
\center{
\setlength{\unitlength}{1mm}
\begin{picture}(100,75)
\thinlines
\put( 15,15){\vector(1,0){70}}
\put( 15,15){\vector(0,1){55}}
\put( 15,15){\vector(-1,-1){12}}
\put(  0, 0){$x_{\text{link}}$}
\put( 87,14){$y_{\text{link}}$}
\put( 13,72){$z_{\text{link}}$}
\thicklines
\put( 24,24){\circle{2}}\put( 21,21){$1$}
\put( 64,20){\circle{2}}\put( 65,16){$2$}
\put( 84,38){\circle{2}}\put( 87,38){$3$}
\put( 42,42){\circle{2}}\put( 39,42){$4$}

\put( 28,54){\circle{2}}\put( 23,56){$5$}
\put( 68,50){\circle{2}}\put( 70,46){$6$}
\put( 88,64){\circle{2}}\put( 90,66){$7$}
\put( 46,68){\circle{2}}\put( 41,71){$8$}

\qbezier    (24,24)(44,22)(64,20)
\qbezier    (64,20)(74,29)(84,38)
\qbezier[30](84,38)(63,40)(42,42)
\qbezier[20](42,42)(33,33)(24,24)
\qbezier    (28,54)(48,52)(68,50)
\qbezier    (68,50)(78,57)(88,64)
\qbezier    (88,64)(67,66)(46,68)
\qbezier    (46,68)(37,61)(28,54)
\qbezier    (24,24)(26,39)(28,54)
\qbezier    (64,20)(66,35)(68,50)
\qbezier    (84,38)(86,51)(88,64)
\qbezier[20](42,42)(44,55)(46,68)
%Nodal dofs
\thinlines
\put(64,20){\vector(1,0){6}}    \put(72,19){$v$}
\put(64,20){\vector(0,1){6}}    \put(62,28){$w$}
\put(64,20){\vector(-1,-1){3.6}}\put(57,16){$u$}
\end{picture}
}
\caption{HEX8, Isoparametric hexahedron element}
\label{fig:HEX8}
\end{figure}

\section{HEX20}
\label{s:HEX20}

HEX20 is an isoparametric hexahedron element with 60 DOFs;
3 DOFs $(u,v,w)$ at each of the 20 nodes (8 corner and 12 mid-edge nodes).
The edges may be straight or curved, and Figure~\ref{fig:HEX20} shows a
typical element with local node numbering.
The local node numbering 1 through 20 must be carried out in a right-hand
direction with nodes 1, 9 and 13 on the same edge.
HEX20 yields an incomplete quadratic variation of the displacements.
The minimum variation of stresses along a border is linear for this element.

\begin{figure}[p]
\center{
\setlength{\unitlength}{1mm}
\begin{picture}(100,75)
\thinlines
\put( 15,15){\vector(1,0){70}}
\put( 15,15){\vector(0,1){55}}
\put( 15,15){\vector(-1,-1){12}}
\put(  0, 0){$x_{\text{link}}$}
\put( 87,14){$y_{\text{link}}$}
\put( 13,72){$z_{\text{link}}$}
\thicklines
\put( 24,24){\circle{2}}\put( 21,21){$1$}
\put( 64,20){\circle{2}}\put( 65,16){$3$}
\put( 84,38){\circle{2}}\put( 87,38){$5$}
\put( 42,42){\circle{2}}\put( 39,42){$7$}

\put( 28,54){\circle{2}}\put( 23,56){$13$}
\put( 68,50){\circle{2}}\put( 70,46){$15$}
\put( 88,64){\circle{2}}\put( 90,66){$17$}
\put( 46,68){\circle{2}}\put( 41,71){$19$}

\put( 43,20){\circle{2}}\put( 42,16){$2$}
\put( 74,31){\circle{2}}\put( 76,28){$4$}
\put( 58,42){\circle{2}}\put( 57,44){$6$}
\put( 33,36){\circle{2}}\put( 29,36){$8$}
\put( 25,40){\circle{2}}\put( 21,39){$9$}
\put( 65,36){\circle{2}}\put( 67,35){$10$}
\put( 85,52){\circle{2}}\put( 87,50){$11$}
\put( 43,56){\circle{2}}\put( 45,56){$12$}
\put( 48,50){\circle{2}}\put( 47,52){$14$}
\put( 77,58){\circle{2}}\put( 79,54){$16$}
\put( 64,68){\circle{2}}\put( 62,71){$18$}
\put( 36,63){\circle{2}}\put( 31,64){$20$}

\qbezier    (24,24)(44,18)(64,20)
\qbezier    (64,20)(74,32)(84,38)
\qbezier[30](84,38)(63,43)(42,42)
\qbezier[20](42,42)(33,39)(24,24)
\qbezier    (28,54)(48,48)(68,50)
\qbezier    (68,50)(78,60)(88,64)
\qbezier    (88,64)(67,69)(46,68)
\qbezier    (46,68)(37,67)(28,54)
\qbezier    (24,24)(23,39)(28,54)
\qbezier    (64,20)(63,35)(68,50)
\qbezier    (84,38)(83,51)(88,64)
\qbezier[20](42,42)(41,55)(46,68)
%Nodal dofs
\thinlines
\put(64,20){\vector(1,0){6}}    \put(72,19){$v$}
\put(64,20){\vector(0,1){6}}    \put(60,28){$w$}
\put(64,20){\vector(-1,-1){3.6}}\put(57,16){$u$}
%\graphpaper[2](0,0)(100,80)
\end{picture}
}
\caption{HEX20, Isoparametric hexahedron element}
\label{fig:HEX20}
\end{figure}

% SPDX-FileCopyrightText: 2023 SAP SE
%
% SPDX-License-Identifier: Apache-2.0
%
% This file is part of FEDEM - https://openfedem.org

%%%%%%%%%%%%%%%%%%%%%%%%%%%%%%%%%%%%%%%%%%%%%%%%%%%%%%%%%%%%%%%%%%%%%%%%%%%%%%%%
%
% FEDEM Theory Guide.
%
%%%%%%%%%%%%%%%%%%%%%%%%%%%%%%%%%%%%%%%%%%%%%%%%%%%%%%%%%%%%%%%%%%%%%%%%%%%%%%%%

\section{BEAM2}
\label{s:BEAM2}

The BEAM2 element is based on Euler-Bernoulli's beam theory with quadratic shape
functions and continuous first derivatives.
The deformations account for are bending, shear, axial and St.\ Venant torsion.
The element is straight with uniform cross section and material properties.
The cross section does not need to be symmetrical.
The 2 nodal points of the element, one at each end, may be offset in relation
to the principle axis.

Figure~\ref{fig:BEAM2} shows an arbitrary beam element referred to in a link
coordinate system $(X, Y, Z)$ and a local system $(x, y, z)$.
The local $x$-axis coincides with the beam axis through the center of gravity of
the cross sections, and is positive in the direction from point $I$ to $J$.
The local $y$- and $z$-axes coincide with the principal axes of the cross section.
An auxiliary point $K$ defines together with the beam axis the local $xz$-plane.
The local $z$-axis is positive in the direction from the element toward point $K$.
The end points $I$ and $J$ are connected, via fictitious rigid eccentricities,
to the nodal points $II$ and $JJ$, respectively, at which nodal parameters are defined.

\begin{figure}[t]
\begin{center}
\setlength{\unitlength}{1mm}
\begin{picture}(100,75)
\thinlines
\put(15,15){\vector( 1, 0){70}}
\put(15,15){\vector( 0, 1){55}}
\put(15,15){\vector(-1,-1){12}}
\put( 0, 0){$x_{\text{link}}$}
\put(87,14){$y_{\text{link}}$}
\put(13,72){$z_{\text{link}}$}
\thicklines
\qbezier(30,30)(32,30.5) (34,31)
\qbezier(30,30)(30,32)   (30,34)
\qbezier(30,34)(31,34.25)(32,34.5)
\qbezier(34,31)(33,32.75)(32,34.5)
\put(31.5,32.375){\circle*{1.5}}\put(31,27){$I$}
\put(34,31)  {\line(3,2){35}}
\put(32,34.5){\line(3,2){35}}
\put(30,34)  {\line(3,2){35}}
\qbezier(65,57.33)(66,57.58)(67,57.83)
\qbezier(69,54.33)(68,56.08)(67,57.83)
\put(66.5,55.708){\circle*{1.5}}\put(66,60){$J$}
\thinlines
\put(66.5,55.708){\vector(3,2){15}}\put(83,65){$x_{\text{elem}}$}
\put(31.5,32.375){\vector(-1,4){5}}\put(25,55){$y_{\text{elem}}$}
\put(31.5,32.375){\vector(4,1){40}}\put(73,42){$z_{\text{elem}}$}
\put(66.5,45){\circle*{1.5}}\put(68,45){$K$}
\put(66.5,55.708){\line(0,-1){14.5}}
%Eccentricity at node I to II
\put(31.5,32.375){\line(-1,-1){10}}
\put(21.5,22.375){\line( 1, 0){25}}
\put(46.5,22.375){\line( 0, 1){ 5}}
\put(46.5,27.375){\circle{1.5}}\put(48,26){$II$}
\put(21,27){$e_x$}
\put(32,19){$e_y$}
\put(48,23){$e_z$}
%Eccentricity at node j to JJ
\put(66.5,55.708){\line(-1,-1){5}}
\put(61.5,50.708){\line( 1, 0){15}}
\put(76.5,50.708){\line( 0, 1){ 5}}
\put(76.5,55.708){\circle{1.5}}\put(78,54){$JJ$}
\put(78,50){\scriptsize{fictitious rigid arm}}
%Nodal dofs
%\thinlines
%\put(70,40){\vector(1,0){6}}    \put(79,39){$v$}
%\put(70,40){\vector(0,1){6}}    \put(70,49){$w$}
%\put(70,40){\vector(-1,-1){3.6}}\put(64,37){$u$}
\end{picture}
\end{center}
\caption{BEAM2, Beam element}
\label{fig:BEAM2}
\end{figure}

The BEAM2 element may optionally be equipped with pin flags in end $I$ and/or $J$.
They are used to remove connections between the associated grid point and
selected DOFs of the beam defined in the local element coordinate system.
Thus, they work like inserting a local hinge in the beam for the selected DOFs.
The beam must have stiffness associated with the DOFs that are released in this
manner, e.g.\ if local DOF 4 is released in one end, the beam must have a
nonzero torsional stiffness.

\subsection{Spot weld element}
\label{subs:Spot weld element}

The BEAM2 element described above is also used to represent spot welds\footnote{
Known as CWELD elements in Nastran.} in Fedem.
A circular massive cross section is assumed for a spot weld beam.
An explicit local $z$-axis definition is therefore not needed for such elements.

Instead of the rigid arms, a spot weld BEAM2 element is equipped with a WAVGM
element in each end (see Section~\ref{s:WAVGM} below), in order to distribute
the forces transferred by the beam over a group of nodes in the welded surfaces.
The end points of the beam, which are connected to the reference node of the
WAVGM elements, are then defined by the projection of a specified point
onto each of the 2 welded surfaces.
The WAVGM elements may span an arbitrary number of nodes, but typically they
are connected to all surface nodes of the finite element that is intersected by
the spot weld beam.

Since the welded surfaces typically are quite close, the actual length
of the spot weld beam will be relatively small (or maybe zero).
However, it is possible to equip the spot weld beam with a separate effective
length property to be used instead of its actual length when calculating the
element stiffness.

% SPDX-FileCopyrightText: 2023 SAP SE
%
% SPDX-License-Identifier: Apache-2.0
%
% This file is part of FEDEM - https://openfedem.org

%%%%%%%%%%%%%%%%%%%%%%%%%%%%%%%%%%%%%%%%%%%%%%%%%%%%%%%%%%%%%%%%%%%%%%%%%%%%%%%%
%
% FEDEM Theory Guide.
%
%%%%%%%%%%%%%%%%%%%%%%%%%%%%%%%%%%%%%%%%%%%%%%%%%%%%%%%%%%%%%%%%%%%%%%%%%%%%%%%%

\section{BUSH}
\label{s:BUSH}

BUSH is a 2-node generalized spring element of zero length.
The two element nodes may, or may not be coincident.
The spring element is positioned independently of the two element nodes, and is
connected to these nodes via rigid arms, as shown in Figure~\ref{fig:BUSH}.
Nominal stiffness values for the 3 translational and 3 rotational DOFs are given
in a local element coordinate system.
This local system is either specified explicitly as an element property,
or defined implicitly through the two nodes and a given auxiliary point.
The implicit definition is similar to that of the local coordinate system for a
beam element (see Section~\ref{s:BEAM2}).
Thus, this definition is not applicable if the two element nodes are coincident.
The BUSH element does not contribute to the mass matrix.

\begin{figure}[b]
\begin{center}
\setlength{\unitlength}{1mm}
\begin{picture}(100,75)
\thinlines
\put(15,15){\vector(1,0){70}}
\put(15,15){\vector(0,1){55}}
\put(15,15){\vector(-1,-1){12}}
\put( 0, 0){$x_{\text{link}}$}
\put(87,14){$y_{\text{link}}$}
\put(13,72){$z_{\text{link}}$}
\thicklines
\put(26,20){\circle{2}}\put(22,20){$I$}
\put(70,40){\circle{2}}\put(66,40){$J$}
\put(50,50){\circle*{2}}
%Eccentric arms
\thinlines
\put(50,50){\line(-1,-1){10}}\put(45,42){$e_x$}
\put(40,40){\line(-1, 0){14}}\put(33,37){$e_y$}
\put(26,40){\line( 0,-1){20}}\put(27,30){$e_z$}
\qbezier[8](50,50)(52,52)(55,55)
\qbezier[16](55,55)(62,55)(70,55)
\qbezier[16](70,55)(70,48)(70,40)
\put(71,50){\scriptsize fictitious rigid arm}
%Local system
\thinlines
\put(50,50){\vector( 2, 1){20}}\put(70.5,60){$x_\text{elem}$}
\put(50,50){\vector(-1, 2){4}} \put(42,60){$y_\text{elem}$}
\put(50,50){\vector( 3,-1){10}}\put(60.5,46){$z_\text{elem}$}
\end{picture}
\end{center}
\caption{BUSH, Generalized spring element}
\label{fig:BUSH}
\end{figure}

The BUSH element may also be specified without any stiffness properties,
but only the element topology.
Such elements are automatically created during modeling in Fedem,
for instance when a mechanism joint is attached to a link at a slave node of an
RGD, RBAR or WAVGM element (see the \FedemUG, {\em Section~3.6,
``Attaching and detaching elements''}).
A property-less BUSH element is the created as the connection between this slave
node and an added added external node (triad) at the same location.

The nominal stiffness values for the property-less BUSH element are computed
automatically based on the overall stiffness properties of the link\footnote{
Unless the link is completely rigid, e.g., it consists of a single RGD element.
In that case $k_t = k_r = 2.0 \cdot 10^{11}$ is used instead.}.
For the translational and rotational stiffnesses, $k_t$ and $k_r$, respectively,
the following three alternative procedures are available:
%
\begin{eqnarray}
k_t \;=\; \frac{0.1}{\varepsilon_{\rm sing}}
          \min\left\{{\rm diag}({\bf K}_{\rm tra})\right\} &,&
k_r \;=\; \frac{0.1}{\varepsilon_{\rm sing}}
          \min\left\{{\rm diag}({\bf K}_{\rm rot})\right\} \\
%
k_t \;=\; C_s \frac{{\rm tr}({\bf K}_{\rm tra})}{n_{\rm tra}} \hskip47pt &,&
k_r \;=\; C_s \frac{{\rm tr}({\bf K}_{\rm rot})}{n_{\rm rot}} \\
%
k_t \;=\; C_s \max\left\{{\rm diag}({\bf K}_{\rm tra})\right\} \hskip7pt &,&
k_r \;=\; C_s \max\left\{{\rm diag}({\bf K}_{\rm rot})\right\} \label{eq:defB}
\end{eqnarray}
%
Here, ${\bf K}_{\rm tra}$ and ${\bf K}_{\rm rot}$ are the translational and
rotational parts, respectively, of the fully assembled link stiffness matrix,
and $n_{\rm tra}$ and $n_{\rm rot}$ denote the total number of translational
and rotational DOFs, respectively.
Moreover, diag($\cdot$) denotes the diagonal elements of a given matrix,
whereas tr($\cdot$) is the trace operator (sum of diagonal elements).
Finally, $\varepsilon_{\rm sing}$ denotes the {\em singularity criterion}
used by the Fedem Link Reducer when factoring the link stiffness matrix
(specified by the user through the Link property panel,
see the \FedemUG, {\em Section~4.1.5,``Link properties''}),
and $C_s$ is a user-defined scaling factor that may be specified through the
command-line option {\tt -autoStiffScale} when running the Link Reducer
(default is $10^2$).

The wanted procedure is selected through the command-line option
{\tt -autoStiffMethod}.
The default is to use \eqnref{eq:defB}.

% SPDX-FileCopyrightText: 2023 SAP SE
%
% SPDX-License-Identifier: Apache-2.0
%
% This file is part of FEDEM - https://openfedem.org

%%%%%%%%%%%%%%%%%%%%%%%%%%%%%%%%%%%%%%%%%%%%%%%%%%%%%%%%%%%%%%%%%%%%%%%%%%%%%%%%
%
% FEDEM Theory Guide.
%
%%%%%%%%%%%%%%%%%%%%%%%%%%%%%%%%%%%%%%%%%%%%%%%%%%%%%%%%%%%%%%%%%%%%%%%%%%%%%%%%

\section{SPRING and RSPRING}
\label{s:SPRING and RSPRING}

SPRING is a 2-node linear spring element which adds a 6$\times$6 symmetric
stiffness matrix to the translatory DOFs of the 2 nodes.
The element stiffness matrix is referred to in the global (or link) coordinate system.
The element may be of zero length, i.e.\ the 2 nodes can have identical coordinates.
RSPRING is similar to SPRING, but for the rotational DOFs.
The SPRING element may be connected to both 3-DOF and 6-DOF nodes\footnote{
3-DOF nodes are internal nodes that are used by solid finite elements only
and thus lack rotational DOFs.
6-DOF nodes have 3 translatory- and 3 rotational DOFs,
and are connected to at least one shell, beam or RBAR element,
or is a master node in a RGD element},
whereas the RSPRING element may be connected to 6-DOF nodes only.
The SPRING and RSPRING elements do not contribute to the mass matrix.

\section{CMASS}
\label{s:CMASS}

CMASS is a 1-node concentrated mass element which adds a 6$\times$6 symmetric
mass matrix to a 3- or 6-DOF node.
The element mass matrix is referred to in the global (or link) coordinate system.
Note that any non-zero inertia terms are ignored for CMASS elements
that are connected to 3-DOF nodes.
The CMASS element does not contribute to the stiffness matrix.

The CMASS element may also be specified without any mass properties,
but only the element topology.
During modeling in Fedem, such elements are automatically created at the extra
node that is added when creating property-less BUSH elements at slave nodes,
see Section~\ref{s:BUSH}.
The CMASS element is needed at such added nodes to avoid that the assembled mass
matrix becomes singular, with subsequent failure in the eigenvalue analysis.
For such property-less CMASS elements, a diagonal element matrix is assumed with
the following values for the translational and rotational DOFs, respectively:
%
\begin{eqnarray}
\label{eq:max translational mass}
m_t &=& C_m \max\left\{{\rm diag}({\bf M}_{\rm tra})\right\} \\
\label{eq:max rotational mass}
m_r &=& C_m \max\left\{{\rm diag}({\bf M}_{\rm rot})\right\}
\end{eqnarray}
%
Here, ${\bf M}_{\rm tra}$ and ${\bf M}_{\rm rot}$ are the translational and
rotational parts, respectively, of the fully assembled mass matrix,
and $C_m$ is a user-defined scaling factor that is specified through the
command-line argument {\tt -autoMassScale} when running the Fedem Link Reducer
(default value is $10^{-9}$).

In some cases, it may happen that the value $m_r$ defined by
\eqnref{eq:max rotational mass}, is identically zero.
For instance, if the finite element model consists of solid elements only,
in addition to at least one WAVGM element with automatically added BUSH and
CMASS elements at its slave node, there are no mass contributions to the
rotational DOFs in the model.
In such cases, a value for $m_r$ is instead derived from the global inertia
tensor, ${\bf I}$, that may be computed from the finite element model, i.e.
%
\begin{equation}
{\bf I} := \left[\begin{array}{ccc}
I_{xx} & I_{xy} & I_{xz} \\
I_{xy} & I_{yy} & I_{yz} \\
I_{xz} & I_{yz} & I_{zz} \\
\end{array}\right] = \int\limits_\Omega \rho \left[\begin{array}{ccc}
y^2 + z^2    & xy        & xz \\
             & x^2 + z^2 & yz \\
\mbox{symm.} &           & x^2 + y^2
\end{array}\right] dV
\end{equation}
%
where $\rho$ is the mass density, and
%
\begin{equation}
m_r \;=\; \frac{C_m}{3} \left( I_{xx} + I_{yy} + I_{zz} \right)
\end{equation}

% SPDX-FileCopyrightText: 2023 SAP SE
%
% SPDX-License-Identifier: Apache-2.0
%
% This file is part of FEDEM - https://openfedem.org

%%%%%%%%%%%%%%%%%%%%%%%%%%%%%%%%%%%%%%%%%%%%%%%%%%%%%%%%%%%%%%%%%%%%%%%%%%%%%%%%
%
% FEDEM Theory Guide.
%
%%%%%%%%%%%%%%%%%%%%%%%%%%%%%%%%%%%%%%%%%%%%%%%%%%%%%%%%%%%%%%%%%%%%%%%%%%%%%%%%

\section{RBAR}
\label{s:RBAR}

RBAR is a rigid bar element with 2 nodal points and 6 DOFs assigned to each node.
The nodal points located at each end of the bar are numbered 1 and 2 as shown in Figure~\ref{fig:RBAR}.
The nodal DOFs are referred to in the link coordinate system.
The element has no material properties.
The 12 DOFs of the element are related to each other through the following set of equations:
%
\begin{eqnarray}
u_2 &=& u_1\, + e_z\theta_{y1} - e_y\theta_{z1} \label{equ:RBAR1} \\
v_2 &=& v_1\: + e_x\theta_{z1} - e_z\theta_{x1} \\
w_2 &=& w_1   + e_y\theta_{x1} - e_x\theta_{y1} \\
\theta_{x2} &=& \theta_{x1} \\
\theta_{y2} &=& \theta_{y1} \\
\theta_{z2} &=& \theta_{z1} \label{equ:RBAR6}
\end{eqnarray}
%
where $e_x,e_y,e_z$ are the relative distance between the two nodes
in the respective coordinate directions, as depicted in Figure~\ref{fig:RBAR}

\begin{figure}[t]
\begin{center}
\setlength{\unitlength}{1mm}
\begin{picture}(100,65)
\thinlines
\put(15,15){\vector(1,0){70}}
\put(15,15){\vector(0,1){45}}
\put(15,15){\vector(-1,-1){12}}
\put( 0, 0){$x_{\text{link}}$}
\put(87,14){$y_{\text{link}}$}
\put(13,62){$z_{\text{link}}$}
\thicklines
\put(36,30){\circle*{2}}\put(32,30){$1$}
\put(70,50){\circle*{2}}\put(66,50){$2$}
\qbezier(36,30)(53,40)(70,50)
%Eccentric arms
\thinlines
\put(36,30){\line(-1,-1){10}}\put(32,24){$e_x$}
\put(26,20){\line( 1, 0){44}}\put(48,22){$e_y$}
\put(70,20){\line( 0, 1){30}}\put(71,35){$e_z$}
%Nodal dofs
\thinlines
\put(70,50){\vector(-1,-1){4.8}}\put(62,42){$u,\theta_x$}
\put(70,50){\vector( 1, 0){8}}  \put(79,49){$v,\theta_y$}
\put(70,50){\vector( 0, 1){8}}  \put(67,60){$w,\theta_z$}
\end{picture}
\end{center}
\caption{RBAR, Rigid bar element}
\label{fig:RBAR}
\end{figure}

The physical properties of the RBAR element are two sets of component numbers
at each node, identifying the dependent and independent DOFs at the node.
The total number of independent DOFs in the element must be equal to 6 and they
must jointly be capable of representing any general rigid body motion of the element.
The element may have up to 6 dependent DOFs.
If no dependent DOFs are specified, all DOFs that are not specified
as independent will be made dependent.

Since the RBAR element may have dependent DOFs at both nodes, none of the nodes
can be a triad (external node) in Fedem.
If all independent DOFs are gathered at one node (and all the dependent DOFs
are at the other node), the RBAR element is equivalent to a 2-node RGD element,
see Section~\ref{s:RGD}.
During an analysis, rigid bar elements are processed using a DOF elimination method.
The constraint equations~(\ref{equ:RBAR1})--(\ref{equ:RBAR6}) are generated for
each element and are used to eliminate the dependent DOFs from the global
system of equations, before that system is assembled.

\section{RGD}
\label{s:RGD}

RGD is a rigid element with one master node having 6 independent DOFs, 3~translations and 3~rotations.
All remaining nodes are slave nodes having either 3 or 6 dependent DOFs.
Only the master node can be a triad (external node) in Fedem.
The nodal points are numbered 1 through number of nodes as shown in Figure~\ref{fig:RGD}.
A RGD element has no material properties.
The set of dependent DOFs at the slave nodes may optionally be specified as a physical property.
The default behavior if no dependent DOFs are specified is that all DOFs
at the slave nodes are made dependent on the master node DOFs.

\begin{figure}[t]
\begin{center}
\setlength{\unitlength}{1mm}
\begin{picture}(100,65)
\thinlines
\put(15,15){\vector( 1, 0){70}}
\put(15,15){\vector( 0, 1){45}}
\put(15,15){\vector(-1,-1){12}}
\put( 0, 0){$x_{\text{link}}$}
\put(87,14){$y_{\text{link}}$}
\put(13,62){$z_{\text{link}}$}
\thicklines
\put(36,30){\circle{2}} \put(32,27){$1_m$}
\put(80,20){\circle*{2}}\put(82,19){$2_s$}
\put(70,40){\circle*{2}}\put(71,36){$3_s$}
\put(60,60){\circle*{2}}\put(62,60){$4_s$}
\put(30,50){\circle*{2}}\put(29,53){$5_s$}
\qbezier(36,30)(58,25)(80,20)
\qbezier(36,30)(53,35)(70,40)
\qbezier(36,30)(48,45)(60,60)
\qbezier(36,30)(33,40)(30,50)
%Nodal dofs
\thinlines
\put(70,40){\vector(-1,-1){3.6}}\put(64,35){$u$}
\put(70,40){\vector(1,0){6}}    \put(77,39){$v$}
\put(70,40){\vector(0,1){6}}    \put(69,47){$w$}
\end{picture}
\end{center}
\caption{RGD, Multi-node rigid element}
\label{fig:RGD}
\end{figure}

The presence of a rigid element in a model implies that the motion of all the
slave nodes on the element are to be constrained as though they were connected
to the master node by mass less rigid beams (or semi-rigid if not all slave
node DOFs are made dependent).
During an analysis, rigid elements are processed using a DOF elimination method.
A set of constraint equations, equivalent to Equations~(\ref{equ:RBAR1})--(\ref{equ:RBAR6})
is generated for each slave node, relating the dependent DOFs to the
independent DOFs of the master node.
These equations are then used to eliminate all dependent DOFs from the
global system of equations, prior to the system matrix assembly.
Note that a master node in one RGD element may be a slave node in another RGD.
Such chains of RGD elements are resolved explicitly in such a way that all
slave DOFs ultimately are coupled only to independent DOFs that do not depend
on other RGD constraints.

% SPDX-FileCopyrightText: 2023 SAP SE
%
% SPDX-License-Identifier: Apache-2.0
%
% This file is part of FEDEM - https://openfedem.org

%%%%%%%%%%%%%%%%%%%%%%%%%%%%%%%%%%%%%%%%%%%%%%%%%%%%%%%%%%%%%%%%%%%%%%%%%%%%%%%%
%
% FEDEM Theory Guide.
%
%%%%%%%%%%%%%%%%%%%%%%%%%%%%%%%%%%%%%%%%%%%%%%%%%%%%%%%%%%%%%%%%%%%%%%%%%%%%%%%%

\section{WAVGM}
\label{s:WAVGM}

WAVGM is an interpolation constraint element\footnote{
This element is known as RBE3 in Nastran.},
which defines the motion at a reference (slave) node as the weighted average of
the motions at a set of other (master) nodes.
The element topology is similar to that of the RGD-element,
see Figure~\ref{fig:WAVGM}, except that for WAVGM node 1 is a slave node
containing the dependent DOFs, whereas all other nodes are masters with
independent DOFs.

\begin{figure}[t]
\begin{center}
\setlength{\unitlength}{1mm}
\begin{picture}(100,75)
\thinlines
\put(15,15){\vector( 1, 0){70}}
\put(15,15){\vector( 0, 1){55}}
\put(15,15){\vector(-1,-1){12}}
\put( 0, 0){$x_{\text{link}}$}
\put(87,14){$y_{\text{link}}$}
\put(13,72){$z_{\text{link}}$}
\thicklines
\put(36,30){\circle*{2}} \put(31,30){$1_s$}
\put(80,20){\circle{2}}\put(82,19){$2_m$}
\put(70,40){\circle{2}}\put(71,36){$3_m$}
\put(60,60){\circle{2}}\put(62,60){$4_m$}
\put(30,50){\circle{2}}\put(29,53){$5_m$}
\qbezier[25](36,30)(58,25)(80,20)
\qbezier[20](36,30)(53,35)(70,40)
\qbezier[20](36,30)(48,45)(60,60)
\qbezier[12](36,30)(33,40)(30,50)
%Nodal dofs
\thinlines
\put(36,30){\vector(-1,-1){4.8}}\put(28,22.5){$u,\theta_x$}
\put(36,30){\vector(1,0){8}}    \put(45,29){$v,\theta_y$}
\put(36,30){\vector(0,1){8}}    \put(34,39){$w,\theta_z$}
\end{picture}
\end{center}
\caption{WAVGM, Multi-node weighted averaged motion element}
\label{fig:WAVGM}
\end{figure}

Unlike the RBAR and RGD elements described in the previous sections,
the WAVGM element does not add stiffness to the link, unless the slave node
already is connected to some of the master nodes via other finite elements.
Thus, the WAVGM element works like a force distributor; forces that are applied
at the reference (slave) node are distributed over the master nodes depending
on the given weighting factors and the relative distance to the reference node.

The manner in which the forces are distributed is analogous to the classical
bolt pattern analysis.
Consider a given force \bm{F} and a moment\bm{M} applied at the reference node
of the WAVGM element.
They are first replaced by an equivalent force $\widetilde{\bm{F}}=\bm{F}$
and moment $\widetilde{\bm{M}}=\bm{M}+\bm{F}\times\bm{e}$ at the weighted center
of gravity of the master nodes, where \bm{e} is the offset vector between the
reference node and the weighted center of gravity.
The force $\widetilde{\bm{F}}$ is then distributed to the master nodes
proportional to the given weighting factors, whereas the moment
$\widetilde{\bm{M}}$ is distributed as forces proportional to their distance
from the weighted center of gravity times their weighting factors.
Alternatively, the moment $\widetilde{\bm{M}}$ may be distributed directly as
moments at the master nodes (provided they are 6-DOF nodes) proportional to
their weighting factors, in the same manner as the force $\widetilde{\bm{F}}$.
This can be used if all master nodes of the element are co-linear, such that
they cannot absorb a moment though a set of forces.

The weighting factors can in principle be different for the various force
component at a given master node.
Thus, the force distribution is carried out on a component-by-component basis.
The resulting expression for the $i$'th force component at master
node number $a$ is then
%
\begin{equation}
\label{equ:WAVGM force distribution}
f_i^a \;=\; \frac{F_i\,\omega_i^a}{\sum_b\omega_i^b}
      \;+\; \frac{\left( M_j - F_k e_i + F_i e_k \right) \omega_i^a r_{ik}^a}
                 {\sum_b\left( {r_{ik}^b}^2 + {r_{ii}^b}^2 \right)\omega_i^b}
      \;-\; \frac{\left( M_k - F_i e_j + F_j e_i \right) \omega_i^a r_{ij}^a}
                 {\sum_b\left( {r_{ii}^b}^2 + {r_{ij}^b}^2 \right)\omega_i^b}
\end{equation}
%
where $(i,j,k)$ forms a cyclic permutation of the components $x,y,z$,
and $r_{ij}^a$ denotes the $j$'th component of the relative position vector
$\bm{r}_i^a$ from the weighted center of gravity to the $a$'th master node,
based on the weighting factors $\omega_i^a$:
%
\begin{equation}
\bm{r}_i^a = \bm{x}^a - \frac{\sum_b\omega_i^b \bm{x}^b}{\sum_b\omega_i^b}
\end{equation}
%
Alternatively, when the master nodes are co-linear such that the moment has to
be distributed directly, the force components at master node $a$ are given by
only the first term of Equation~(\ref{equ:WAVGM force distribution}), whereas
the moment components are
%
\begin{equation}
\label{equ:WAVGM moment distribution}
m_i^a \;=\; \frac{\left( M_i - F_j e_k + F_k e_j \right) \omega_{3+i}^a}
                 {\sum_b\omega_{3+i}^b}
\end{equation}

The above expressions are now used to establish the governing constraint
equations for the WAVGM element, which are used to eliminate the slave node DOFs
in the global system of equations of the link FE model.
Let ${\bf R}_m$ and ${\bf R}_s$ denote vectors that collect force components at
all master DOFs and slave DOFs, respectively, in the element.
Similarly, let ${\bf r}_m$ and ${\bf r}_s$ denote the associated displacement
 vectors.
The master and slave components are then related through
%
\begin{eqnarray}
{\bf r}_s &=& {\bf T}_c   \;{\bf r}_m \label{equ:WAVGM constraints}\\
{\bf R}_m &=& {\bf T}_c^T \:{\bf R}_s
\end{eqnarray}
%
The row of ${\bf T}_c^T$ (column of ${\bf T}_c$) corresponding to a given
slave DOF is then obtained by in inserting a unit value for $F_x,F_y,F_z$,
$M_x,M_y,M_z$, respectively, in turn while letting the other components be zero.

It is clear that for some WAVGM element geometries the denominators of
Equation~(\ref{equ:WAVGM force distribution}) might be small, or even zero.
For instance, for a three-noded element where all nodes lie on a line that is
parallel to the global x-axis, the denominator of the first term is zero when
$i=3$.
The size of the denominator is therefore checked against a threshold value,
and the resulting constraint coefficient is omitted if the denominator is
smaller than this threshold. These checks are performed as follows for the
two terms:
%
\begin{eqnarray}
\sum_b\left( {r_{ik}^b}^2 + {r_{ii}^b}^2 \right)\omega_i^b &>&
\left(\max_b \|r_{ik}^b\|\,\epsilon_{\rm tol} \right)^2 +
\left(\max_b \|r_{ii}^b\|\,\epsilon_{\rm tol} \right)^2 \\[1mm]
\sum_b\left( {r_{ii}^b}^2 + {r_{ij}^b}^2 \right)\omega_i^b &>&
\left(\max_b \|r_{ii}^b\|\,\epsilon_{\rm tol} \right)^2 +
\left(\max_b \|r_{ij}^b\|\,\epsilon_{\rm tol} \right)^2
\end{eqnarray}
%
where $\epsilon_{tol}$ is a relative tolerance parameter that may be set by the
user through the command-line option {\tt -tolWAVGM} of the Fedem Link Reducer
(default value $=10^{-4}$).
Thus, constraint coefficients are added only for those terms satisfying the
above conditions.

It should be emphasized that the constraints given by
Equation~(\ref{equ:WAVGM constraints}) are enforced in strong form in Fedem
(the same is true for the RGD and RBAR elements as well).
This implies that a WAVGM slave node can not be a triad (external node) in Fedem.
Moreover, WAVGM elements where the slave node already is connected to the master
nodes trough other finite elements should be used with caution.
Such element may result in an over-constrained system of equations, such that
the resulting reduced link does not possess the necessary 6 rigid body modes.
This may in turn make the dynamics simulation unstable.

% SPDX-FileCopyrightText: 2023 SAP SE
%
% SPDX-License-Identifier: Apache-2.0
%
% This file is part of FEDEM - https://openfedem.org

%%%%%%%%%%%%%%%%%%%%%%%%%%%%%%%%%%%%%%%%%%%%%%%%%%%%%%%%%%%%%%%%%%%%%%%%%%%%%%%%
%
% FEDEM Theory Guide.
%
%%%%%%%%%%%%%%%%%%%%%%%%%%%%%%%%%%%%%%%%%%%%%%%%%%%%%%%%%%%%%%%%%%%%%%%%%%%%%%%%

\section{Generic part element}
\label{sec:GenericPart}

A generic part consists of a rigid spider attached to
a Center of Gravity (CG) node.
Each spider leg spans the distance from the CG node to
one of the other element nodes.
At the end of each leg there is a linear spring (equivalent to the BUSH element
described in Section~\ref{s:BUSH}) with some stiffness $k_t$ against translation
in all directions, and stiffness $k_r$ against rotation in all three directions.
For two co-located nodes ($i$ and $j$) the overall stiffness
of one such spring element is then given by
%
\begin{equation}
\label{eq:leg stiffness}
\left[\begin{array}{l}
{\mf f}_i \\ {\mf m}_i \\ {\mf f}_j \\ {\mf m}_j
\end{array}\right] =
\left[\begin{array}{rrrr}
 k_t {\mf I} &      {\mf 0} & -k_t {\mf I} &      {\mf 0} \\
     {\mf 0} &  k_r {\mf I} &      {\mf 0} & -k_r {\mf I} \\
-k_t {\mf I} &      {\mf 0} &  k_t {\mf I} &      {\mf 0} \\
     {\mf 0} & -k_r {\mf I} &      {\mf 0} &  k_r {\mf I}
\end{array}\right]
\left[\begin{array}{l}
{\mf v}_i \\ {\tf \theta}_i \\ {\mf v}_j \\ {\tf \theta}_j
\end{array}\right]
\end{equation}

With node $j$ being rigidly attached to the CG node,
one can establish the virtual displacement relation
%
\begin{equation}
\left[\begin{array}{c}
{\mf v}_j \\ {\tf \theta}_j
\end{array}\right] =
\left[\begin{array}{rr}
{\mf 1} & -\widehat{{\mf e}} \\
{\mf 0} & {\mf 1}
\end{array}\right]
\left[\begin{array}{c}
{\mf v}_\textit{CG} \\ {\tf \theta}_\textit{CG}
\end{array}\right]
\quad\text{where}\quad {\mf e} =
\left[\begin{array}{c}
x_j - x_\textit{CG} \\
y_j - y_\textit{CG} \\
z_j - z_\textit{CG}
\end{array}\right]
\end{equation}

Using the kinematic relationship above in a virtual work expression, one
can establish the stiffness matrix for the spider leg between element node $i$
and the CG node as
%
\begin{equation}
\label{eq:spider stiffness}
\left[\begin{array}{l}
{\mf f}_i \\ {\mf m}_i \\ {\mf f}_\textit{CG} \\ {\mf m}_\text{CG}
\end{array}\right] =
\left[\begin{array}{cccc}
 {\mf k}_t & {\mf 0}   & -{\mf k}_t &  {\mf k}_t \hat{{\mf e}} \\
 {\mf 0}   & {\mf k}_r &  {\mf 0}   & -{\mf k}_r \\
-{\mf k}_t & {\mf 0}   &  {\mf k}_t & -{\mf k}_t \hat{{\mf e}} \\
 \hat{{\mf e}}^T {\mf k}_t & -{\mf k}_r &
-\hat{{\mf e}}^T {\mf k}_t &
 ({\mf k}_r + \hat{{\mf e}}^T{\mf k}_T \hat{{\mf e}})
\end{array}\right]
\left[\begin{array}{l}
{\mf v}_i \\ {\tf \theta}_i \\ {\mf v}_\textit{CG} \\ {\tf \theta}_\text{CG}
\end{array}\right]
\end{equation}
%
where ${\mf k}_t = k_t{\mf I}$ and ${\mf k}_r = k_r{\mf I}$.

The full stiffness matrix for the generic part is then formed by assembling the
nodal contributions from all such spider legs.
Rewriting \eqnref{eq:spider stiffness} with the more compact notation
%
\begin{equation}
\left[\begin{array}{c}
{\mf f}^i_1 \\ {\mf f}^i_2
\end{array}\right] =
\left[\begin{array}{cc}
 {\mf k}^i_{11} &  {\mf k}^i_{12} \\
 {\mf k}^i_{21} &  {\mf k}^i_{22}
\end{array}\right]
\left[\begin{array}{c}
{\mf v}^i_1 \\ {\mf v}^i_2
\end{array}\right]
\end{equation}
%
the assembly of \eqnref{eq:spider stiffness} for all nodes from $1$ to $n$
(with $n$ being the CG node) gives the following generic part stiffness matrix
%
\begin{equation}
\label{eq:generic part stiffness matrix}
\left[\begin{array}{c}
{\mf f}^1_1 \\ \vdots \\ \;\;\;{\mf f}^{n-1}_1 \\
\sum\limits_{j=1}^{n-1} {\mf f}^j_2
\end{array}\right] =
\left[\begin{array}{cccc}
{\mf k}^1_{11} & \hdots & {\mf 0}           & {\mf k}^1_{12} \\
\vdots         & \ddots &                   & \vdots         \\
{\mf 0}        &        & {\mf k}^{n-1}_{11} & \;\;\;{\mf k}^{n-1}_{12} \\
{\mf k}^1_{21} & \hdots & {\mf k}^{n-1}_{21} &
\sum\limits_{j=1}^{n-1} {\mf k}^j_{22}
\end{array}\right]
\left[\begin{array}{l}
{\mf v}^1_1 \\ \;\vdots \\ {\mf v}^{n-1}_1 \!\!\\[2mm] {\mf v}^n_2 \\[2mm]
\end{array}\right]
\end{equation}

The mass properties of the generic part element are assumed concentrated at the
CG node, both with respect to translation and rotation.
This gives the mass matrix defined in the relationship
%
\begin{equation}
\label{eq:generic part mass matrix}
\left[\begin{array}{l}
{\mf f}^1_1 \\ \;\vdots \\ {\mf f}^{n-1}_1 \!\!\\ {\mf f}^n_2
\end{array}\right] =
\left[\begin{array}{cccc}
{\mf 0} & \hdots & {\mf 0} & {\mf 0} \\
\vdots  & \ddots &         & \vdots  \\
{\mf 0} &        & {\mf 0} & {\mf 0} \\
{\mf 0} & \hdots & {\mf 0} & {\mf m}^n_{22}
\end{array}\right]
\left[\begin{array}{l}
\ddot{\mf v}^1_1 \\ \;\vdots \\ \ddot{\mf v}^{n-1}_1 \!\!\\ \ddot{\mf v}^n_2
\end{array}\right]
\end{equation}
%
where
%
\begin{equation}
\label{eq:m22}
{\mf m}^n_{22} =
\left[\begin{array}{cccccc}
m & 0 & 0 & 0 & 0 & 0 \\
0 & m & 0 & 0 & 0 & 0 \\
0 & 0 & m & 0 & 0 & 0 \\
0 & 0 & 0 & I_{xx} & I_{xy} & I_{xz} \\
0 & 0 & 0 & I_{xy} & I_{yy} & I_{yz} \\
0 & 0 & 0 & I_{xz} & I_{yz} & I_{zz}
\end{array}\right]
\end{equation}

When using the generic part element with the element matrices defined by
\eqsref{eq:generic part stiffness matrix}{eq:generic part mass matrix}
in a dynamics simulation, one might get artificial oscillations in the CG node
DOFs, depending on the actual magnitude of the characteristic mass and
stiffness being used.
This problem might be avoided by eliminating the CG node DOFs through static
condensation of
\eqsref{eq:generic part stiffness matrix}{eq:generic part mass matrix},
before the time integration is started.
This is similar to what is being done for the internal DOFs of the finite
element parts in the model reduction process,
see Sections~\ref{subsec:Static modes} and~\ref{subsec:Reduced system}.

Assuming that all ${\mf f}^j_2$ in \eqnref{eq:generic part stiffness matrix}
are always zero (since no distributed loads are associated with a generic part),
the last line of \eqnref{eq:generic part stiffness matrix} yields
%
\begin{equation}
\label{eq:generic part B-matrix}
{\mf v}^n_2 = {\mf B}
\left[\begin{array}{l}
{\mf v}^1_1 \\ \;\vdots \\ {\mf v}^{n-1}_1
\end{array}\right] \quad\text{where}\quad
{\mf B} =
\left[ \sum\limits_{j=1}^{n-1} {\mf k}^j_{22} \right]^{-1}
\left[\begin{array}{cccc}
{\mf k}^1_{21} & \hdots & {\mf k}^{n-1}_{21}
\end{array}\right]
\end{equation}
%
Combining \eqnref{eq:generic part B-matrix} and its associated second-derivative
with the first $n-1$ lines of
\eqsref{eq:generic part stiffness matrix}{eq:generic part mass matrix},
respectively, while pre-multiplying with ${\mf B}^T$ produces the following
element matrices for the generic part
%
\begin{eqnarray}
{\mf k} &=&
\left[\begin{array}{cccc}
{\mf k}^1_{21} & \hdots & {\mf k}^{n-1}_{21}
\end{array}\right]^T {\mf B} +
\left[\begin{array}{ccc}
{\mf k}^1_{11} & \hdots & {\mf 0} \\
\vdots         & \ddots & \vdots  \\
{\mf 0}        & \hdots & {\mf k}^{n-1}_{11}
\end{array}\right] \\
{\mf m} &=& {\mf B}^Ti {\mf m}_{22}^n {\mf B}
\end{eqnarray}

The stiffness coefficients $k_t$ and $k_r$ that is used in
\eqnref{eq:leg stiffness} may be specified by the user for each part
(see the \FedemUG, {\em Section~4.1.5 ``Link properties''}).
However, it is also possible to let the coefficients be automatically computed,
in such a way that the generic part behaves like an ``almost'' rigid element.

The automatic ``rigid'' stiffness is computed from the mass properties of the
part and a given target eigenfrequency:
%
\begin{eqnarray}
k_t &=& \left(2\pi f_{\rm rig}\right)^2 \,m \\
k_r &=& \left(2\pi f_{\rm rig}\right)^2 \,\frac{1}{3}\sum_{i,j = x,y,z} I_{ij}
\end{eqnarray}
%
where $m$ and $I_{ij}$ are components of the mass matrix~\eqref{eq:m22},
and $f_{\rm rig}$ is the desired target eigenfrequency of the part (in [Hz]).
This target value may be set by the user through the command-line option
{\tt -targetFrequencyRigid} for the Dynamics Solver.
The default value is 10000 Hz.

