% SPDX-FileCopyrightText: 2023 SAP SE
%
% SPDX-License-Identifier: Apache-2.0
%
% This file is part of FEDEM - https://openfedem.org

%%%%%%%%%%%%%%%%%%%%%%%%%%%%%%%%%%%%%%%%%%%%%%%%%%%%%%%%%%%%%%%%%%%%%%%%%%%%%%%%
%
% FEDEM Theory Guide.
%
%%%%%%%%%%%%%%%%%%%%%%%%%%%%%%%%%%%%%%%%%%%%%%%%%%%%%%%%%%%%%%%%%%%%%%%%%%%%%%%%

\section{SPRING and RSPRING}
\label{s:SPRING and RSPRING}

SPRING is a 2-node linear spring element which adds a 6$\times$6 symmetric
stiffness matrix to the translatory DOFs of the 2 nodes.
The element stiffness matrix is referred to in the global (or link) coordinate system.
The element may be of zero length, i.e.\ the 2 nodes can have identical coordinates.
RSPRING is similar to SPRING, but for the rotational DOFs.
The SPRING element may be connected to both 3-DOF and 6-DOF nodes\footnote{
3-DOF nodes are internal nodes that are used by solid finite elements only
and thus lack rotational DOFs.
6-DOF nodes have 3 translatory- and 3 rotational DOFs,
and are connected to at least one shell, beam or RBAR element,
or is a master node in a RGD element},
whereas the RSPRING element may be connected to 6-DOF nodes only.
The SPRING and RSPRING elements do not contribute to the mass matrix.

\section{CMASS}
\label{s:CMASS}

CMASS is a 1-node concentrated mass element which adds a 6$\times$6 symmetric
mass matrix to a 3- or 6-DOF node.
The element mass matrix is referred to in the global (or link) coordinate system.
Note that any non-zero inertia terms are ignored for CMASS elements
that are connected to 3-DOF nodes.
The CMASS element does not contribute to the stiffness matrix.

The CMASS element may also be specified without any mass properties,
but only the element topology.
During modeling in Fedem, such elements are automatically created at the extra
node that is added when creating property-less BUSH elements at slave nodes,
see Section~\ref{s:BUSH}.
The CMASS element is needed at such added nodes to avoid that the assembled mass
matrix becomes singular, with subsequent failure in the eigenvalue analysis.
For such property-less CMASS elements, a diagonal element matrix is assumed with
the following values for the translational and rotational DOFs, respectively:
%
\begin{eqnarray}
\label{eq:max translational mass}
m_t &=& C_m \max\left\{{\rm diag}({\bf M}_{\rm tra})\right\} \\
\label{eq:max rotational mass}
m_r &=& C_m \max\left\{{\rm diag}({\bf M}_{\rm rot})\right\}
\end{eqnarray}
%
Here, ${\bf M}_{\rm tra}$ and ${\bf M}_{\rm rot}$ are the translational and
rotational parts, respectively, of the fully assembled mass matrix,
and $C_m$ is a user-defined scaling factor that is specified through the
command-line argument {\tt -autoMassScale} when running the Fedem Link Reducer
(default value is $10^{-9}$).

In some cases, it may happen that the value $m_r$ defined by
\eqnref{eq:max rotational mass}, is identically zero.
For instance, if the finite element model consists of solid elements only,
in addition to at least one WAVGM element with automatically added BUSH and
CMASS elements at its slave node, there are no mass contributions to the
rotational DOFs in the model.
In such cases, a value for $m_r$ is instead derived from the global inertia
tensor, ${\bf I}$, that may be computed from the finite element model, i.e.
%
\begin{equation}
{\bf I} := \left[\begin{array}{ccc}
I_{xx} & I_{xy} & I_{xz} \\
I_{xy} & I_{yy} & I_{yz} \\
I_{xz} & I_{yz} & I_{zz} \\
\end{array}\right] = \int\limits_\Omega \rho \left[\begin{array}{ccc}
y^2 + z^2    & xy        & xz \\
             & x^2 + z^2 & yz \\
\mbox{symm.} &           & x^2 + y^2
\end{array}\right] dV
\end{equation}
%
where $\rho$ is the mass density, and
%
\begin{equation}
m_r \;=\; \frac{C_m}{3} \left( I_{xx} + I_{yy} + I_{zz} \right)
\end{equation}
