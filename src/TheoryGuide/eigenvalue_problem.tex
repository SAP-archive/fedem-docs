% SPDX-FileCopyrightText: 2023 SAP SE
%
% SPDX-License-Identifier: Apache-2.0
%
% This file is part of FEDEM - https://openfedem.org

%%%%%%%%%%%%%%%%%%%%%%%%%%%%%%%%%%%%%%%%%%%%%%%%%%%%%%%%%%%%%%%%%%%%%%%%%%%%%%%%
%
% FEDEM Theory Guide.
%
%%%%%%%%%%%%%%%%%%%%%%%%%%%%%%%%%%%%%%%%%%%%%%%%%%%%%%%%%%%%%%%%%%%%%%%%%%%%%%%%

\section{Eigenvalue results}

The general eigenvalue problem may be written
%
\begin{equation}
{\mf A}{\mf x} = \lambda{\mf B}{\mf x} \quad\text{or}\quad
({\mf A} -\lambda{\mf B}){\mf x} = {\mf 0}
\label{eqEIG:Def}
\end{equation}
%
where $\mf A$ and $\mf B$ need not necessarily be positive definite
or symmetrical.

\subsection{Undamped eigenvalue problem}

The second order differential equation governing free undamped vibration of the
discretized system can be written
%
\begin{equation}
{\mf M}\ddot{\mf v} + {\mf K}{\mf v} = {\mf 0}
\label{eqEIG:DiffUndamped}
\end{equation}
%
Assuming a solution of the form ${\mf v}(t) = {\mf v}_\lambda e^{\lambda_u t}$
gives
%
\begin{equation}
{\mf v}(t) \;=\; {\mf v}_{\lambda} e^{\lambda_u t}, \quad
\dot{\mf v}(t) \;=\; \lambda_u{\mf v}_{\lambda} e^{\lambda_u t}, \quad
\ddot{\mf v}(t) \;= \lambda_u^2{\mf v}_{\lambda} e^{\lambda_u t}
\label{eqEIG:AssumedSol}
\end{equation}
%
Substituting into \eqnref{eqEIG:DiffUndamped} gives
%
\begin{equation}
\left({\mf K} + \lambda_u^2 {\mf M}\right) {\mf v}_{\lambda} \;=\; {\mf 0}
\label{eqEIG:Undamp1}
\end{equation}
%
or
%
\begin{equation}
\left({\mf A} - \lambda{\mf B}\right){\mf x}_{\lambda} \;=\; {\mf 0}
\quad\text{where}\quad
\left\{\begin{array}{l}
{\mf A} \;=\; {\mf K} \\
{\mf B} \;=\; {\mf M} \\
\lambda \;=\; -\lambda_u^2 \quad\Rightarrow\quad \lambda_u = \sqrt{-\lambda}
\end{array}\right.
\label{eqEIG:Undamp2}
\end{equation}
%
When both $\mf A$ and $\mf B$ are positive definite, one will have positive
eigenvalues $\lambda$, and thus purely imaginary values for $\lambda_u$;
in other words, $\lambda_u = \text{i}\,\omega$.
In addition, because $\mf A$ and $\mf B$ are symmetrical, the eigenvectors
${\mf x}_{\lambda}$ are real.

The general solution of \eqsref{eqEIG:DiffUndamped}{eqEIG:Undamp2}
can then be written as
%
\begin{equation}
{\mf v}(t) \;=\; {\mf v}_{\lambda} e^{\text{i}\,\omega t} \;=\;
{\mf v}_{\lambda} \left(\cos\omega t + \text{i}\sin\omega t\right)
\label{eqEIG:UndampSolution}
\end{equation}
%
where the circle frequency $\omega$ is given by $\omega = \sqrt{\lambda}$.

\subsubsection{Animation of the eigenvectors}

The real and imaginary parts of the solution above are solutions of the original
homogeneous \eqnref{eqEIG:DiffUndamped}.
However, when animating the eigenvector, the real part of the solution of
\eqnref{eqEIG:UndampSolution} is displayed.
For example:
%
\begin{equation}
{\mf v}(t) \;=\; {\mf v}_{\lambda} \cos\omega t
\label{eqEIG:UndampAnimation}
\end{equation}



\subsection{Damped eigenvalue problem}

The second order differential equation governing free damped vibration of the
discretized system can be written
%
\begin{equation}
{\mf M}\ddot{\mf v} + {\mf C}\dot{\mf v}+ {\mf K}{\mf v} \;=\; {\mf 0}
\label{eqEIG:DiffDamped}
\end{equation}
%
Assuming a solution on the form ${\mf v}(t) = {\mf v}_{\lambda} e^{\lambda_d t}$
gives
%
\begin{equation}
{\mf v}(t) \;=\; {\mf v}_{\lambda} e^{\lambda_d t}, \quad
\dot{\mf v}(t) \;=\; \lambda_d{\mf v}_{\lambda} e^{\lambda_d t}, \quad
\ddot{\mf v}(t) \;=\; \lambda_d^2{\mf v}_{\lambda} e^{\lambda_d t}
\label{eqEIG:AssumedSol2}
\end{equation}
%
and substituting into \eqnref{eqEIG:DiffDamped} gives
%
\begin{equation}
\left({\mf K} + \lambda_d{\mf C} + \lambda_d^2{\mf M}\right){\mf v}_{\lambda}
\;=\; {\mf 0}
\label{eqEIG:Damp1}
\end{equation}
%
which cannot be cast directly into the form of \eqnref{eqEIG:Def}
unless the damping matrix $\mf C$ is given by Rayleigh damping:
${\mf C} = \alpha{\mf K} + \beta{\mf M}$
(see Section~\ref{sec:RayleighDamping}).

For a general damping matrix, the $n$-dimensional second order system of
\eqnref{eqEIG:DiffUndamped} can be cast into a first order system of dimension
$2n$ by introducing
%
\begin{equation}
{\mf u} \;= \left[\begin{array}{c} {\mf v} \\ \dot{\mf v} \end{array}\right]
\label{eqEIG:uDef}
\end{equation}
%
Equation~\eqref{eqEIG:uDef} allows a number of different ways of writing
\eqnref{eqEIG:DiffUndamped} as a first order equation; but to
preserve possible symmetry, the canonical form is chosen:
%
\begin{equation}
{\mf A u} - {\mf B}\dot{\mf u} \;=\; {\mf 0} \quad\text{where}\quad
\left\{\begin{array}{l}
{\mf A} \;= \left[\begin{array}{cc} {\mf K} & {\mf 0} \\ {\mf 0} & {\mf -M}
            \end{array}\right] \\[5mm]
{\mf B} \;= \left[\begin{array}{cc} {\mf -C} & {\mf -M} \\ {\mf -M} & {\mf 0}
            \end{array}\right]
\end{array}\right.
\label{eqEIG:Canonical}
\end{equation}
%
Assuming the solution on the form ${\mf u} = {\mf u}_\lambda e^{\lambda_d t}$
gives
%
\begin{equation}
{\mf u} \;=\; {\mf u}_\lambda e^{\lambda_d t}, \quad
\dot{\mf u} \;=\; \lambda_d{\mf u}_\lambda e^{\lambda_d t}
\label{eqEIG:AssumedSol3}
\end{equation}
%
and the first order eigenvalue problem has the familiar standard form
%
\begin{equation}
\left({\mf A} - \lambda_d{\mf B}\right){\mf u}_\lambda \;=\; {\mf 0}
\label{eqEIG:FirstOrderEig}
\end{equation}
%
In general, the solution of the damped eigenvalue problem above has
complex eigenvalues $\lambda_d$ and complex eigenvectors ${\mf u}_\lambda$:
%
\begin{equation}
\lambda_d \;=\; \mu + \text{i}\omega \qquad\text{and}\qquad
{\mf u}_\lambda \;=\; {\mf u}_{\Re} + \text{i}{\mf u}_{\Im} \;=
\left[\begin{array}{c}{\mf v}_{\Re} \\ \dot{\mf v}_{\Re} \end{array}\right] +
\text{i}\left[\begin{array}{c}{\mf v}_{\Im} \\ \dot{\mf v}_{\Im} \end{array}\right]
\label{eqEIG:EigSolDamp}
\end{equation}
%
The imaginary part $\omega$ signifies circle frequency, as in the undamped
problem; whereas the real part $\mu$ represents the damping of the eigenmode.
This value is (usually) a negative number, and tells at what rate the amplitude
of the free vibration will decrease.

A positive-values $\mu$ signifies an unstable system in which the amplitude
is increasing.
This can happen with axially loaded structures in which the axial load
is higher than the critical buckling load.
This gives a negative definite stiffness matrix and possible negative damping.

A negative-valued $\mu$ without an accompanying $\omega$, such as a real
eigenvalue, signifies a critically damped mode.

\subsubsection{Animation of complex eigenvectors}

Extracting the $n$ dimensional vector ${\mf v}_\lambda$ from the solution
\eqnref{eqEIG:AssumedSol3} and combining it with \eqnref{eqEIG:EigSolDamp} gives
%
\begin{equation}
\label{eqEIG:DampedSol}
\eqalign{{\mf v}(t)
\;=\; & e^{\lambda_d t}{\mf v}_\lambda
\;=\; e^{(\mu + \text{i}\omega)t}{\mf v}_\lambda
\;=\; e^{\mu t} e^{\text{i}\omega t} ({\mf v}_{\Re} + \text{i}{\mf v}_{\Im}) \cr
  =\; & e^{\mu t}(\cos\omega t + \text{i}\sin\omega t)
                 ({\mf v}_{\Re} + \text{i}{\mf v}_{\Im}) \cr
  =\; & e^{\mu t}\left[({\mf v}_{\Re}\cos\omega t - {\mf v}_{\Im}\sin\omega t) +
          \text{i}({\mf v}_{\Im}\cos\omega t +{\mf v}_{\Re}\sin\omega t)\right]}
\end{equation}
%
As in the undamped case, the real part of the solution is selected for animation
of the eigenvalue and eigenvector:
%
\begin{equation}
{\mf v}_{\Re}(t) \;=\;
e^{\mu t}({\mf v}_{\Re}\cos\omega t - {\mf v}_{\Im}\sin\omega t)
\end{equation}

\subsubsection{Diagonalizing the damped eigenvalue problem}

Having calculated a set of eigenvectors, ${\mf u}_{\lambda_i}$, the canonical
form~\eqref{eqEIG:Canonical}, the quadratic expression~\eqref{eqEIG:Damp1},
or the original differential \eqnref{eqEIG:DiffUndamped} can be diagonalized.
Of primary interest is diagonalizing the mass, stiffness
and damping matrices of \eqsref{eqEIG:Damp1}{eqEIG:DiffDamped}:
%
\begin{eqnarray}
m_i &=& \bar{\mf v}_{\lambda_i}^T {\mf M}{\mf v}_{\lambda_i} \nonumber \\
c_i &=& \bar{\mf v}_{\lambda_i}^T {\mf C}{\mf v}_{\lambda_i} \\
k_i &=& \bar{\mf v}_{\lambda_i}^T {\mf K}{\mf v}_{\lambda_i} \nonumber
\label{eq:modalMCK}
\end{eqnarray}
%
where $\bar{\mf v}_{\lambda_i}$ is the complex conjugate of ${\mf v}_{\lambda_i}$.
In other words:
%
\begin{equation}
{\mf v} \;=\; {\mf v}_{\Re} + \text{i}{\mf v}_{\Im}
\quad\text{and}\quad
\bar{\mf v} \;=\; {\mf v}_{\Re} - \text{i}{\mf v}_{\Im}
\end{equation}


\subsection{Calculating damping ratios}

Once the eigenvalues and eigenvectors have been calculated,
one can calculate damping ratios for the different eigenmodes.
For a single-DOF system with mass $m$, stiffness $k$ and damping $c$,
one has the undamped eigenvalue given by
%
\begin{equation}
\omega \;=\; \sqrt{\frac{k}{m}}
\end{equation}

For a system with damping one can show that no true vibration will occur when
the damping $c$ is sufficiently high; called critical damping $c_\textit{cr}$.
In this case the dynamic motion will be an asymptotic motion to the static
equilibrium position.
The critical damping is given by
%
\begin{equation}
c_\textit{cr} \;=\; 2 m \omega %= 2 m \sqrt{\frac{k}{m}} = 2 \sqrt{k m}
\end{equation}

The critical damping is a convenient reference point for describing
the amount of damping in a system.
This is done by defining the damping ratio
%
\begin{equation}
\xi \;=\; \frac{c}{c_\textit{cr}} = \frac{c}{2 m \omega}
\end{equation}
%
This is often given in percent (\%) of the critical damping.

\subsubsection{Based on damped eigenvalue problem}

The eigenvectors ${\mf v}_{\lambda_i}$ for each eigenmode $i$ of the damped
system will diagonalize the mass, stiffness, and damping matrix
as given by \eqnref{eq:modalMCK}.
This gives the damping ratio for eigenmode $i$ as
%
\begin{equation}
\xi_i \;=\; \frac{c_i}{c_{\textit{cr}_i}} = \frac{c_i}{2 \sqrt{k_i m_i}}
\end{equation}

\subsubsection{Based on undamped eigenvalue problem}

The eigenvectors ${\mf v}_{\lambda_i}$ for each eigenmode $i$ of the undamped
system will diagonalize the mass matrix, i.e. create the modal mass
%
\begin{equation}
m_i \;=\; {\mf v}_{\lambda_i}^T {\mf C}{\mf v}_{\lambda_i}
\label{eq:modalMass}
\end{equation}

An approximate damping coefficient can be calculated based on the
same eigenvector
%
\begin{equation}
c_i \;\approx\; {\mf v}_{\lambda_i}^T {\mf C}{\mf v}_{\lambda_i}
\label{eq:approxDampingCoeff}
\end{equation}
%
which gives an approximate damping ratio
%
\begin{equation}
\xi_i \;\approx\; \frac{c_i}{2 m_i \omega_i} = \frac{c_i}{2\sqrt{k_i m_i}}
\end{equation}

Note that if the damping matrix constitutes Rayleigh damping, see
Section~\ref{sec:RayleighDamping}, the damping ratios calculated above are exact.
For damping ratios $\ll 1$ the approximation is very good
also for a more general damping matrix.


\subsection{Using shift when solving the eigenvalue problem}

To improve the convergence of the iterations, or to find eigenvalues in the
neighborhood of a particular value, a \textit{shift} can be applied to the
initial eigenvalue problem.
The eigenvalue problem $({\mf K} - \lambda {\mf M}){\mf v} = {\mf 0}$ may be
written as the equivalent expression
%
\begin{equation}
\left( \hat{\mf K} - \hat{\lambda}{\mf M} \right) {\mf v} \;=\; {\mf 0}
\quad\text{where}\quad
\hat{\mf K} \;=\; {\mf K} - \mu{\mf M}
\label{eqK25:21}
\end{equation}
%
and $\mu$ is the shift value.
The modified eigenvalues $\hat{\lambda}_i$ are given by the relation
%
\begin{equation}
\hat{\lambda}_i \;=\; \lambda_{i} - \mu
\end{equation}
%
where $\lambda_i$ represents the eigenvalues and  the eigenvectors are the same
as those in the original eigenvalue problem.

Matrix $\hat{\mf K}$ must be invertible.
If, for instance, the original $\mf K$ is not positive definite, as is the case
with a free structure, a negative shift can be applied that makes the matrix
$\hat{\mf K}$ positive definite.
This makes it possible to find the zero frequency (rigid body) modes of a free
system, as well as the other lowest eigenvalues and eigenvectors.

The iteration procedure for eigenvalue solvers, such as Lanczos and sub-space
iterations, converges toward a set of eigenvectors that have corresponding
eigenvalues closest to $\mu$, which are a set of eigenvalues with
$\hat{\lambda}_i=\lambda_i-\mu$ and the smallest absolute value.
This procedure can be used to add desired generalized modes around a particular
frequency for the CMS-reduction of Chapter~\ref{chap:CMS}.
