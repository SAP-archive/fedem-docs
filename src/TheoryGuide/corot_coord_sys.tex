% SPDX-FileCopyrightText: 2023 SAP SE
%
% SPDX-License-Identifier: Apache-2.0
%
% This file is part of FEDEM - https://openfedem.org

%%%%%%%%%%%%%%%%%%%%%%%%%%%%%%%%%%%%%%%%%%%%%%%%%%%%%%%%%%%%%%%%%%%%%%%%%%%%%%%%
%
% FEDEM Theory Guide.
%
%%%%%%%%%%%%%%%%%%%%%%%%%%%%%%%%%%%%%%%%%%%%%%%%%%%%%%%%%%%%%%%%%%%%%%%%%%%%%%%%

\section{Local element coordinate system}
\label{sec:RefDef}

The element matrices are all expressed in a local element coordinate system.
For superelements this is the coordinate system in which the finite element
mesh was modeled at substructure level, and for generic parts this is simply
the local element coordinate system of the underlying rigid spider.
In the system analysis, this local coordinate system becomes the {\it Element
Co-rotated Coordinate System} (ECCS) because it moves with the element.

Regarding the nonlinear behavior of an element, the co-rotational formulation is
independent of whether the element is a reduced superelement, resulting from the
Component Mode Synthesis reduction described in Section~\ref{sec:CMS reduction},
a generic part element as described in Section~\ref{sec:GenericPart}, or a more
traditional element as the other ones described in Appendix~\ref{chap:ElLibrary}.
In the following description of the co-rotational formulation,
element and superelement are thus interchangeable terms.

To calculate the transformations of the elements at each new position,
the directions for the corresponding ECCS must first be calculated based on the
positions of the element nodes.

%%%%%%%%%%%%%%%%%%%%%%%%%%%%%%%%%%%%%%%%%%%%%%%%%%%%%%%%%%%%%%%%%%%%%%%%%%%%%%%%
\subsection{Method 1: Best fit of max sized triangle}
\label{sec:RefDefMeth1}

In this method, the ECCS is connected to three nodal points for each element.
The goal is to select among all nodal points within an element,
those three that form the largest triangle.
This is achieved by the following algorithm:
%
\begin{enumerate}
\item The first reference point, $R_1$,
is selected as the node furthest away from the centroid of all element nodes.
This first reference point is always at the node itself, i.e., zero offset.
%
\item The second reference point, $R_2$,
is selected as the node furthest away from the point $R_1$.
If the element only has one node the same node is used again,
but with the second reference point defined by an offset of unit length
along the global $x$-axis from the nodal position.
%
\item The third reference point, $R_3$,
is selected as the node furthest from a line through the points $R_1$ and $R_2$.
If the element has only one or two nodes, or the third node is on (or close to)
this straight line\footnote{The criterion for whether the third node is on the
straight line between $R_1$ and $R_2$ can be configured.
Originally, this tolerance was absolute and equal to 0.001 regardless of the
model dimensions. Later, it was changed to half the distance between the first
two points, i.e., $0.5|\vec{R}_{12}|$ but it is now configurable by setting
the relative tolerance parameter $\epsilon$ in the absolute tolerance
$\epsilon|\vec{R}_{12}|$. The default value on $\epsilon$ is now 0.05.},
the reference point is defined using an offset from the
third node so that the three points form a proper triangle\footnote{
The offset on the third node (when needed) was originally selected
simply as a vector of unit length away from the straight line.
This is clearly not a good strategy if the distance between the first two nodes
is much greater (or much less) than 1,
since the resulting triangle then will be far from even-sided.
These shortcomings of the unit offset spurred an improved method where the
offset is scaled with respect to the distance between the first two points.
This results in a more even sided triangle, and a more robust coordinate system.
The original method is available for backward compatibility of older models.}.
The direction of the third-node offset vector is obtained by rotating the vector
$\vec{R}_{12}$ from $R_1$ to $R_2$ 90 degrees about either the global $x$-, $y$-
or $z$-axis, depending on along which global axis direction the $\vec{R}_{12}$
vector has its smallest component.
%
\item A temporary coordinate system is now formed based on the positions
of the three reference points.
The origin is defined at $R_1$,
the $x$-axis is defined in the direction from $R_1$ to $R_2$,
the $z$-axis is next calculated from the cross product of the $x$-axis and the
axis extending from $R_1$ to $R_3$, and finally the $y$-axis is calculated from
the cross product of the $z$- and $x$-axes.
During the analysis, this coordinate system is recalculated for each new
position of the three reference points.
%
\item The position of the triangle system is always fixed in relation to the
ECCS during the analysis.
\end{enumerate}

\clearpage
Using the $4\times4$ position matrix notation defined in \eqnref{eq:posDef}, the
{\underline p}osition matrix, ${\mf P}_{TE}$, relates the {\underline t}riangle
coordinate system to the {\underline e}lement coordinate system:
%
\begin{equation}
{\mf P}_{TE} \;=\; \left[\barr{cc}
{\mf R}_{TE} & {\mf p}_1 \\
{\mf 0} & 1 \earr\right]
\end{equation}
%
where ${\mf R}_{TE}$ is the {\underline r}otation or orientation of the
{\underline t}riangle system measured in the {\underline e}lement system,
and ${\mf p}_1$ is the {\underline p}osition of the triangle system relative
to the element system (${\mf p}_1$ is the position of $R_1$ measured in the
element coordinate system).

It follows that the positions of the reference points measured in global
coordinates establish ${\mf P}_{TG}$, the position of the {\underline t}riangle
coordinate system in relation to the {\underline g}lobal system.
The following relationship then relates the triangle,
element and global coordinate systems:
%
\begin{equation}
{\mf P}_{TG} \;=\; {\mf P}_{EG}{\mf P}_{TE}
\end{equation}
%
where ${\mf P}_{EG}$ is the {\underline e}lement coordinate system relative to
the {\underline g}lobal system, and is stated
%
\begin{equation}
{\mf P}_{EG} \;=\; {\mf P}_{TG}{\mf P}_{TE}^{-1}
\end{equation}

%%%%%%%%%%%%%%%%%%%%%%%%%%%%%%%%%%%%%%%%%%%%%%%%%%%%%%%%%%%%%%%%%%%%%%%%%%%%%%%%
\subsection{Method 2: Mass based weighted average}
\label{sec:RefDefMeth2}

This method ultimately gives an expression requiring a weighted average of the
nodal discrepancies between the actual deformed element and the undeformed
shadow element to be zero.
This expression can be established from an equilibrium point of view.

One assumes that the (rigid) shadow element is attached to the deformed element
with springs, rotational and translational, at each node.
The spring stiffnesses at each node is selected to be, for instance,
proportional to the mass for translation and proportional to the inertia for
the rotation of each node.
When the nodes move, the shadow element will reposition itself according to
equilibrium based on the redistribution of nodal displacements.
The rigid body motion of the shadow element is then represented by the motion of
a fictitious node at the Center of Gravity (CG) of the shadow element.

The equilibrium equations for the shadow element can be established about CG.
This yields six equations for the movement of the shadow element as a function
of the nodal displacements.
The equations end up being non-linear and have to be solved iteratively.

With $k_t$ as the mass-proportional translational stiffness in all three
directions at each node, and $k_r$ as inertia-proportional rotational
stiffness in all three directions, the stiffness of a spring element
with two co-located nodes $i$ and $j$ is given by
%
\begin{equation}
\left[\begin{array}{c}
{\mf f}_i \\ {\mf m}_i \\ {\mf f}_j \\ {\mf m}_j \\
\end{array}
\right]
=\left[\begin{array}{cccc}
 k_t {\mf I}  & {\mf 0}  & -k_t {\mf I}  & {\mf 0} \\
 {\mf 0} &  k_r {\mf I}  & {\mf 0} &  -k_r {\mf I} \\
 -k_t {\mf I}  & {\mf 0}  & k_t {\mf I}  & {\mf 0} \\
 {\mf 0} &  -k_r {\mf I}  & {\mf 0} &  k_r {\mf I} \\
\end{array}\right]
\left[\begin{array}{c}
{\mf v}_i \\ {\tf \theta}_i \\ {\mf v}_j \\ {\tf \theta}_j \end{array}\right]
\label{eqCS2_nodeStiff}
\end{equation}

With node $j$ being a dependent node rigidly attached to the node at \textit{CG},
one can establish the virtual displacement equation
%
\begin{equation}
{\mf v}_j \;=\; {\mf E}_i {\mf v}_\textit{CG}
\end{equation}
or
\begin{equation}
\left[\begin{array}{c}
{\mf v}_j \\ {\tf \theta}_j \end{array}\right] =
\left[\begin{array}{cc}
{\mf 1} & -\widehat{\mf e}_i \\
{\mf 0} & {\mf 1}
\end{array}\right]
\left[\begin{array}{c}
{\mf v}_\textit{CG} \\ {\tf \theta}_\textit{CG} \end{array}\right]
\quad\text{where}\quad
{\mf e}_i = \left[\begin{array}{c}
x_i - x_\textit{CG} \\
y_i - y_\textit{CG} \\
z_i - z_\textit{CG}
\end{array}\right]
\label{eq:rigDispEcc}
\end{equation}

Using the kinematic relationship above in a virtual work expression,
one can establish the stiffness matrix for the spring between node $i$
and the \textit{CG} node as
%
\begin{equation}
\left[\begin{array}{c}
{\mf f}_i \\ {\mf m}_i \\ {\mf f}_\textit{CG} \\ {\mf m}_\text{CG} \\
\end{array}\right]
=
\left[\begin{array}{cccc}
 {\mf k}_t                   &  {\mf 0}    & -{\mf k}_t &  {\mf k}_t \widehat{{\mf e}} \\
 {\mf 0}                     &  {\mf k}_r  &  {\mf 0}   & -{\mf k}_r \\
-{\mf k}_t                   &  {\mf 0}    &  {\mf k}_t & -{\mf k}_t \widehat{{\mf e}} \\
 \widehat{\mf e}^T {\mf k}_t & -{\mf k}_r  & -\widehat{\mf e}^T {\mf k}_t &
 ({\mf k}_r + \widehat{\mf e}^T{\mf k}_T \widehat{\mf e}) \\
\end{array}\right]
\left[\begin{array}{c}
{\mf v}_i \\ {\tf \theta}_i \\ {\mf v}_\textit{CG} \\ {\tf \theta}_\text{CG} \\
\end{array}\right]
\end{equation}
%
where ${\mf k}_t = k_t{\mf I}$ and ${\mf k}_r = k_r{\mf I}$.

The full stiffness matrix for the shadow element is then formed by the assembly
of all nodal contributions.
Rewriting the equation above with a slightly different notation:
%
\begin{equation}
\left[\begin{array}{c}
{\mf f}^i_1 \\ {\mf f}^i_2 \\
\end{array}\right]
=
\left[\begin{array}{cc}
 {\mf k}^i_{11} &  {\mf k}^i_{12} \\
 {\mf k}^i_{21} &  {\mf k}^i_{22} \\
\end{array}\right]
\left[\begin{array}{c}
{\mf v}^i_1 \\ {\mf v}^i_2\\
\end{array}\right]
\end{equation}
%
the assembly for all nodes from $1$ to $n$ (where $n+1$ is the \textit{CG} node)
gives the shadow element stiffness matrix:
%
\begin{equation}
\left[\begin{array}{c}
{\mf f}^1_1 \\ \vdots \\ {\mf f}^i_1 \\ \vdots \\ {\mf f}_\textit{CG} \\
\end{array}\right]
=
\left[\begin{array}{ccccc}
 {\mf k}^1_{11} & \hdots & {\mf 0}        & \hdots   &  {\mf k}^1_{12} \\
 \vdots         & \ddots &                &          &  \vdots         \\
 {\mf 0}        &        & {\mf k}^i_{11} &          &  {\mf k}^i_{12}  \\
 \vdots         &        &                & \ddots   &  \vdots         \\
 {\mf k}^1_{21} & \hdots & {\mf k}^i_{21} & \hdots   &
\left(\sum\limits_{i=1}^n {\mf k}^i_{22}\right) \\
\end{array}\right]
\left[\begin{array}{c}
{\mf v}^1_1 \\ \vdots \\ {\mf v}^i_1 \\ \vdots \\ {\mf v}_\textit{CG} \\
\end{array}\right]
\end{equation}

The equilibrium equation at the \textit{CG} node is seen as the last 6 rows
of the assembled shadow element stiffness matrix, which can be rewritten as
%
\begin{equation}
{\mf f}_\textit{CG} \;=\; \sum_{i=1}^n{\mf k}^i_{21} {\mf v}_i
\quad\text{or}\quad
\left[\begin{array}{c}
{\mf f}_\textit{CG} \\ {\mf m}_\text{CG} \\
\end{array}\right]
\;=\; \sum_{i=1}^n
\left[\begin{array}{cc}
-{\mf k}^i_t               &  {\mf 0}     \\
 \hat{\mf e}^T {\mf k}^i_t & -{\mf k}^i_r \\
\end{array}\right]
\left[\begin{array}{c}
{\mf v}_i \\ {\tf \theta}_i \\
\end{array}\right]
\end{equation}

If the above equation does not yield zero forces and moments
${\mf f}_\textit{CG}$ at the \textit{CG} node when the deformational
\iftoggle{publicedition}{% The following is for the public version
displacement vector is used,
}{% The following is for the in-house edition only
displacement vector ${\mf v}_d$ of \eqnref{eqSC:supelDefDisp} is used,
}
the shadow element is given a corrective translation and rotation that will
again alter the deformation vector for all the nodes in according to
\eqnref{eq:rigDispEcc}, such that the unbalanced forces and moments vanish.

Equilibrium of the \textit{CG} node gives
%
\begin{equation}
{\mf f}_\textit{CG} \;=\;
\sum_{i=1}^n{\mf k}^i_{21}{\mf v}_i =
\sum_{i=1}^n{\mf k}^i_{21}{\mf E}_i\,\Delta{\mf v}_\textit{CG}
\end{equation}
%
Solving this with respect to increment in Shadow element position
$\Delta{\mf v}_\textit{CG}$ gives
%
\begin{equation}
\Delta{\mf v}_\textit{CG} \;=\;
\left(\sum_{i=1}^n{\mf k}^i_{21}{\mf E}_i\right)^{-1}
\sum_{i=1}^n{\mf k}^i_{21}{\mf v}_{d_i}
\label{eq:refSysMeth2}
\end{equation}

The equation above is solved repeatedly until the deformational displacement
vector ${\mf v}_d$ produces no update to the shadow element position given by
$\Delta{\mf v}_\textit{CG}$.
