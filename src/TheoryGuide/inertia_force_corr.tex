% SPDX-FileCopyrightText: 2023 SAP SE
%
% SPDX-License-Identifier: Apache-2.0
%
% This file is part of FEDEM - https://openfedem.org

%%%%%%%%%%%%%%%%%%%%%%%%%%%%%%%%%%%%%%%%%%%%%%%%%%%%%%%%%%%%%%%%%%%%%%%%%%%%%%%%
%
% FEDEM Theory Guide.
%
%%%%%%%%%%%%%%%%%%%%%%%%%%%%%%%%%%%%%%%%%%%%%%%%%%%%%%%%%%%%%%%%%%%%%%%%%%%%%%%%

\def\CGtiny{{\mbox{\tiny CG}}}
\def\CRtiny{{\mbox{\tiny CR}}}

\section{Inertia forces and high-speed rotation}
\label{s:Inertia forces and high-speed rotation}

When performing FE-based dynamics analysis with high-speed rotation, the inertia
forces can be poorly represented if superelements with very few external DOFs
have motion with high-speed rotation components.

This problem is usually only encountered with rigid body analysis, in which the
number of external DOFs for each superelement is too limited to calculate the
stresses with any accuracy.
However, such a model is often desirable, as it can provide fairly accurate
motion and joint force characteristics while being very efficient numerically.
The cause of the problem is that model reduction (as described earlier in
Chapter~\ref{chap:CMS}) invariably reduces the interpolation order of
displacements, velocities and accelerations over the superelement,
thereby producing a poor representation of the centripetal acceleration field
for the superelement or element mesh.

There are two remedies for this problem:
%
\begin{enumerate}
\item Using an adequate number of external nodes to represent the centripetal
acceleration field.
This generally means selecting nodes that effectively span the complete
superelement and surround the center of gravity.
\item Using an automatic algorithm to correct the inertia forces with respect to
rigid body resultants.
\end{enumerate}

Remedy 1 is not always desirable, as it mandates the use of more DOFs than are
strictly necessary to achieve satisfactory result.
In addition, it takes some experience to determine an adequate selection of
external nodes.

In order to eliminate the guesswork from remedy 1 above, a procedure to obtain
accurate rigid body inertia forces while still allowing a minimal number of
external DOFs has been developed.
The procedure uses the rigid body mass properties for each superelement to check
consistency between the nodal inertia forces and rigid body resultant forces.
If they are not consistent, the nodal inertia forces are corrected to ensure
correct rigid body resultants.

It is important to note that if a fairly large number of external nodes are
chosen in order to obtain good stress results for a particular superelement,
a good representation of the acceleration field is also ensured and the inertia
force corrections outlined in this section are reduced (approaches zero).
In this way the additional correction forces will not affect the stress results.

\subsection{Rigid body mass properties}
\label{subs:Rigid body mass properties}

The rigid body mass properties for a structure can be calculated from the
FE~representation of the structure---either the full mass matrix $\mf M$,
or the reduced mass matrix ${\mf M}_r$ given by \eqnref{eqCMS:ReducedModelEq}.
It can be shown that both $\mf M$ and ${\mf M}_r$ yield the same mass property
matrix, ${\mf I}_a$, about a given point $a$.

\subsubsection{Inertia matrix about an arbitrary point}

For a given structure, an $n_{\text{\tiny DOFs}}\times 6$ matrix for the rigid
body displacement vector can be formed, in which each column consists of a rigid
body displacement vector of unit magnitude:
%
\begin{equation}
{\mf R}_a = \left[ \begin{array}{*6c}
{\mf v}_{x} & {\mf v}_y & {\mf v}_z &
{\mf v}_{\omega_x} & {\mf v}_{\omega_y} & {\mf v}_{\omega_z}
\end{array}\right]
\end{equation}
%
Using this matrix in a congruence transformation of the mass matrix yields
%
\begin{equation}
\eqalign{
{\mf I}_a =\;& {\mf R}_a^T{\mf M}_r {\mf R}_a \cr =\;&
\left[\begin{array}{*6c}
m & 0 & 0 & 0 & mz_\CGtiny & -my_\CGtiny \\
0 & m & 0 & -mz_\CGtiny & 0 & mx_\CGtiny \\
0 & 0 & m & my_\CGtiny & -mx_\CGtiny & 0 \\
0 & -mz_\CGtiny & my_\CGtiny & I_{xx_a} & I_{xy_a} & I_{xz_a} \\
mz_\CGtiny & 0 & -mx_\CGtiny & I_{yx_a} & I_{yy_a} & I_{yz_a} \\
-my_\CGtiny & mx_\CGtiny & 0 & I_{zx_a} & I_{zy_a} & I_{zz_a}
\end{array}\right]
}\label{eq:Ia}
\end{equation}

\subsubsection{Inertia matrix about center of gravity}

From the off-diagonal submatrices in \eqnref{eq:Ia},
the position of the Center of Gravity (CG) can be established.
Performing the transformation about the CG yields
%
\begin{equation}
{\mf I}_\CGtiny = {\mf R}_\CGtiny^T {\mf M}_r {\mf R}_\CGtiny =
\left[\begin{array}{*6c}
m & 0 & 0 & 0 & 0 & 0 \\
0 & m & 0 & 0 & 0 & 0 \\
0 & 0 & m & 0 & 0 & 0 \\
0 & 0 & 0 & I_{xx} & I_{xy} & I_{xz} \\
0 & 0 & 0 & I_{yx} & I_{yy} & I_{yz} \\
0 & 0 & 0 & I_{zx} & I_{zy} & I_{zz}
\end{array}\right]
\label{eq:Icg}
\end{equation}

\begin{figure}[b]
\center{
\setlength{\unitlength}{1mm}
\begin{picture}(55,70)(15,2)
\thinlines
% Global coordinate system
\thicklines
\put( 37,53){CG}
\put( 40,50){\vector(3, 1){20}}\put(63,57){${\tf n}_1$}
\put( 40,50){\vector(2,-1){15}}\put(59,41){${\tf n}_2$}
\put( 40,50){\vector(0,-1){25}}\put(43,25){${\tf n}_3$}
\thinlines
\put( 46,52){\line(2,-1){4}}
\put( 44,48){\line(3, 1){6}}
\thicklines
\put( 40,50){\vector(1,0){10}}\put(51,49){${\tf u}_\CGtiny$}
\put( 19,3){\vector(3, 1){40}}
\put( 19,3){\vector(3, 1){38}}\put(62,17){${\tf\omega}_\CGtiny$}
\thinlines
\put( 40,50){\line(0,-1){40}}\put(39,6){CR}
\put( 40,15){\line(3, 1){4.5}}
\put( 44.5,11.5){\line(0, 1){5}}
% The ``potato''
\thicklines
\qbezier(27,40)( 5,46)(25,52)
\qbezier(25,52)(32,54)(37,62)
\qbezier(37,62)(46,74)(55,60)
\qbezier(55,60)(63,50)(53,44)
\qbezier(53,44)(40,36)(27,40)
%\graphpaper[2](0,0)(70,70)
\end{picture}
}
\caption{Rotation axis of screw-like motion}
\label{fig:screw-like motion}
\end{figure}

\subsection{Extracting rigid body velocity}
\label{subs:Extracting rigid body velocity}

The element velocity vector is composed of both rigid body displacements,
${\mf v}_r$, and deformational displacements, ${\mf v}_d$, where the
deformational displacements are orthogonal to rigid body displacements, i.e.
%
\begin{equation}
\dot{\mf v} \;=\; \dot{\mf v}_r + \dot{\mf v}_d \;=\;
{\mf R}_\CGtiny \dot{\mf a} + \dot{\mf v}_d
\quad\mbox{where}\quad
{\mf R}_\CGtiny^T \dot{\mf v}_d = {\mf 0}
\end{equation}
%
The rigid body velocities can thus be established as
%
\begin{equation}
\dot{\mf a} \;=\; \left({\mf R}_\CGtiny^T {\mf R}_\CGtiny\right)^{-1}
{\mf R}_\CGtiny^T \dot{\mf v}
\quad\mbox{where}\quad
\dot{\mf a} \;= \left[\begin{array}{c}
\dot{u}_\CGtiny \\ \dot{v}_\CGtiny \\ \dot{w}_\CGtiny \\
\omega_{x} \\ \omega_{y} \\ \omega_{z}
\end{array}\right] = \left[\begin{array}{c}
\dot{\mf u}_\CGtiny \\
{\tf\omega}_\CGtiny
\end{array}\right]
\label{eq:rigid velocity}
\end{equation}
%
which, again, can be used to establish the axis of rotation relative to CG
(see Figure~\ref{fig:screw-like motion})
%
\begin{equation}
{\tf n}_1 \;=\; \frac{\tf\omega}{\|\tf\omega\|},\quad
{\tf n}_3 \;=\; \frac{\dot{\tf u}_\CGtiny\times{\tf n}_1}{\|\dot{\tf u}_\CGtiny
\times{\tf n}_1\|},\quad
{\tf n}_2 \;=\; {\tf n}_3\times{\tf n}_1
\end{equation}
%
The axis of rotation can then be established as
%
\begin{equation}
{\tf r}_\CRtiny \;=\; r_\CRtiny{\tf n}_3 \quad\mbox{where}\quad
r_\CRtiny \;=\; \frac{\dot{\tf u}_\CGtiny \cdot {\tf n}_2}{\|{\tf\omega}\|}
\end{equation}

\subsection{Rigid body inertia forces}
\label{subs:Rigid body inertia forces}

From the rigid body velocities established in \eqnref{eq:rigid velocity}
and the rigid body properties in \eqnref{eq:Icg},
the forces about the Center of Rotation (CR) can be established as follows
%
\begin{eqnarray}
\mbox{Force:~~}  {\tf f}_{RB} &=&
m\,{\tf\omega}\times\left({\tf\omega}\times{\tf r}_\CRtiny\right) \\
\mbox{Torque:~~~}{\tf t}_{RB} &=&
\dot{\tf L}_\CRtiny = \dot{\tf L}_\CGtiny =
\tf\omega\times\left({\tf I}_\CGtiny\cdot\tf\omega\right)
\end{eqnarray}

\subsection{FE Resultant inertia forces}
\label{subs:FE Resultant inertia forces}

Having established the rigid body velocity vector, the associated FE
acceleration vector for this $\ddot{\mf v}_{\omega_\CRtiny}$ can be established.

The virtual work equation about the CR is
%
\begin{eqnarray}
\mbox{Force:~~~} {\mf f}_{FE} &=&
\delta{\mf v}_f^T {\mf M} \ddot{\mf v}_{\omega_\CRtiny} \label{eq:Ffe} \\
\mbox{Torque:~~~}{\mf t}_{FE} &=&
\delta{\mf v}_t^T {\mf M} \ddot{\mf v}_{\omega_\CRtiny}
\end{eqnarray}
%
where $\delta{\mf v}_f$ contains virtual displacement vectors for unit
translations in $X$, $Y$ and $Z$, whereas $\delta{\mf v}_t$ contains virtual
displacement vectors for rotations $X$, $Y$ and $Z$ about CR.

\subsection{Correction to FE inertia forces}
\label{subs:Correction to FE inertia forces}

For models with few DOFs, there is often a discrepancy between the forces
resulting from the FE mass matrix and the rigid body force resultants.
The correction of the FE resultant forces should then be
%
\begin{eqnarray}
\mbox{Force:~~~} {\mf f}_C &=& {\mf f}_{RB} - {\mf f}_{FE} \label{eq:Fc} \\
\mbox{Torque:~~~}{\mf t}_C &=& {\mf t}_{RB} - {\mf t}_{FE} \label{eq:Tc}
\end{eqnarray}

Forces and torques must now be added at the available nodes in such a way that
the resultants after the correction are equal to those of \eqsref{eq:Fc}{eq:Tc}.
The solution to this is not unique, but from an empirical or heuristic
viewpoint, the following procedure has been adopted:
%
\begin{enumerate}
\item The resultant force correction is evenly distributed to each node $i$
on the superelement
%
\begin{equation}
{\mf f}_{C_i} \;=\; \frac{1}{n_{\mbox{\tiny nodes}}}{\mf f}_C
\label{eq:Fci}
\end{equation}
%
This, of course, alters the torque about CR, which requires additional
compensation when correcting the resultant torque.
%
\item The new FE resultant torque, which includes the forces above, becomes
%
\begin{equation}
{\mf t}_{FE_2} \;=\; \delta{\mf v}_t^T \left(
{\mf M}\ddot{\mf v}_{\omega_\CRtiny} + {\mf f}_{\mbox{\tiny corr}}
\right)
\end{equation}
%
when combining the forces from \eqsref{eq:Ffe}{eq:Fci}.
The resultant corrected torque is
%
\begin{equation}
{\mf t}_{C_2} \;=\; {\mf t}_{RB} - {\mf t}_{FE_2}
\end{equation}
%
and the accompanying nodal correction torque is
%
\begin{equation}
{\mf t}_{C_i} \;=\; \frac{1}{n_{\mbox{\tiny nodes}}}{\mf t}_{C_2}
\label{eq:Tci}
\end{equation}
%
\item Combining the \eqsref{eq:Fci}{eq:Tci}
into the superelement correction vector results in
%
\begin{equation}
{\mf F}^{I_c} \;= \left[\begin{array}{c}
{\mf f}^{C_1} \\ {\mf t}_{C_1} \\ \vdots \\ {\mf f}_{C_n} \\ {\mf t}_{C_n}
\end{array}\right]
\end{equation}
%
\end{enumerate}

\remark{Models with high-speed rotation components are sometimes integrated more
accurately using standard Newmark rather than Newmark with numerical damping.
Using a low damping factor for the HHT-$alpha$ method improves accuracy.
See Section~\ref{s:Newmark time integration}
and~\ref{s:Newmark integration with numerical damping} with remarks.}
