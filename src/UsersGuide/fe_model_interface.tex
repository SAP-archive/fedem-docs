% SPDX-FileCopyrightText: 2023 SAP SE
%
% SPDX-License-Identifier: Apache-2.0
%
% This file is part of FEDEM - https://openfedem.org

%%%%%%%%%%%%%%%%%%%%%%%%%%%%%%%%%%%%%%%%%%%%%%%%%%%%%%%%%%%%%%%%%%%%%%%%%%%%%%%%
%
% FEDEM User Guide.
%
%%%%%%%%%%%%%%%%%%%%%%%%%%%%%%%%%%%%%%%%%%%%%%%%%%%%%%%%%%%%%%%%%%%%%%%%%%%%%%%%

\def\LinkFormatText#1{
  \medskip
  {\raggedright\tt#1}
  \medskip}

\Chapter{FE Model Interface}{fe-model-interface}

Finite Element (FE) models are generated in external
Computer Aided Engineering (CAE) systems and stored in separate files,
which Fedem refers to as {\sl FE data files}.
Fedem stores these files in the Fedem Technology Link format (\File{.ftl}), and
has import filters for the Nastran Bulk Data Format (\File{.nas} \File{.bdf}),
the SESAM Input Interface File format (\File{.fem}),
and the older Fedem Link Model format (\File{.flm}).
This appendix describes these files and their formats.

Sections in this appendix address the following topics:

\begin{itemize}
\item
  \protect\hyperlink{fedem-technology-link-format}
                    {Fedem Technology Link format}
\item
  \protect\hyperlink{nastran-bulk-data-file-format}
                    {Nastran Bulk Data File format}
\item
  \protect\hyperlink{sesam-input-interface-file-format}
                    {SESAM Input Interface File format}
\end{itemize}

\clearpage


%%%%%%%%%%%%%%%%%%%%%%%%%%%%%%%%%%%%%%%%%%%%%%%%%%%%%%%%%%%%%%%%%%%%%%%%%%%%%%%%
\Section{Fedem Technology Link format}{fedem-technology-link-format}

In Release 2.5, Fedem Technology introduced the new \File{.ftl} file format
for the definition of FE parts. This format is designed to be flexible,
powerful, and extensible for adding new entries in the model. The file
is defined in ASCII format and can be easily edited using a text editor.


\SubSection{Syntax}{syntax}

An \File{.ftl} file contains a set of identifiers (nodes, elements, control
data, and attributes) and parameters expressed with the same overall syntax:

\LinkFormatText{identifier\{id value1 \ldots{} valueN \{reference id text\}\}}

\noindent
\begin{tabular}{| m{0.25\textwidth} | m{0.67\textwidth} |}
  \hline
  \rowcolor[HTML]{EFEFEF}
  Name             & Description \\
  \hline\hline
  {\tt identifier} & Specifies field type (e.g., element type, attribute type)\\
  \hline
  {\tt id} & Unique ID for the field (relative to the other fields with
             the same identifier) \\
  \hline
  {\tt value1\ldots valueN} & Primary values for the object
                              (can be text, integers, or decimal digits). \\
  \hline
  {\tt references} & Additional data or other fields can be referred to
                     \newline using this field. \\
  \hline
  {\tt reference and id} & Field reference (field type specified in
                           combination with a valid ID). \\
  \hline
  {\tt text} & Can be used as additional information for a field
               \newline reference or as an optional tag (e.g., a group name) \\
  \hline
\end{tabular}

\medskip
The following are examples:

\LinkFormatText{TET4\{4 22 34 12 32 \{PMAT 1\}\}}

\noindent
A 4-noded tetrahedron element with ID 4, referring to nodes 22, 34, 12 and 32.
The element uses an attribute of type PMAT with ID 1.

\LinkFormatText{PMAT\{1 2.10e+11 8.00e+10 2.90oe-01 7.82e+03\}}

\noindent
A material property entry with ID 1 and four decimal numbers describing
the different parameters in the material.

\Note{All text between a comment symbol ({\tt\#}) and the end of the line
is ignored.}


\SubSection{Nodes}{nodes}

Nodes are expressed by the following syntax:

\LinkFormatText{NODE\{id state x y z\}}

\noindent
\begin{tabular}{| m{0.12\textwidth} | m{0.15\textwidth} | m{0.62\textwidth} |}
  \hline
  \rowcolor[HTML]{EFEFEF}
  Parameter   & Value Type & Description  \\
  \hline\hline
  {\tt id}    & Integer & External node identifier \\
  \hline
  {\tt state} & Integer & Internal/external state flag \\
              &         & \small%\footnotesize
   $=0$ : an internal node condensed out in the reduction \newline
   $=1$ : an external node retained in the reduction \newline
   $<0$ : an internal node with constraints, the value is
   \newline\mbox{}\hskip24pt a binary coding of the fixed DOFs \\
  \hline
  \texttt{x y z} &  Real & Global nodal coordinates \\
  \hline
\end{tabular}


\SubSection{Structural elements}{structural-elements}

Elements are expressed in different ways depending on the element type,
using the statements in the table below.

\noindent
  \begin{tabular}{| m{0.96\textwidth} |}
  \hline
  \rowcolor[HTML]{EFEFEF}
  Element statements \\
  \hline\hline
  {\tt BEAM2\{id n1 n2 \{PMAT pid\}\{PBEAMSECTION gid\}} \\\hskip11mm
  {\tt[\{PORIENT oid\}][\{PBEAMECCENT eid\}][\{PBEAMPIN bpid\}]} \\\hskip11mm
  {\tt[\{PEFFLENGTH lid\}][\{PNSM nid\}]\}} \\
  \hline
  {\tt BEAM3\{id n1 n2 n3 \{PMAT pid\}\{PBEAMSECTION gid\}} \\\hskip11mm
  {\tt[\{PORIENT oid\}|\{PORIENT3 oid\}][\{PBEAMECCENT eid\}]} \\\hskip11mm
  {\tt[\{PBEAMPIN bpid\}][\{PEFFLENGTH lid\}][\{PNSM nid\}]\}} \\
  \hline
  {\tt TRI3\{id n1 n2 n3 \{PMAT pid\}\{PTHICK gid\}[\{PNSM nid\}]\}} \\
  \hline
  {\tt TRI6\{id n1 n2 ... n6 \{PMAT pid\}\{PTHICK gid\}[\{PNSM nid\}]\}} \\
  \hline
  {\tt QUAD4\{id n1 n2 n3 n4 \{PMAT pid\}\{PTHICK gid\}[\{PNSM nid\}]\}} \\
  \hline
  {\tt QUAD8\{id n1 n2 ... n8 \{PMAT pid\}\{PTHICK gid\}[\{PNSM nid\}]\}} \\
  \hline
  {\tt TET4\{id n1 n2 n3 n4 \{PMAT pid\}\}} \\
  \hline
  {\tt TET10\{id n1 n2 ... n10 \{PMAT pid\}\}} \\
  \hline
  {\tt WEDG6\{id n1 n2 ... n6 \{PMAT pid\}\}} \\
  \hline
  {\tt WEDG15\{id n1 n2 ... n15 \{PMAT pid\}\}} \\
  \hline
  {\tt HEX8\{id n1 n2 ... n8 \{PMAT pid\}\}} \\
  \hline
  {\tt HEX20\{id n1 n2 ... n20 \{PMAT pid\}\}} \\
  \hline
  {\tt BUSH\{id n1 n2 [\{PBUSHCOEFF bcid\}][\{PBUSHECCENT beid\}]} \\\hskip9mm
  {\tt[\{PORIENT oid\}|\{PCOORDSYS csid\}]\}} \\
  \hline
  {\tt SPRING\{id n1 n2 \{PSPRING sid\}\}} \\
  \hline
  {\tt RSPRING\{id n1 n2 \{PSPRING sid\}\}} \\
  \hline
  {\tt CMASS\{id n1 [\{PMASS mid\}]\}} \\
  \hline
\end{tabular}

\noindent
\begin{tabular}{| m{0.96\textwidth} |}
  \hline
  \rowcolor[HTML]{EFEFEF}
  Element statements   \\
  \hline\hline
  {\tt RBAR\{id n1 n2 \{PRBAR rid\}\}}\\
  \hline
  {\tt RGD\{id n1 n2 ... nn [\{PRGD rid\}]\}} \\
  \hline
  {\tt WAVGM\{id n1 n2 ... nn \{PWAVGM wid\}\}} \\
  \hline
\end{tabular}

\Note{The identifiers correspond to the element types defined in the
  \FedemTGuide{Appendix A, "Finite Element library"}.}

\Note{The square brackets ([]) denote optional parameters. The vertical
  bar (\textbar) means that either one of the two parameters on each side
  of it may be specified, but not both.}

\Note{All element types listed above may have the reference field
  \{\tt VDETAIL vid\} in addition to the property references listed.}

\Note{The elements {\tt BUSH} and {\tt CMASS} may exist without any associated
  properties ({\tt PBUSHCOEFF} and {\tt PMASS} , respectively)
  in the \File{.ftl} file.
  Such elements are created automatically by Fedem during modeling, e.g.,
  when a mechanism joint is attached to a slave FE node in a part (see
  \refSection{attaching-and-detaching-elements}{Attaching and detaching elements}).
  When the Fedem Reducer encounters such property-less elements, some
  stiffness/mass properties are computed automatically for these elements
  based on the assembled stiffness/mass matrix of the whole part, such that
  the element can be regarded as nearly mass-less and rigid. Refer to the
  \FedemTGuide{Appendix A, "Finite Element library"} for details on this computation.}

Parameters for element statements are given in the table below.

\noindent
\begin{tabular}{| m{0.08\textwidth} | m{0.85\textwidth} |}
  \hline
  \rowcolor[HTML]{EFEFEF}
  Param. & Description  \\
  \hline\hline
  {\tt id} & External element identifier \\
  \hline
  {\tt ni} & Reference to the ith node connected to this element \\
  \hline
  {\tt bcid} & Reference to a stiffness coefficient field for bushing elements \\
  \hline
  {\tt beid} & Reference to an eccentricity field for bushing elements \\
  \hline
  {\tt bpid} & Reference to a pin flag field for beam elements \\
  \hline
  {\tt csid} & Reference to a local coordinate system field \\
  \hline
  {\tt eid} & Reference to an eccentricity field for beam elements \\
  \hline
  {\tt gid} & Reference to a geometric property field for beam and shell elements \\
  \hline
  {\tt lid} & Reference to an effective length field for beam elements \\
  \hline
  {\tt mid} & Reference to a mass property field for concentrated mass elements \\
  \hline
  {\tt nid} & Reference to a non-structural mass field for beam and shell elements \\
  \hline
  {\tt oid} & Reference to an orientation field for this element \\
  \hline
  {\tt pid} & Reference to a material field for this element \\
  \hline
  {\tt rid} & Reference to a component numbers field for rigid elements \\
  \hline
\end{tabular}

\noindent
\begin{tabular}{| m{0.08\textwidth} | m{0.85\textwidth} |}
  \hline
  \rowcolor[HTML]{EFEFEF}
  Parameter & Description  \\
  \hline\hline
  {\tt sid} & Reference to a stiffness matrix field for spring elements.\newline
              For {\tt SPRING} elements it refers to a {\tt PSPRING}
              entry with {\tt type}=1,
              whereas for {\tt RSPRING} elements it refers to a {\tt PSPRING}
              entry with {\tt type}=2. \\
  \hline
  {\tt vid} & Reference to a visibility status field for this element \\
  \hline
  {\tt wid} & Reference to a weight- and component numbers field for
              weighted averaged motion elements \\
  \hline
\end{tabular}


\SubSection{Properties}{properties}

The various properties that are used in the structural element expressions
have the following syntax:

% PMat
\LinkFormatText{PMAT\{pid $E$ $G$ $\nu$ $\rho$\}}

\noindent
\begin{tabular}{| m{0.12\textwidth} | m{0.15\textwidth} | m{0.62\textwidth} |}
  \hline
  \rowcolor[HTML]{EFEFEF}
  Parameter & Value Type & Description \\
  \hline\hline
  {\tt pid} & Integer & Material property identifier \\
  \hline
  {\tt $E$} & Real & Young’s modulus \\
  \hline
  {\tt $G$} & Real & Shear modulus (used by beam elements only) \\
  \hline
  $\nu$     & Real &  Poisson’s ratio \\
  \hline
  $\rho$    & Real &  Material density \\
  \hline
\end{tabular}

% PBeamSection
\LinkFormatText{PBEAMSECTION\{gid a iyy izz ixx ky kz cy cz\}}

\noindent
\begin{tabular}{| m{0.12\textwidth} | m{0.15\textwidth} | m{0.62\textwidth} |}
  \hline
  \rowcolor[HTML]{EFEFEF}
  Parameter & Value Type & Description  \\
  \hline\hline
  {\tt gid} & Integer & Geometric property identifier \\
  \hline
  {\tt a}   & Real & Cross-sectional area \\
  \hline
  {\tt iyy izz} & Real & Moments of inertia about the local
                        $Y$- and $Z$-axes of the beam \\
  \hline
  {\tt ixx}   & Real & Torsional stiffness parameter \\
  \hline
  {\tt ky kz} & Real & Transverse shear reduction factors \\
  \hline
  {\tt cx cz} & Real & Local $Y$- and $Z$-coordinates of the shear center
                of the beam cross section \\
  \hline
\end{tabular}

%PBeamEccent
\LinkFormatText{PBEAMECCENT\{eid ex1 ey1 ez1 ex2 ey2 ez2 [ex3 ey3 ez3]\}}

\noindent
\begin{tabular}{| m{0.18\textwidth} | m{0.15\textwidth} | m{0.56\textwidth} |}
  \hline
  \rowcolor[HTML]{EFEFEF}
  Parameter   & Value Type & Description  \\
  \hline\hline
  {\tt eid}      & Integer & Eccentricity vectors identifier \\
  \hline
  {\tt ex1 ey1 ez1} & Real & Eccentricity vector at local node 1 \\
  \hline
  {\tt ex2 ey2 ez2} & Real & Eccentricity vector at local node 2 \\
  \hline
  {\tt ex3 ey3 ez3} & Real & Eccentricity vector at local node 3 \\
  \hline
\end{tabular}

% PEFFLENGTH
\LinkFormatText{PEFFLENGTH\{lid leff\}}

\noindent
\begin{tabular}{| m{0.13\textwidth} | m{0.15\textwidth} | m{0.61\textwidth} |}
  \hline
  \rowcolor[HTML]{EFEFEF}
  Parameter  & Value Type & Description  \\
  \hline\hline
  {\tt lid}  & Integer & Effective length identifier \\
  \hline
  {\tt leff} & Real    & Effective beam length \\
  \hline
\end{tabular}

% PBEAMPIN
\LinkFormatText{PBEAMPIN\{bpid pa pb\}}

\noindent
\begin{tabular}{| m{0.13\textwidth} | m{0.15\textwidth} | m{0.61\textwidth} |}
  \hline
  \rowcolor[HTML]{EFEFEF}
  Parameter  & Value Type & Description  \\
  \hline\hline
  {\tt bpid} & Integer & Beam pin flag identifier \\
  \hline
  {\tt pa}   & Integer & Local DOFs in end 1 that are released \\
  \hline
  {\tt pb}   & Integer & Local DOFs in end 2 that are released \\
  \hline
\end{tabular}

% PNSM
\LinkFormatText{PNSM\{nid rho flag\}}

\noindent
\begin{tabular}{| m{0.13\textwidth} | m{0.15\textwidth} | m{0.61\textwidth} |}
  \hline
  \rowcolor[HTML]{EFEFEF}
  Parameter & Value Type & Description  \\
  \hline\hline
  {\tt nid} & Integer & Non-structural mass identifier \\
  \hline
  {\tt rho} & Real & The non-structural mass per unit length \newline
                     if {\tt flag}=0, and per unit area if {\tt flag}=1 \\
  \hline
  {\tt flag} & Integer & Flag indicating if this entry is used by a beam or shell \\
  \hline
\end{tabular}

% PTHICK
\LinkFormatText{PTHICK \{gid t\}}

\noindent
\begin{tabular}{| m{0.13\textwidth} | m{0.15\textwidth} | m{0.61\textwidth} |}
  \hline
  \rowcolor[HTML]{EFEFEF}
  Parameter & Value Type & Description \\
  \hline\hline
  {\tt gid} & Integer & Geometric property identifier \\
  \hline
  {\tt t} & Real & Shell thickness \\
  \hline
\end{tabular}

% PBUSHCOEFF
\LinkFormatText{PBUSHCOEFF\{bcid k1 k2 k3 k4 k5 k6\}}

\noindent
\begin{tabular}{| m{0.13\textwidth} | m{0.15\textwidth} | m{0.61\textwidth} |}
  \hline
  \rowcolor[HTML]{EFEFEF}
  Parameter  & Value Type & Description \\
  \hline\hline
  {\tt bcid} & Integer & Bushing coefficients identifier \\
  \hline
  {\tt ki} & Real & Stiffness coefficients in local element directions \\
  \hline
\end{tabular}

% PBUSHECCENT
\LinkFormatText{PBUSHECCENT\{beid ex ey ez\}}

\noindent
\begin{tabular}{| m{0.13\textwidth} | m{0.15\textwidth} | m{0.61\textwidth} |}
  \hline
  \rowcolor[HTML]{EFEFEF}
  Parameter & Value Type & Description  \\
  \hline\hline
  {\tt beid} & Integer & Eccentricity vector identifier \\
  \hline
  {\tt ex ey ez} & Real & Offset vector from local node 1 to the
                          bushing element location \\
  \hline
\end{tabular}

\clearpage
% PORIENT
\LinkFormatText{PORIENT\{oid ox oy oz\}}

\noindent
\begin{tabular}{| m{0.13\textwidth} | m{0.15\textwidth} | m{0.61\textwidth} |}
  \hline
  \rowcolor[HTML]{EFEFEF}
  Parameter & Value Type & Description \\
  \hline\hline
  {\tt oid} & Integer & Orientation vector identifier \\
  \hline
  {\tt ox oy oz} & Real & Local z-axis of the element \\
  \hline
\end{tabular}

% PORIENT3
\LinkFormatText{PORIENT3\{oid ox1 oy1 oz1 ox2 oy2 oz2 ox3 oy3 oz3\}}

\noindent
\begin{tabular}{| m{0.18\textwidth} | m{0.15\textwidth} | m{0.56\textwidth} |}
  \hline
  \rowcolor[HTML]{EFEFEF}
  Parameter & Value Type & Description \\
  \hline\hline
  {\tt oid} & Integer & Orientation vector identifier \\
  \hline
  {\tt ox1 oy1 oz1} & Real & Local $Z$-axis at element node 1 \\
  \hline
  {\tt ox2 oy2 oz2} & Real & Local $Z$-axis at element node 2 \\
  \hline
  {\tt ox3 oy3 oz3} & Real & Local $Z$-axis at element node 3 \\
  \hline
\end{tabular}

% PCOORDSYS
\LinkFormatText{PCOORDSYS\{csid ox oy oz zx zy zz px py pz\}}

\noindent
\begin{tabular}{| m{0.13\textwidth} | m{0.15\textwidth} | m{0.61\textwidth} |}
  \hline
  \rowcolor[HTML]{EFEFEF}
  Parameter  & Value Type & Description \\
  \hline\hline
  {\tt csid} & Integer & Local coordinate system identifier \\
  \hline
  {\tt ox oy oz} & Real & Origin of the local coordinate system \\
  \hline
  {\tt zx zy zz} & Real & Local $Z$-axis in the coordinate system \\
  \hline
  {\tt px py pz} & Real & Point in the local $XZ$-plane \\
  \hline
\end{tabular}

% PSPRING
\LinkFormatText{PSPRING\{sid k11 k21 k22 k31 k32 k33 k41 k42 k43 k44
\newline\mbox{}\hskip22mm k51 k52 k53 k54 k55 k61 k62 k63 k64 k65 k66 type\}}

\noindent
  \begin{tabular}{| m{0.13\textwidth} | m{0.15\textwidth} | m{0.61\textwidth} |}
  \hline
  \rowcolor[HTML]{EFEFEF}
  Parameter  & Value Type & Description \\
  \hline\hline
  {\tt sid}  & Integer & Spring stiffness matrix identifier \\
  \hline
  {\tt kij}  & Real & Component $(i,j)$ of the symmetric $6\times6$
               spring stiffness matrix \\
  \hline
  {\tt type} & Integer & Spring type flag \newline
  (1 = Translatory spring, 2 = Rotational spring) \\
  \hline
\end{tabular}

% PMASS
\LinkFormatText{PMASS\{mid I11 I21 I22 I31 I32 I33 I41 I42 I43 I44
\newline\mbox{}\hskip18.4mm I51 I52 I53 I54 I55 I61 I62 I63 I64 I65 I66\}}

\noindent
\begin{tabular}{| m{0.13\textwidth} | m{0.15\textwidth} | m{0.61\textwidth} |}
  \hline
  \rowcolor[HTML]{EFEFEF}
  Parameter & Value Type & Description \\
  \hline\hline
  {\tt mid} & Integer & Mass property identifier \\
  \hline
  {\tt Iij} & Real & Component $(i,j)$ of the symmetric point-mass matrix \\
  \hline
\end{tabular}

\clearpage
% PRBAR
\LinkFormatText{PRBAR\{rid cn1 cn2 cm1 cm2\}}

\noindent
\begin{tabular}{| m{0.13\textwidth} | m{0.15\textwidth} | m{0.61\textwidth} |}
  \hline
  \rowcolor[HTML]{EFEFEF}
  Parameter & Value Type & Description \\
  \hline\hline
  {\tt rid} & Integer & Rigid bar property identifier \\
  \hline
  {\tt cn1 cn2} & Integer & Component numbers of independent DOFs in
                            the part coordinate system for the element
                            at end 1 and 2, respectively \\
  \hline
  {\tt cm1 cm2} & Integer & Component numbers of dependent DOFs in
                            the part coordinate system assigned by the element
                            at end 1 and 2, respectively \\
  \hline
\end{tabular}

% PRGD
\LinkFormatText{PRGD \{rid cm\}}

\noindent
\begin{tabular}{| m{0.13\textwidth} | m{0.15\textwidth} | m{0.61\textwidth} |}
  \hline
  \rowcolor[HTML]{EFEFEF}
  Parameter & Value Type & Description \\
  \hline\hline
  {\tt rid} & Integer & Rigid body property identifier \\
  \hline
  {\tt cm}  & Integer & Component numbers of the dependent DOFs in
                        the part coordinate system at all slave nodes of
                        the rigid element \\
  \hline
\end{tabular}

% PWAVGM
\LinkFormatText{PWAVGM\{wid rc x1...x6 w11...w1n w21...w2n ...\}}

\noindent
\begin{tabular}{| m{0.13\textwidth} | m{0.15\textwidth} | m{0.61\textwidth} |}
  \hline
  \rowcolor[HTML]{EFEFEF}
  Parameter & Value Type & Description \\
  \hline\hline
  {\tt wid} & Integer & Weighted averaged motion property identifier \\
  \hline
  {\tt rc}  & Integer & Component numbers of the dependent DOFs at the slave
                        node of the weighted average motion element. \\
  \hline
  {\tt xi}  & Integer & Row index into the weighting matrix for local
                        DOF $i$ at the slave node \\
  \hline
  {\tt wij} & Real & Weight factor at master node $j$ for the element,
                     for all slave DOFs whose row index equals $i$ \\
  \hline
\end{tabular}

% VDETAIL
\LinkFormatText{VDETAIL\{vid visible\}}

\noindent
\begin{tabular}{ | m{0.13\textwidth} | m{0.15\textwidth} | m{0.61\textwidth} |}
  \hline
  \rowcolor[HTML]{EFEFEF}
  Parameter     & Value Type & Description \\
  \hline\hline
  {\tt vid}     & Integer & Visibility status identifier \\
  \hline
  {\tt visible} & Integer & Visibility flag (value 0 means invisible element
                            faces whereas 1 means visible element faces) \\
  \hline
\end{tabular}

\clearpage


\SubSection{Loads}{ftl-loads}

Both concentrated point loads on nodes and distributed surface loads on shell
and solid elements are supported. The \File{.ftl} syntax is as follows:

\medskip{\tt\noindent
CFORCE\{sid fx fy fz n$_1$ ... n$_n$\} \\[5pt]\noindent
CMOMENT\{sid mx my mz n$_1$ ... n$_n$\} \\[5pt]\noindent
SURFLOAD\{sid p$_1$ ... p$_n$ e$_1$ ... e$_n$ [\{PORIENT oid\}]\}\\[5pt]\noindent
FACELOAD\{sid p$_1$ ... p$_n$ e$_1$ f$_1$ ... e$_n$ f$_n$ [\{PORIENT oid\}]\}}

\medskip\noindent
\begin{tabular}{| m{0.13\textwidth} | m{0.15\textwidth} | m{0.61\textwidth} |}
  \hline
  \rowcolor[HTML]{EFEFEF}
  Parameter & Value Type & Description \\
  \hline\hline
  {\tt sid}   & Integer & Load set identifier \\
  \hline
  {\tt fx fy fz} & Real & Global force components \\
  \hline
  {\tt mx my mz} & Real & Global torque components \\
  \hline
  {\tt p$_i$} & Real    & Surface force intensity in local node $i$ \\
  \hline
  {\tt n$_i$} & Integer & Node IDs of load target points \\
  \hline
  {\tt e$_i$} & Integer & Element IDs \\
  \hline
  {\tt f$_i$} & Integer & Local face numbers \\
  \hline
\end{tabular}


\SubSection{Strain Coat Elements}{strain-coat-lements}

Four strain coat element types are supported, where the type names
reflect the number of element nodes. The \File{.ftl} syntax is as follows:

\medskip{\tt\noindent
STRCT3\{id n$_1$ n$_2$ n$_3$
\{PSTRC pid$_1$\} ... \{PSTRC pid$_n$\} \{FE eid\}\} \\[5pt]\noindent
STRCT6\{id n$_1$ n$_2$ ... n$_6$
\{PSTRC pid$_1$\} ... \{PSTRC pid$_n$\} \{FE eid\}\} \\[5pt]\noindent
STRCQ4\{id n$_1$ n$_2$ n$_3$ n$_4$
\{PSTRC pid$_1$\} ... \{PSTRC pid$_n$\} \{FE eid\}\} \\[5pt]\noindent
STRCQ8\{id n$_1$ n$_2$ ... n$_8$
\{PSTRC pid$_1$\} ... \{PSTRC pid$_n$\} \{FE eid\}\}}

\medskip\noindent
\begin{tabular}{| m{0.13\textwidth} | m{0.79\textwidth} |}
  \hline
  \rowcolor[HTML]{EFEFEF}
  Parameter & Description \\
  \hline\hline
  {\tt id}  & External element identifier \\
  \hline
  {\tt n$_i$} & Reference to the ith node connected to this element \\
  \hline
  {\tt pid$_i$} & Reference to the property field of the $i^{th}$
                  calculation point for this element \\
  \hline
  {\tt eid} & Reference to the underlying structural FE element \\
  \hline
\end{tabular}


\SubSection{Strain Coat Properties}{strain-coat-properties}

The property fields that are referred to by the strain coat element
fields have the following syntax:

\LinkFormatText{PSTRC\{pid name \{PMAT mid\}[\{PTHICKREF tid\}|\{PHEIGHT hid\}]\}}

\noindent
\begin{tabular}{| m{0.13\textwidth} | m{0.15\textwidth} | m{0.61\textwidth} |}
  \hline
  \rowcolor[HTML]{EFEFEF}
  Parameter  & Value Type & Description \\
  \hline\hline
  {\tt pid}  & Integer & Strain coat property identifier \\
  \hline
  {\tt name} & String & Result set name to be displayed in the animation UI
                        (one of {\sl Top}, {\sl Bottom} or {\sl Basic}) \\
  \hline
  {\tt mid}  & Integer & Reference to a material property field \\
  \hline
  {\tt tid}  & Integer & Reference to a thickness-relative z-position field \\
  \hline
  {\tt hid}  & Integer & Reference to an absolute z-position field \\
  \hline
\end{tabular}

% PTHICKREF
\LinkFormatText{PTHICKREF\{tid fact \{PTHICK gid\}\}}

\noindent
\begin{tabular}{| m{0.13\textwidth} | m{0.15\textwidth} | m{0.61\textwidth} |}
  \hline
  \rowcolor[HTML]{EFEFEF}
  Parameter  & Value Type & Description \\
  \hline\hline
  {\tt tid}  & Integer & z-position identifier \\
  \hline
  {\tt fact} & Real & Location of the calculation point in the thickness
                      direction of a shell as a fraction of the referenced
                      shell thickness, i.e., the z-position is $z={\tt fact}*t$ \\
             &      & where $t$ is the thickness referenced through the
                      parameter {\tt gid}. \\
  \hline
  {\tt gid}  & Integer & Reference to a thickness field \\
  \hline
\end{tabular}

% PHEIGHT
\LinkFormatText{PHEIGHT \{hid h\}}

\noindent
\begin{tabular}{| m{0.13\textwidth} | m{0.15\textwidth} | m{0.61\textwidth} |}
  \hline
  \rowcolor[HTML]{EFEFEF}
  Parameter & Value Type & Description \\
  \hline\hline
  {\tt hid} & Integer & $Z$-position identifier \\
  \hline
  {\tt h} & Real & Absolute location of the calculation point in the
                   thickness direction of a shell, i.e.,
                   the $Z$-position is $z=h$ \\
  \hline
\end{tabular}


\SubSection{Other identifiers}{other-identifiers}

The following identifiers are used to define element groups:

\LinkFormatText{GROUP\{id e1 e2 ... en \{NAME name\}\}}

\noindent
\begin{tabular}{| m{0.13\textwidth} | m{0.15\textwidth} | m{0.61\textwidth} |}
  \hline
  \rowcolor[HTML]{EFEFEF}
  Parameter & Value Type & Description \\
  \hline\hline
  {\tt id} & Integer & Group identifier \\
  \hline
  {\tt ei} & Integer & ID of the $i^{th}$ element in this group \\
  \hline
  {\tt name} & String & Name tag of this group \\
  \hline
\end{tabular}

\clearpage


%%%%%%%%%%%%%%%%%%%%%%%%%%%%%%%%%%%%%%%%%%%%%%%%%%%%%%%%%%%%%%%%%%%%%%%%%%%%%%%%
\Section{Nastran Bulk Data File format}{nastran-bulk-data-file-format}

Fedem supports a wide range of Nastran bulk data entries (see table below).
For most element types, implicit conversion to a known Fedem element type is
performed. Refer to Nastran's Bulk Data File documentation for details about
properties and syntax for each entry, and the
\FedemTGuide{Appendix A, "Finite Element Library"}
for details about the Finite Element library in Fedem.

FE models in {\sl Nastran bulk data file} format can be
directly imported as parts into Fedem using the Import Part command
(see \refSection{creating-parts-by-file-import}{Creating parts by file import}).

\Note{Each FE model to be imported into Fedem must be stored in a separate bulk
  data file with the Bulk Data File extension (\File{.nas} or \File{.bdf}).}

\begin{tabular}{| m{0.25\textwidth} |  m{0.5\textwidth} |}
  \hline
  \rowcolor[HTML]{EFEFEF} Nastran Bulk      & Comments \\
  \rowcolor[HTML]{EFEFEF} Data Conversation &          \\
  \hline\hline
  {\tt BAROR}  & \\\hline
  {\tt BEAMOR} & \\\hline
  {\tt CBAR}   & \\\hline
  {\tt CBEAM}  & Same as {\tt CBAR} \\\hline
  {\tt CBUSH}  & \\\hline
  {\tt CELAS1} & \\\hline
  {\tt CELAS2} & Same as {\tt CELAS1} \\\hline
  {\tt CHEXA}  & Supports both 8 and 20 nodes \\\hline
  {\tt CONM1}  & \\\hline
  {\tt CONM2}  & Same as {\tt CONM1} \\\hline
  {\tt CONROD} & \\\hline
  {\tt CORD1C} & \\\hline
  {\tt CORD1R} & \\\hline
  {\tt CORD1S} & \\\hline
  {\tt CORD2C} & \\\hline
  {\tt CORD2R} & \\\hline
  {\tt CORD2S} & \\\hline
  {\tt CPENTA} & Supports both 6 and 15 nodes \\\hline
  {\tt CQUAD4} & \\\hline
  {\tt CQUAD8} & \\\hline
  {\tt CROD}   & Same as {\tt CONROD} \\\hline
  {\tt CTETRA} & Supports both 4 and 10 nodes \\\hline
  {\tt CTRIA3} & \\\hline
  {\tt CTRIA6} & \\\hline
\end{tabular}

\begin{tabular}{| m{0.25\textwidth} |  m{0.5\textwidth} |}
  \hline
  \rowcolor[HTML]{EFEFEF} Nastran Bulk      & Comments \\
  \rowcolor[HTML]{EFEFEF} Data Conversation &          \\
  \hline\hline
  {\tt CWELD}  & \\\hline
  {\tt CORD2R} & \\\hline
  {\tt CORD2S} & \\\hline
  {\tt CPENTA} & Supports both 6 and 15 nodes \\\hline
  {\tt CQUAD4} & \\\hline
  {\tt CQUAD8} & \\\hline
  {\tt CROD}   & Same as {\tt CONROD} \\\hline
  {\tt CTETRA} & Supports both 4 and 10 nodes \\\hline
  {\tt CTRIA3} & \\\hline
  {\tt CTRIA6} & \\\hline
  {\tt CWELD}  & \\\hline
  {\tt FORCE}  & \\\hline
  {\tt GRDSET} & \\\hline
  {\tt GRID}   & \\\hline
  {\tt INCLUDE} & \\\hline
  {\tt MAT1}   & See below \\\hline
  {\tt MOMENT} & \\\hline
  {\tt PBAR}   & \\\hline
  {\tt PBARL}  & See below \\\hline
  {\tt PBEAM}  & \\\hline
  {\tt PBEAML} & See below \\\hline
  {\tt PBUSH}  & \\\hline
  {\tt PELAS}  & \\\hline
  {\tt PLOAD2} & \\\hline
  {\tt PLOAD4} & \\\hline
  {\tt PROD}   & \\\hline
  {\tt PSHELL} & \\\hline
  {\tt PSOLID} & \\\hline
  {\tt PWELD}  & \\\hline
  {\tt RBAR}   & \\\hline
  {\tt RBE2}   & \\\hline
  {\tt RBE3}   & \\\hline
  {\tt ASET}   & Automatic definition of external nodes \\\hline
  {\tt ASET1}  & Same as ASET \\\hline
  {\tt SPC}    & See below \\\hline
  {\tt SPC1}   & See below \\\hline
\end{tabular}

\Note{For the {\tt PBARL}/{\tt BPEAML} entries,
  Fedem currently supports the following cross section types
  (see the MSC/Nastran Reference guide for details):
  {\tt ROD}, {\tt TUBE}, {\tt BAR}, {\tt BOX}, {\tt I} and {\tt T}.
  Any other cross section types have to be manually replaced by equivalent
  {\tt PBAR}/{\tt PBEAM} entries.}

\Note{The shear modulus in the {\tt MAT1} bulk entry is only used by beam
  elements. If the value on the Nastran file is zero for a {\tt MAT1} entry that
  is used by a beam element, the shear modulus is automatically recomputed from
  the Young's modulus and Poisson's ratio through the formula $G=E/(2+2\nu)$.
  However, if $G=0$ is desired for a beam element, that is still possible by
  editing the \File{.ftl} file created in the \File{link\_DB} directory when
  the model is saved.}

\Note{If the Poisson's ratio in the MAT1 bulk entry is not given or is outside
  the valid range {\rm[0, 0.5\textgreater}, but the shear modulus is given,
  the Poisson's ratio will be derived from the Young's modulus and the
  shear modulus, through the expression $\nu=E/2G-1$, if that yields a
  value within the valid range.}

\Note{The {\tt SPC} (and {\tt SPC1}) bulk entry is used to specify fixed DOFs in
  FE nodes and is an alternative to specify the node as external and constrain
  it on the system level. Non-zero prescribed values are h owever treated as
  zero in Fedem (always fixed).}

\clearpage


%%%%%%%%%%%%%%%%%%%%%%%%%%%%%%%%%%%%%%%%%%%%%%%%%%%%%%%%%%%%%%%%%%%%%%%%%%%%%%%%
\Section{SESAM Input Interface File format}{sesam-input-interface-file-format}

Fedem supports the SESAM input interface file data entries as shown in the table
below.

\medskip
\begin{tabular}{| m{0.22\textwidth} | m{0.6\textwidth} |}
  \hline
  \rowcolor[HTML]{EFEFEF} SESAM keyword  &  Comments  \\
  \hline\hline
  {\tt BELFIX}   & \\\hline
  {\tt BLDEP}    & \\\hline
  {\tt BNBCD }   & Only the {\sl Fixed}(1) and {\sl Free}(0) status codes are \\
                 & considered in Fedem. Other values are ignored. \\\hline
  {\tt BNMASS}   & \\\hline
  {\tt GBEAMG}   & \\\hline
  {\tt GCOORD}   & \\\hline
  {\tt GECCEN}   & \\\hline
  {\tt GELMNT1}  & See table below for supported element types \\\hline
  {\tt GELREF1}  & \\\hline
  {\tt GELTH}    & \\\hline
  {\tt GSETMEMB} & \\\hline
  {\tt GUNIVEC}  & \\\hline
  {\tt MGSPRNG}  & \\\hline
  {\tt MISOSEL}  & \\\hline
  {\tt TDMATER}  & \\\hline
  {\tt TDSECT}   & \\\hline
  {\tt TDSETNAM} & \\\hline
\end{tabular}

\clearpage

The following SESAM element types are supported:

\medskip
\begin{tabular}{| m{0.25\textwidth} |  m{0.6\textwidth} |}
  \hline
  \rowcolor[HTML]{EFEFEF} SESAM type name & Equivalent FEDEM type name \\
  \hline\hline
  {\tt BEPS} & {\tt BEAM2}   \\\hline
  {\tt BEAS} & {\tt BEAM2}   \\\hline
  {\tt SECB} & {\tt BEAM3}   \\\hline
  {\tt BTSS} & {\tt BEAM3} subdivided to two {\tt BEAM2} \\\hline
  {\tt GSPR} & {\tt RSPRING} \\\hline
  {\tt CSTA} & {\tt TRI3}    \\\hline
  {\tt FTRS} & {\tt TRI3}    \\\hline
  {\tt LQUA} & {\tt QUAD4}   \\\hline
  {\tt FQUS} & {\tt QUAD4}   \\\hline
  {\tt ILST} & {\tt TRI6}    \\\hline
  {\tt SCTS} & {\tt TRI6}    \\\hline
  {\tt IQQE} & {\tt QUAD8}   \\\hline
  {\tt SCQS} & {\tt QUAD8}   \\\hline
  {\tt ITET} & {\tt TET10}   \\\hline
  {\tt IPRI} & {\tt WEDG15}  \\\hline
  {\tt IHEX} & {\tt HEX20}   \\\hline
  {\tt TETR} & {\tt TET4}    \\\hline
  {\tt TPRI} & {\tt WEDG6}   \\\hline
  {\tt LHEX} & {\tt HEX8}    \\\hline
  {\tt GMAS} & {\tt CMASS}   \\\hline
\end{tabular}
